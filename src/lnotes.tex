\chapter{逻辑}

\section{形式逻辑简介}

\index{形式逻辑}
英国哲学家弗兰西斯·培根说:“逻辑与修辞使人善辩。”从逻辑学的发展史看,逻辑学是一
门研究思维规律的科学,有人把逻辑称为“思维的语法”。而其中的形式逻辑研究的对象是
思维的逻辑形式、基本规律及简单的逻辑方法。思维的逻辑形式包括概念、判断和推理三
种形式。学习逻辑可以提高人们的逻辑思维能力,指导人们正确地进行思维活动,除此之
外,学习逻辑还有助于提高语言表达能力,有助于减少或避免思维表达中的谬误,有助于
识别和驳斥诡辩的议论。

\subsection{逻辑}

\index{逻辑}

“逻辑”一词是英文 Logic 的音译,而英文 Logic 又源于希腊文 λ\'ογος (逻各斯),其
原意是指思想、言辞、理性、规律性等。近代思想家严复最先把 Logic 译成“逻辑”,但他
并没有加以推广和倡导,而是用“名学”作为他逻辑学著作的书名。到了二十世纪三四十年
代,“逻辑”的译名才逐渐流行起来,并被我国逻辑学界接受,后来通称为“逻辑学”。

“逻辑”常见的含义有以下四种:

\begin{enumerate}
  \item 思维的规律。例如:这几句话不合逻辑。

  \item 客观的规律性。例如:生活的逻辑。

  \item 某种理论、观点。例如:侵略者奉行的是强盗逻辑。

  \item 逻辑学。例如:写好议论文,学点逻辑是必要的。
\end{enumerate}

\subsection{逻辑思维}

\index{逻辑思维}
逻辑思维指人在认识过程中借助于概念、判断、推理反映现实的思维方式。它以抽象性为
特征,撇开具体形象,解释事物的本质属性。也叫抽象思维。

正确运用逻辑工具,可以帮助人们更好地认识事物、寻求真理,也有助于我们提高思维能
力,正确表达思想。例如根据莎士比亚《威尼斯商人》中的情节,有人编了一道推理题:
\begin{equotation}
  女主人公鲍西娅对求婚者说:“这里有三个盒子,每个盒子上写着一句话。三句话中只有
  一句是真话。谁能够猜中我的肖像放在哪个盒子里,谁就可以做我的意中人。”金盒子上
  写的是:“肖像在这个盒子里。”银盒子上写的是:“肖像不在这个盒子里。”铅盒子上写
  的是:“肖像不在金盒子里。”哪位求婚者猜中了,鲍西娅就嫁给他。
\end{equotation}
有一位求婚者利用逻辑学的知识很快猜中了。金盒子上刻的话是“肖像在这个盒子里”,铅
盒子上刻的话是“肖像不在金盒子里”。从逻辑上看这两句话是相互矛盾的,它们表示两个
相互矛盾的判断。排中律规定:两个相互矛盾的判断不能同假,必有一真。据给的条件之
一,即三句话中只有一句话是真话,这样就可以断定这句唯一的真话,必定是金盒或铅盒
上刻着的话,银盒上刻的话不可能是真话。而银盒子上刻的话“肖像不在这个盒子里”既然
是假的,那么按排中律的要求,就可确定“肖像不在金盒子里”是真的,从而断定肖像在银
盒子中。

\subsection{逻辑学}

\index{逻辑学} \index{名学} \index{辩学} \index{论理学}
逻辑学是哲学的一个分支,研究思维的形式和规律。旧称“名学”,“辩学”,“论理学”。

逻辑学是一门古老的科学,有两千多年的历史,主要有三个发源地:古代中国、古代印度
和古代希腊。

中国在春秋战国时代逻辑思想大为发展,出现了惠施、公孙龙、墨子、韩非子、荀况等逻
辑学家。如墨子及其后人所著《墨经》论述了概念、判断、推理等各种思维形式,较为系
统地阐述了关于思维的基本规律内容。例如《墨子·非攻》中有一段论证翻译成现代汉语,
是这样的:
\begin{equotation}
  如果有一个人,进人家果园,偷人家桃李,大家听说后就谴责他,上面执政的人就捉获
  并惩罚他。这是为什么呢?因为他损人利己。至于偷人家鸡犬猪的,比进人家果园偷桃
  李更不义。这是什么缘故呢?因为他损人更多。如果损人越多,那么他越是不仁,罪越
  重。至于进入人家牲口棚,牵走人家马牛的,比偷人家鸡犬猪更不义。这是什么原故呢?
  因为他损人更多。如果损人越多,那么他越是不仁,罪越重。至于杀无辜的人,剥下他
  们的衣服皮袄,拿走戈剑,这比进人家牲口棚牵走马牛又更不义。这是什么缘故呢?因
  为他损人更严重。如果损人越严重,那么他就越不仁,罪越大。对此,世上有道德之人
  都明白其中的道理并会认为这种做法不对,说这些是不义的。而今最不义的事就是进攻
  别国,却不知道谴责,反而称赞它,说它是义。这能说是知道义与不义的分别吗?
\end{equotation}
这段论证就是归纳推理与类比推理的结合。既用归纳推理得出了“损人利己之事都是不义
的”,又用“入人园圃,窃其桃李”等四件事类比攻国。墨子把几种推理方法结合在一起的逻
辑推理不仅罕为墨子之前的学者所用,对于现代人进行推理也有借鉴意义。

\subsection{逻辑学分类}

形式逻辑和辩证逻辑是逻辑学中的两大基本门类,此外,还有数理逻辑。

\subsubsection{形式逻辑}

\index{形式逻辑}
形式逻辑是研究思维形式、基本规律和一些逻辑方法的科学。

形式逻辑研究概念、判断、推理等主要思维形式,研究同一律、矛盾律、排中律等思维规
律。形式逻辑研究的逻辑方法包括:定义、划分、限制、概括、寻求因果联系的方法、假
说及论证等。

形式逻辑具体分为两种:
\begin{enumerate}
  \item 传统形式逻辑。指用自然语言表述的演绎逻辑和归纳逻辑。

  \item 现代形式逻辑。它又称符号逻辑,主要是指人工符号语言表述的数理逻辑,以及
    模态逻辑、多值逻辑、认识逻辑、时态逻辑等。
\end{enumerate}

\subsubsection{辩证逻辑}

\index{辩证逻辑}
辩证逻辑是马克思主义哲学的组成部分,要求人们必须把握、研究事物的总和,从事物本
身矛盾的发展、运动、变化来观察它,把握它,只有这样,才能认识客观世界的本质。例
如刘安的《淮南子·人间训》中“塞翁失马”的故事就生动揭示了福祸相依、福祸转化的辩证
关系。

\subsubsection{数理逻辑}

\index{数理逻辑}
数理逻辑是数学的一个分支,用数学方法研究推理、计算等逻辑问题。也叫符号逻辑。

\subsection{学习形式逻辑的方法}

理论联系实际是有效的学习方法,特别是与写作的实际结合起来。留心实际生活中遇到的
各种逻辑问题,并自觉地用学过的逻辑理论和知识去分析和解决,逐渐养成经常进行逻辑
分析的习惯。例如:
\begin{equotation}
  某学校贴出通知:出入校门请出示工作证和学生证。
\end{equotation}
出入校门须同时出示工作证和学生证,无论是学生还是教师都不能同时具有这两种证件。
从逻辑学来分析,写通知的人误把选言判断写成了联言判断,因此,上述联言判断应改为
选言判断,即“出入校门请出示工作证或学生证”。

这个逻辑学例子说明,理论联系实际的学习原则和方法,是学好形式逻辑的关键。

\section{概念}

\index{概念}
概念是通过揭示对象的特性或本质来反映对象的一种思维形式。人类在认识过程中,把所
感觉到的事物的共同特点抽出来,加以概括,就成为概念。比如从白雪、白马、白纸等事
物中抽出它们的共同特点,就得出“白”的概念。

\subsection{概念的内涵}

\index{概念的内涵}
概念的内涵是指反映在概念中的对象的本质属性。它指的是“什么是”,反映概念质的方面。

例如,“法律”的内涵:由立法机关或国家机关制定,国家政权保证执行的行为规则的总和。

\index{概念的外延}
\subsection{概念的外延}

概念的外延是指具有概念所反映的本质属性的对象。它指的是“哪些是”,反映概念量的方
面。

例如,“法律”的外延:凡是具有法律的本质属性的一切事物,即古今中外的所有法律。

\subsection{概念的分类}

\subsubsection{普遍概念、单独概念}

\index{普遍概念}
普遍概念:指反映某一类事物的概念,它的外延不是由一个单独的分子构成的,而是由两
个以上乃至许许多多的分子组成的类。例如:“城市”,“小说”,“书”等。

\index{单独概念}
单独概念:指反映某一个事物的概念,它的外延仅指一个单独的对象。例如:“黄河”,“曹
雪芹”,“七七事变”。

\subsubsection{集合概念、非集合概念}

\index{集合概念}
集合概念:以事物的集合体为反映对象的概念。例如:“森林”,“经济联合体”等。

\index{非集合概念}
非集合概念:不反映对象群体属性的概念。例如:“树”,“书”等。

了解一个概念是集合概念还是非集合概念,不能脱离具体语境。在思维中容易把一个普遍
名词表达的集合概念与非集合概念相混淆,造成逻辑错误。例如:
\begin{equotation}
  鲁迅的作品不是一天能读完的,

  《祝福》是鲁迅的作品,

  所以,《祝福》不是一天能读完的。
\end{equotation}
前一个“鲁迅的作品”是集合概念,后一个“鲁迅的作品”是非集合概念,这样“鲁迅的作品”
没有起到推理的中介作用,所以上述推理是错误的。

\subsection{明确概念的方法}

\subsubsection{定义}

\index{定义}
这是揭示概念所反映的对象的特点或本质的一种逻辑方法。通常下定义的简单公式是:被
定义项=种差+临近的属概念。例如给“商品”下定义:“商品是用来交换的劳动产品”,其中
“用来交换”这一性质就是区别“商品”与一切其他劳动产品的种差。“劳动产品”这个概念是
属概念。

定义不应包括含混的概念,不能用隐喻,不应当是否定的。

\subsubsection{划分}

\index{划分}
切分是将一个概念所反映的一类事物,按照某个或某些性质分为若干个小类,例如将“民事
诉讼证据”划分为书证、物证、视听资料、证人证言、当事人陈述、鉴定结论及勘验笔录七
个并列的子项。划分必须按照统一标准进行。

\section{判断}

\index{判断}
判断是对思维对象有所断定(肯定或者否定)的思维形式。例如:
\begin{equotation}
  历史绝不是少数帝王将相的历史。
\end{equotation}

\subsection{判断的特征}

判断的特征主要有两个:判断对事物情况有所断定;判断有真假之分。

\subsection{判断的种类}

判断一般可以分为简单判断和复合判断。

\subsubsection{简单判断}

不包含其他判断的判断。简单判断有两种,一种是性质判断,另一种是关系判断。

\subparagraph{性质判断}就是断定某事物具有(或不具有)某性质的判断。
\begin{equotation}
  屈原是伟大的诗人。
\end{equotation}
本句为性质判断,它断定了屈原“伟大的诗人”的性质。在这个性质判断中,“屈原”是主项,
“伟大的诗人”是谓项。“是”是联项。

\index{周延}
项的周延性,就是指在性质判断中对主项、谓项外延数量的断定情况。如果在一个判断中,
对它的主项(或谓项)的全部外延作了断定,那么,这个判断的主项(或谓项)就是周延
的;如果未对主项(或谓项)的全部外延作断定,那么,这个判断的主项(或谓项)就是
不周延的。例如:
\begin{equotation}
  一切师范大学都是培养教师的学校。

  有的师范大学不是面向全国招生的。
\end{equotation}
就主项来说,前一个判断对“师范大学”这个概念的全部外延都作了断定,所以“师范大学”
就是周延的;相反,后一个判断由于并未对“师范大学”的全部外延作出断定,因而这个判
断的主项“师范大学”就是不周延的。

再就谓项来说,由于前一个判断只是断定了“师范大学”的全部外延都包含在“培养教师的学
校”的外延中,并没有断定“培养教师的学校”的全部外延都包含在“师范大学”的全部外延中,
因而谓项“培养教师的学校”就是不周延的;相反,由于后一个判断断定了“面向全国招生的”
这个概念的全部外延与“师范大学”的部分外延是互相排斥的,因而谓项“面向全国招生的”
就是周延的。

\subparagraph{关系判断}就是断定事物与事物之间的关系的判断。例如:
\begin{equotation}
  李白与杜甫是朋友。
\end{equotation}

\subsubsection{复合判断}

复合判断是由若干个简单判断通过一定的逻辑联结词组合而成的,主要包括:假言判断、
选言判断、联言判断。

\index{假言判断}
\subparagraph{假言判断}就是断定某一事物情况存在是另一事物情况存在的条件的判断。
例如:
\begin{equotation}
  假如语言能够生产物质资料,那么夸夸其谈的人就会成为世界上最富的人了。(充分条
  件假言判断)

  只有认识了自己的缺点,才能改正自己的缺点。(必要条件假言判断)

  当且仅当一个三角形等角,它才等边。(充分必要假言判断)
\end{equotation}

假言判断在生活中使用比较广泛。例如杜牧在《阿房宫赋》中曾发出这样的感叹:
\begin{equotation}
  使六国各爱其人,则足以拒秦;使秦复爱六国之人,则递三世可至万世而为君,谁得而
  族灭也?
\end{equotation}
这段话翻译成现代汉语,意思是:假使六国各自爱护自己的人民,就完全可以依靠人民来
抵抗秦国。假使秦王朝又爱护六国的人民,那么秦可以传到三世以至万世而为王,谁能够
族灭它呢?

这里使用了充分条件假言判断,表达了对六国和秦灭亡的原因的判断:不爱其民。这给当
时的最高统治者敲响了警钟。

\index{选言判断}
\subparagraph{选言判断}就是断定在几种事物情况之中至少有一种事物情况存在的判断。
例如:
\begin{equotation}
  你说错了或者我听错了。(选言肢相互包容)

  或者你去,或者我去。(选言肢不包容)
\end{equotation}

\subparagraph{联言判断}就是断定几种事物情况都存在的判断。例如:

\index{联言判断}
\begin{equotation}
  苏轼是文学家,也是画家。(并列)

  劳动人民不但创造了物质财富,而且创造了精神财富。(递进)

  流水线提升了生产效率,但是也要防止机器对人的异化。(转折)
\end{equotation}

\section{推理}

\index{推理}
推理是从一个或几个已知判断推出未知新判断的思维形式。任何推理都由两部分组成:推
理所依据的判断以及推出的新判断。前者叫前提,后者叫结论。

推理也是同语言联系在一起的,推理在语言上表现为复句或句群。在这类复句或句群中,
如果有“因为……所以……”,“由于……因此……”,“由此可见”等关联词语,则往往表达推理。

要正确地运用推理,就必须使推理具有逻辑性。一个正确的推理应具备两个条件:一是前
提真实,即应当是正确反映客观事物情况的真实命题;二是推理形式正确,推理的前提和
结论间的关系是符合思维规律的要求的。例如:
\begin{equotation}
  一切文艺作品都有社会作用;

  小说是文艺作品;

  所以,小说有社会作用。
\end{equotation}
判断“一切文艺作品都有社会作用”包含的内容是客观事实的正确反映,这是一个真实的、
经过证明了的命题。同样,判断“小说是文艺作品”包含的内容也是真实的,也是为古往今
来的一切小说都属于文艺作品的这个客观事实所证明了的。其次,在这个推理中,“一切文
艺作品都有社会作用”和“小说是文艺作品”这两个前提与“小说有社会作用”这一结论之间的
联系,遵守了推理形式的逻辑规则。

再如:
\begin{equotation}
  清人纪晓岚《阅微草堂笔记》中记述了一则故事。清雍正十年六月某夜,河北献县一带
  下起大雷雨,雷电异常迅猛暴烈,一村民被雷击死。县令明晟去查看了现场,发现火是
  从地下起的,屋顶、屋梁都被炸飞,土坑地面也被揭掉。明晟想,雷击人应自上而下,
  不会使地开裂;如果毁坏房屋,也应是自上而下,不会是像现场所见这样自下而上炸开
  去的。于是,他怀疑这是一起冒做假雷的谋杀案。
\end{equotation}
明晟对雷击的怀疑是一个推理:
\begin{equotation}
  凡雷击是自上而下的,这次爆炸不是自上而下的,所以,这次爆炸不是雷击。
\end{equotation}
“凡雷击是自上而下的”和“这次爆炸不是自上而下的”两个判断是前提,“这次爆炸不是雷击”
是结论。

这个推理在这次破案中起着关键性的作用。后经进一步侦查并经审讯,真相是凶手事先制
作好炸药,趁着雷雨之夜将村民炸死,并伪装成雷击现场。

根据推理的思维进程方向的不同,把推理分为演绎推理、归纳推理和类比推理。

\subsection{演绎推理}

\index{演绎推理}
\subsubsection{演绎推理的概念}

演绎推理是由反映一般性知识的前提得出有关特殊性知识的结论的一种推理。

以演绎推理中的三段论为例,初步认识一下演绎推理的特点。

《鸿门宴》是司马迁《史记·项羽本纪》中的精彩篇章,其中有一段樊哙在宴会上指责项羽
的语言描写:
\begin{equotation}
  “夫秦王有虎狼之心,杀人如不能举,刑人如恐不胜,天下皆叛之。怀王与诸将约曰:
  ‘先破秦入咸阳者王之。’今沛公先破秦入咸阳,毫毛不敢有所近,封闭宫室,还军霸上,
  以待大王来。故遣将守关者,备他盗出入与非常也。劳苦而功高如此,未有封侯之赏,
  而听细说,欲诛有功之人,此亡秦之续耳。窃为大王不取也!”项王未有以应。
\end{equotation}
樊哙这段话借助演绎推理中的三段论,增加语言表达力,使项羽“未有以应”。

第一个三段论:
\begin{equotation}
  先破秦入咸阳者王之,

  沛公先破秦入咸阳,

  沛公应该在咸阳称王。(樊哙虽未说出结论,但听者皆知)
\end{equotation}
按照事先的“怀王之约”:“先破秦入咸阳者王之”,沛公先破秦入咸阳,沛公应该在咸阳称
王。樊哙向项羽指出沛公反而“封闭宫室,还军霸上,以待大王(项羽)来”,这样沛公在
道义上占据了高位,项羽如果想在鸿门宴上杀掉沛公,就要考虑舆论给自己带来的消极影
响。

第二个三段论:
\begin{equotation}
  秦王有虎狼之心,杀人如不能举,刑人如恐不胜,天下皆叛之,

  (项羽)听细说,欲诛有功之人。

  此亡秦之续耳。(项羽会延续秦朝灭亡的命运)
\end{equotation}
樊哙利用这个三段论不仅揭露了项羽包藏祸心,而且还一针见血地指出了项羽准备杀掉沛
公一事的本质和恶果。

\subsubsection{演绎推理的分类}

在演绎推理中,简单判断的推理主要有两种:直接推理和间接推理。

\index{直接推理}
\paragraph{直接推理}是一种最简单的演绎推理,是以一个命题为前提而推出结论的推
理。例如:
\begin{equotation}
  只有从实际出发,才能把经济工作做好,所以,如果要把经济工作做好,那么就要从实
  际出发。
\end{equotation}
这个推理就是以一个判断为前提推出结论的直接推理。

\index{间接推理}
\paragraph{间接推理}是有两个或两个以上前提的性质判断的推理。下面介绍四种形式
的间接推理。

\index{三段论}
\subparagraph{三段论} 三段论是传统逻辑的重要内容,古希腊哲学家亚里士多德是三段
论公理体系的创始人。“三段论”是性质判断三段论推理的简称,它是由两个包含着一个共
同概念的性质判断为前提,推出一个新的性质判断为结论的间接推理形式。例如:
\begin{equotation}
  所有的金属都是导电体,

  铜是金属,

  所以,铜是导电体。
\end{equotation}

\index{小项} \index{大项} \index{中项}
三段论所包含的三个不同的概念,分别叫小项、大项与中项。

小项就是作为结论的主项的那个概念,上例结论中的“铜”是小项。

大项就是作为结论的谓项的那个概念,上例结论中的“导电体”是大项。

中项就是在两个前提中都出现的那个概念,上例中的“金属”是中项。

\index{大前提} \index{小前提} \index{结论}
三段论由三个性质判断构成:大前提、小前提、结论。

包含大项的前提叫大前提,上例中的“所有的金属都是导电体”是大前提。

包含小项的前提叫小前提,上例中的“铜是金属”是小前提。

结论是“铜是导电体”。

三段论的一般规则有七条。

\begin{enumerate}
  \item 在一个三段论中,只能有三个不同的项。例如:
    \begin{equotation}
      物质是永恒不灭的,

      钢铁是物质,

      所以,钢铁是永恒不灭的。
    \end{equotation}
    大前提中的“物质”,是表达哲学上物质的概念,指在人们意识之外,并且不依赖于人
    们的意识的客观实在。在小前提中的“物质”,是表达具体物体这个概念。这个三段论
    犯了“四概念”的错误,它违反了三段论的规则,因而是不正确的。

  \item 中项至少要在一个前提中周延。例如:
    \begin{equotation}
      古典小说是文学作品;

      《红楼梦》是文学作品;

      所以,《红楼梦》是古典小说。
    \end{equotation}
    这个推理的中项(文学作品)两次都不周延,因此,它的结论并不是从前提中必然推
    出的。尽管结论是对的,但它并非是由前提必然推出的。

  \item 在前提中不周延的词项,到结论中也不得周延。例如:
    \begin{equotation}
      贪污行为是犯罪行为,

      张某的行为不是贪污行为,

      所以,张某的行为不是犯罪行为。
    \end{equotation}
    错误的三段论。大项“犯罪行为”在前提中不周延,在结论中却周延了,即结论所断定
    的“犯罪行为”的范围超出了前提中的所给予的“犯罪行为”的范围,犯了“大项扩大”的
    逻辑错误。

  \item 两个否定前提不能得出必然结论。
    \begin{equotation}
      艾滋病不是源于中国,

      乙型肝炎不是艾滋病,

      所以?
    \end{equotation}
    上例结论无法确定,是因为当大项、小项都与中项相排斥时,大项和小项之间的关系
    却不是必然排斥的,实际上存在着多种可能。

  \item 前提中有一个是否定的,则结论必然是否定的。例如:
    \begin{equotation}
      犯罪未遂不是犯罪中止,

      被告的行为是犯罪未遂,

      所以,被告的行为不是犯罪中止。
    \end{equotation}
    在两个前提中有一个是否定的情况下,只能推出一个否定的结论。

  \item 两个特称前提得不出确切的结论。例如:
    \begin{equotation}
      有些绿色植物可供食用,

      有些绿色植物是树叶,

      所以?
    \end{equotation}
    由上面两个特称前提就无法推出什么结论。

  \item 如果两个前提中有一个特称,结论必然特称。例如:
    \begin{equotation}
      棉花是经济作物,

      有些农作物不是经济作物,

      所以,有些农作物不是棉花。
    \end{equotation}
    前提中“有些农作物”是特称,不周延,因此,结论也是特称的。

\end{enumerate}

\subparagraph{联言推理} 联言推理是前提或结论为联言判断的推理。例如:
\begin{equotation}
  鲁迅是伟大的文学家,

  鲁迅是伟大的思想家,

  所以,鲁迅是伟大的文学家和思想家。
\end{equotation}
前提从不同角度写出了鲁迅先生的地位所在,结论完整地揭示了鲁迅先生的综合地位。因
此,联言推理能从不同角度对事物各方面的知识进行完整、全面的综合。

再如:
\begin{equotation}
  《敬业与乐业》是梁启超先生一篇演讲稿。

  梁先生在文中先提出论点:“确信‘敬业乐业’四个字,是人类生活的不二法门。”

  然后分别从“第一要敬业”和“第二要乐业”两个角度进行论证。

  最后得出结论:“敬业即是责任心,乐业即是趣味。我深信人类合理的生活应该如此……”
\end{equotation}
这篇文章使用了联言推理:

\begin{equotation}
人类生活,第一要敬业,第二要乐业;所以,人类合理的生活应该是敬业和乐业。
\end{equotation}

在写议论文或作演讲时,使用联言推理,可以使文章、说话脉络清晰,富有逻辑。

联言判断的所有联言肢必须是真的。只要有一个是假的,这个联言判断就是假判断。例如:
\begin{equotation}
  鲸鱼生活在海洋里,属于鱼类。
\end{equotation}
后一个联言肢是假的,这个联言判断是假的。

\index{选言推理}
\subparagraph{选言推理} 选言推理是指大前提中有一个是选言判断,并依据选言判断的
逻辑性质进行的演绎推理。例如鲁迅在《纪念刘和珍君》中有这样一个片段:
\begin{equotation}
  惨象,已使我目不忍视了;流言,尤使我耳不忍闻。我还有什么话可说呢?我懂得衰亡
  民族之所以默无声息的缘由了。沉默啊,沉默啊!不在沉默中爆发,就在沉默中灭亡。
\end{equotation}
这里就使用了选言判断:

\begin{equotation}
  不在沉默中爆发,就在沉默中灭亡,

  不在沉默中灭亡,

  所以在沉默中爆发。
\end{equotation}
大前提肯定了“不在沉默中爆发”和“就在沉默中灭亡”两种情况必居其一,小前提否定了“在
沉默中灭亡”,那结论就是“在沉默中爆发”。这个富有逻辑性的片段显示了鲁迅高超的语言
艺术。

依据选言判断的逻辑性质进行的演绎推理,可以分为相容的选言推理和不相容的选言推理。
例如:
\begin{equotation}
  张华或爱好阅读,或爱好旅游,

  张华不爱好阅读,

  张华爱好旅游。
\end{equotation}
这是相容的选言推理。“阅读”和“旅游”两种可能情况中至少有一种情况存在。前提之一肯
定了“张华爱好阅读”和“张华爱好旅游”中至少有一种情况存在,另一前提则否定了“张华爱
好阅读”这一情况,那么,其结论自然也就为剩下选言肢:“张华爱好旅游”了。

再如:
\begin{equotation}
  战争要么是正义的,要么是非正义的,

  这场战争是正义的,

  所以,这场战争不是非正义的。
\end{equotation}
这是不相容选言推理。“战争是正义的”,“战争是非正义的”两种可能情况中只有一种情况
存在。

选言推理在生活中被广泛应用。例如西晋陈寿《三国志·魏书·高柔传》记载的廷尉高柔用
选言推理审案的故事。
\begin{equotation}
  护军营士窦礼出营后没有回来。军营里以为他逃走,上表说要追捕他,收他的妻子盈和
  儿女为官家奴婢。盈呼冤无人过问。于是她又申诉到廷尉高柔处。高柔问道:“你凭什么
  知道你丈夫不会逃跑?”盈流泪回答说:“我丈夫年少时就孤身一人,奉养一个老太太,
  把她当作母亲,恭谨孝顺;又怜爱儿女,抚慰看顾从不远离,他不是轻薄狡诈不顾家室
  的人。”
\end{equotation}
对于窦礼的去向高柔形成了第一个选言判断:
\begin{equotation}
  窦礼或跑或被害。他奉养母亲,怜爱儿女,不轻薄狡诈,因此窦礼应该是被杀害的。
\end{equotation}

\begin{equotation}
  高柔又问道:“你丈夫与别人有怨仇吗?”回答说:“我丈夫很善良,与别人没有怨仇。”
  又问道:“你丈夫与别人在钱财上没有互助交往吗?”回答说:“曾经借钱给同营军士焦子
  文,让他还,他一直没还。”
\end{equotation}
针对窦礼被杀害,高柔就此又形成了第二个选言推理:
\begin{equotation}
  窦礼被杀或因为与人有怨仇,或因为钱财。既然窦礼为人善良,与别人没有怨仇,那么
  极有可能因为钱财被杀。焦子文向窦礼借钱不还,有很大嫌疑。最后查明,确实是焦子
  文杀害了窦礼。
\end{equotation}

\index{假言推理}
\subparagraph{假言推理}假言推理就是前提中有假言判断,并根据假言判断的逻辑性质而
进行的推理。例如,“新编阿凡提系列丛书”中有这么一段故事:
\begin{equotation}
  阿凡提在给他老伴写信。一个不懂礼貌的青年人跑来偷看,于是阿凡提写道:“有个小伙
  子在偷看我给你写信,千言万语,只好以后说了。”青年人一看,抗议道:“你为什么平
  白无故地诬蔑我偷看你写信?”阿凡提说:“你如果没偷看我写信,为什么知道我在信上
  写你偷看了呢?”小伙子被问得哑口无言。
\end{equotation}

这里包含了一个假言推理:
\begin{equotation}
  如果你没有偷看我写信,就不应该知道我写你偷看信的内容,

  可现在你知道我写你偷看信的内容,

  所以你偷看我写信无疑。
\end{equotation}

\subsection{归纳推理}

\subsubsection{归纳推理的概念}

\index{归纳推理}
“归纳”一词来源于古希腊文,本意是“诱导”。归纳推理是以个别知识为前提,推出以一般
知识为结论的推理。

归纳推理的特征:前提是一组判断,结论为一个全称判断。例如作家秦牧的《画蛋·练功》
一文,为了证明“练功重要”这一观点就使用了归纳推理:
\begin{equotation}
  吴道子一生苦练基本功,

  达·芬奇一生苦练基本功,

  齐白石一生苦练基本功,

  徐悲鸿一生苦练基本功,

  梅兰芳一生苦练基本功,

  盖叫天一生苦练基本功,

  所以所有的艺术工作者都是一生苦练基本功。
\end{equotation}

归纳推理的前提必须是真实的,否则,归纳推理就失去了意义;其次,尽管归纳推理的前提必须真实,但其结论未必一定真。

\subsubsection{归纳推理的分类}

归纳推理可以分为古典归纳推理和现代归纳推理。古典归纳推理又可分为完全归纳推理和
不完全归纳推理。现代归纳推理又可分为概率归纳推理和统计归纳推理。概率归纳推理和
统计归纳推理是由于概率被引入归纳推理而出现的,是归纳推理的前提对结论的支持度无
法永远保持100\%这一特点决定的。

下面重点介绍古典归纳推理。

\index{完全归纳推理}
\paragraph{完全归纳推理}完全归纳推理是由一类事物中每个或每部分对象都具有(或不
具有)某属性,得出该类事物全部对象都具有(或不具有)某属性的推理。例如:
\begin{equotation}
  张的学历是大学本科,

  王的学历是大学本科,

  李的学历是大学本科,

  张、王、李是某科室的全体成员,

  所以,某科室全体成员的学历都是大学本科。
\end{equotation}
根据对科室全部个别对象的考察,发现他们都具有大学本科学历,因而得出结论说:该类
对象都具有某种性质。

\index{不完全归纳推理}
\paragraph{不完全归纳推理}不完全归纳推理是根据一类事物中部分对象具有(或不具有)
某种属性,推出该类事物的全部对象都具有(或不具有)某种属性的归纳推理。

不完全归纳推理可以分为两种:一是简单枚举归纳推理;二是科学归纳推理。

\index{简单枚举归纳推理}
\subparagraph{简单枚举归纳推理}是根据同类部分对象重复出现某一属性而未遇到矛盾的
情况,从而推出该类对象都具有某种属性的结论的推理。简单枚举归纳推理结论不完全可
靠。例如《内经》中记载了这样一个故事:
\begin{equotation}
  一位患头痛病的樵夫,一次碰破足趾,出了一点血,但头部不痛了。后来头痛复发,又
  碰破上次的足趾,头痛又好了。以后头痛时,他就有意刺破该处,都有效应(樵夫碰的
  地方,即现在所称的“大敦穴”)。
\end{equotation}
这是一个不完全归纳推理中的简单枚举归纳推理。推理过程概括如下:
\begin{equotation}
  第一次碰破足趾某处,头痛好了,

  第二次碰破足趾某处,头痛好了,

  没有出现相反的情况,即碰破足趾某处,而头痛不好,

  所以,凡碰破足趾某处,头痛都会好。
\end{equotation}
这种依据经验而进行的不完全归纳推理,就是简单枚举归纳推理。“凡蚂蚁搬家,天要下
雨”,“瑞雪兆丰年”,都是简单枚举归纳推理的具体运用。

\index{科学归纳推理}
\subparagraph{科学归纳推理}是通过分析一类中的部分对象与某属性之间有(不具有)因
果关系,从而推出某类对象全部都具有(不具有)某种属性的推理。科学归纳推理的结论
带有必然性。例如:
\begin{equotation}
  鸡大量食用发霉花生成批死去,

  鸭大量食用发霉花生成批死去,

  鸽大量食用发霉花生成批死去,

  羊大量食用发霉花生成批死去,

  白鼠大量食用发霉花生成批死去,

  研究发现发霉的花生含有大量黄曲霉菌,而黄曲霉菌与致癌有必然联系,

  所以,所有大量食用发霉花生的动物都会成批死去。
\end{equotation}
这是一个典型的科学归纳推理形式。

和简单枚举归纳推理相比,科学归纳推理根据的是对对象何以存在某种性质的必然原因进
行的科学分析,因而其结论是比较可靠的。

\subsection{归纳推理与演绎推理的区别}

下面以演绎推理中的三段论和归纳推理中的简单枚举归纳推理为例来说明归纳推理与演绎
推理的区别。

推理一(三段论):
\begin{equotation}
  当且仅当某行为是乘人不备、公开夺取公私财物据为己有的行为才是抢夺罪,

  某甲的行为不是乘人不备、公开夺取公私财物据为己有的行为,

  所以某甲的行为不是抢夺罪。
\end{equotation}
推理二(简单枚举归纳推理):
\begin{equotation}
  铜(或金)受热体积膨胀,

  铝(或银)受热体积膨胀,

  铁(或铜)受热体积膨胀,

  铜、铝、铁是金属物体的部分对象,

  所以,金属物体受热都体积膨胀。
\end{equotation}
上述两种推理主要有以下四个方面的区别:

区别一,推理方向不同(思维的进程不同):演绎推理往往由一般推导出个别;归纳推理
往往由个别推导出一般。上例三段论由一般推出“某甲”具体行为;上例简单枚举归纳推理
由“铜”等个别“金属受热体积膨胀”推出“所有金属物体受热都体积膨胀”。

区别二,前提的数量不同:演绎推理的前提的数量是比较确定的,如直接推理为一个前提,
间接推理三段论为两个前提;归纳推理的前提的数量是不确定的,它可以有两个、三个或
多个,是根据需要不同而定的。上例三段论为两个前提;上例简单枚举归纳推理有四个前
提,还可以列举更多。

区别三,结论的范围不同:演绎推理的结论所断定的范围没有超出前提所断定的范围;归
纳推理除完全归纳推理外,结论所断定的范围都超出了前提的断定范围。上例三段论结论
断定范围“某甲的行为不是抢夺罪”没有超过大前提“抢夺罪”的理论判定;上例简单枚举归
纳推理结论断定范围“金属物体受热都体积膨胀”超出了前提中“铜”等个别“金属受热体积膨
胀”的断定范围。

区别四,前提与结论联系的性质及可靠性不同:演绎推理的结论是必然的,只要前提真,
结论必真;归纳推理只有完全归纳推理和科学归纳推理的结论带有必然性,其他的都带有
或然性。上例三段论前提真,结论必真;上例简单枚举归纳推理只要有相反情况的存在,
无论暂时碰到与否,其一般性结论就必然是错的。

\subsection{类比推理}

\index{类比推理}
\subsubsection{类比推理的概念}

类比推理是根据两个或两类对象在一系列属性上相同或相似,又知一类对象还具有其他属
性,从而推出另一类对象也具有同样属性的推理方法。例如:
\begin{equotation}
  十八世纪美国伟大的科学家和发明家本杰明·富兰克林通过比较发现,地面上的电与天空
  中的雷电有众多相似的特性,比如二者发出的光颜色相同,爆发时都有声响,速度相当
  之快,等等,因为地面上的电被证实可被传导,于是富兰克林猜想,天电也有可能可被
  传导。后来富兰克林通过研究发现,储存了天电的莱顿瓶可以产生一切地电所能产生的
  现象,证明了天电与地电是一样的,天电也可被传导。
\end{equotation}
推理概括如下:
\begin{equotation}
  地电和天电具有相似的属性,

  发出同色的光,

  爆发时有声响,

  ……

  速度相当之快,

  地电具有属性可被传导,

  所以,天电也具有相似属性即可被传导。
\end{equotation}

类比在反驳上也是很有效力的。通过构造一个和对方相同结构的类比推理进行反驳,可以
直接指出对方推理的无效,如下面这个例子:

原推理:
\begin{equotation}
  你的祖父死于海难,

  你的父亲死于海难,

  所以,你不要去当水手。
\end{equotation}
类比反驳:\index{类比反驳}
\begin{equotation}
  你的祖父死在床上,

  你的父亲死在床上,

  所以,你不要睡在床上。
\end{equotation}

在这个类比反驳中,构造一个和原推理结构形式相同的推理,而新构造的推理其结论很明
显是我们不能接受的,由此推出原推理的结论也是站不住脚的。

\subsubsection{类比推理的特征}

类比推理所类比的两对象,可以是从属于同一类的两个不同的个体事物,如火星与地球;
也可以是两个不同类的事物,如声和光;还可以是一类事物与另一类事物的个体,如以蜻
蜓的飞行特性来类推航空飞行器的某些特性。

类比推理的结论不一定可靠。类比推理的结论超出了前提所断定的范围,因此,它并不被
前提所蕴涵,也就是说,即使前提是真的,类比推理的结论也可能是假的。所以,类比推
理是一种或然性推理。

\subsubsection{提高类比推理结论可靠性的方法和途径}

通过增加前提中两个或两类对象相同属性的类比数量,来提高结论的可靠性。

尽量采用对象的本质属性进行类比。如果两个或两类对象本质属性相同或相似,那么它们
在其他属性上就更有可能相同或相似。

进行类比推理时,还要研究特殊属性,如果这些特殊属性与推出属性是互相矛盾的,那么
结论就不可靠。

\section{形式逻辑的基本规律}

形式逻辑的基本规律是关于思维形式结构的规律,是各种思维形式的特殊规律或规则的依
据。

形式逻辑的基本规律主要包括同一律、矛盾律、排中律和充足理由律。

\subsection{同一律}

\subsubsection{同一律的概念}

\index{同一律}
任何一个思想与其自身是等同的。通俗地说,在同一时间、同一对象、同一关系过程中,
每一思想都必须保持自身同一。

\subsubsection{同一律要求}

\begin{enumerate}
  \item 在同一思维过程中,每个思想都必须是确定的;

  \item 在同一思维过程中,每个思想前后应当保持一致。例如:
\end{enumerate}

\begin{equotation}
  老师:我读的是中国现代作家鲁迅写的《狂人日记》,不是俄国作家果戈里写的《狂人
  日记》。
\end{equotation}
符合同一律。任何概念都有具体的含义,即必须明确所指向的东西,不能把鲁迅的《狂人
日记》和果戈里的《狂人日记》混为一谈。

再如:
\begin{equotation}
  有人随地吐痰,别人批评他:“随地吐痰不卫生。”他理直气壮地说:“有痰不吐更不卫
  生。”别人进行反驳:“是的,有痰不吐不卫生,但那只是你个人的卫生,你不能为了个
  人卫生而影响公共卫生!”
\end{equotation}
上述例子中的“卫生”内涵有二:第一处指公共卫生,第二处指个人卫生。吐痰的人偷换概
念进行狡辩,反驳的人厘清了“卫生”的两个含义。

\subsection{矛盾律}

\index{矛盾律}
在同一思维过程中,两个相互否定的思想,不能同真,必有一假,也就是说,同一思想不
能既是自身,又是对自身的否定。矛盾律的主要作用在于保证思维的无矛盾性即首尾一贯
性。

例如在《韩非子·难一》中“自相矛盾”的故事中,叫卖矛和盾的楚人要么是自己的矛能刺穿
自己的盾,要么是自己的矛不能刺穿自己的盾,两个相互矛盾的判断不能同真,必有一假。

下面这个有关爱迪生的故事,也有违反矛盾律的错误。
\begin{equotation}
  一个年轻人想到爱迪生的实验室工作,为了表现他的雄心大志,就说:“我一定会发明一
  种万能溶液,它能溶解一切物品。”爱迪生便问他:“那么你想拿什么容器来盛放它呢?”
  年轻人一时语塞。
\end{equotation}
“万能溶液有瓶子装”和“万能溶液没有瓶子装”,两个相互矛盾的判断不能同真,必有一假。

\subsection{排中律}

\index{排中律}
在同一思维过程中,两个互相矛盾的思想不能同假,必有一真。“排中”的意思就是排除含
糊,排除既否定这个又否定那个,即要排除含糊其词、骑墙居中。排中律要求对是非问题
必须表示明确的态度:赞成什么,反对什么。例如:
\begin{equotation}
  甲:《聊斋志异》值得读;

  乙:《聊斋志异》不值得读;

  丙:两种观点我都不赞成。读,花很多时间;不读,又有点儿可惜。
\end{equotation}
“我都不赞成”违反排中律。“《聊斋志异》值得读”和“不值得读”是两个相互矛盾的判断,
必有一真,不能同假。因而犯了模棱两可的错误。

\subsection{充足理由律}

\index{充足理由律}
在同一思维过程中,一个论断被确定为真,总有它的充足理由。人们在论证和交流思想时,
只有符合充足理由律,才能使思想立得住,使人信服。

“言之有理,持之有据。”作为充足理由,必须具备两个条件:一是理由本身要真实;二是
理由与论断之间有必然的逻辑联系,即理由能推出论断。只有具备这两个条件的理由才是
充足的理由。例如坊间流传的“量体裁衣”的故事:
\begin{equotation}
  古代有位裁缝名气很响。一位官员请他去裁制一件朝服。裁缝量好了他的身腰尺寸,又
  问:“请教老爷,您当官当了多少年了?”官员很奇怪:“你量体裁衣就够了,还要问这些
  干什么?”裁缝回答说:“年轻相公初任高职,意高气盛,走路时挺胸凸肚,裁衣要后短
  前长;做官有了一定年资,意气微平,衣服应前后一般长短;当官年久而将迁退,则内
  心悒郁不振,走路时低头弯腰,做的衣服就应前短后长。所以,我如果不问明做官的年
  资,怎么能裁出称心合体的衣服来呢?”
\end{equotation}
裁缝有关“裁衣理论”符合充足理由律:理由真实,推论合乎逻辑。

\section{常见逻辑错误举例}

\subsection{概念不清}

概念是反映事物本质属性的思维形式。概念有内涵和外延两个逻辑特征。概念不明确,不
仅会导致不正确的思维,同时,在表达思想时,也会说出许多不通的语句。

\subsubsection{限制、概括不当}

根据内涵与外延的反比规律,我们就可以用逐渐增多概念的内涵的方法,来逐渐减少概念
的外延,这个方法叫作概念的限制法。

概念的限制法,从语言表达方面说,就是增加限制词的方法;所谓“概括”,就是扩大概念
外延,缩小概念内涵。从语言学上说,就是不断去掉形容词、限制词,使概念趋向一般化。
对概念的限制或概括不当就会造成概念不清的语病。例如:
\begin{equotation}
  滚滚的长江基本上可以说是我国的第一大河。
\end{equotation}
去掉“基本上可以说”的限制。

\subsubsection{集合概念误用}

\index{集合概念误用}
集合概念,就是反映集合体的概念,如“森林”,“舰队”等都是集合概念。作为集合体中的
个体事物并不具有该集合体的属性。在实际语言表达中,把集合概念误当作非集合概念、
普遍概念,就会出现逻辑错误。例如:
\begin{equotation}
  元旦,是一个多么富有生命力的词汇啊!
\end{equotation}
集合概念误用。应改“词汇”为“词语”。

\subsubsection{并列不当}

\index{并列不当}
从概念的外延看,概念间的关系可以分为相容和不相容两大类。注意不同关系的概念在不
同语句中的特点,不要随意凑合、随意并列共提,否则,就会犯并列不当的错误。例如:
\begin{equotation}
  今年我省小麦、大麦、棉花和粮食都获得了大丰收。
\end{equotation}
“小麦”,“大麦”和“粮食”是种属关系,“棉花”不属于“粮食”一类,都不宜并列。

\subsubsection{定义错误}

\index{定义错误}
下定义有具体的方法和严格的规则。不按照这些方法和规则,轻率地下定义,就容易产生
定义错误。例如:
\begin{equotation}
  电子计算机不是用手计算的机器。
\end{equotation}
违反了“定义不能是否定判断”的规则。

\begin{equotation}
  法律就是由国家政权保证执行的行为规则。
\end{equotation}
定义过宽。在“由”字后加“立法机关制定”。

\subsubsection{偷换概念}

\index{偷换概念}
在同一个语句中,看上去运用的是同一个语词,而实际上这个语词所反映的概念在不知不
觉中转换成了另一个概念。这就犯了“偷换概念”的逻辑错误,它违反了逻辑思维规律中的
同一律。例如:
\begin{equotation}
  儿子说:“我要去看花灯。”

  爸爸说:“家中有这么多灯,还不够你看啊?”
\end{equotation}
偷换概念。前一个“灯”指“花灯”后一个“灯”指照明用的“灯”,同样一个词,所代表的概念
已转换。

\subsection{判断失当}

\subsubsection{判断失真}

\index{判断失真}
判断要反映客观事物的本来面目,要符合实际情况。违反了这个要求,判断就不符合实际,
不合乎事理,也就是判断失真。例如:
\begin{equotation}
  他脸上豆大的汗珠滚来滚去。
\end{equotation}
“豆大的汗珠”在“他脸上”,“滚来滚去”,不合事实。再大的汗珠也不可能在“脸上”,“滚来
滚去”。改成“直往下滚”才合情合理。

\subsubsection{判断歧义}

\index{判断歧义}
对概念的判断数量不清、范围不明,或是指代不定的,就不能算是准确判断,也就犯了判
断歧义的逻辑错误。例如:
\begin{equotation}
  在军区总医院看病的是他的爸爸。
\end{equotation}
“看病的”既可以指医生,也可以指病人。原句没有说清楚“他爸爸”究竟在医院干什么。在
句末加上“他妈妈没病”,或“他的医术很高明”。

\subsubsection{自相矛盾}

\index{自相矛盾}
在同一个判断句中,使用的概念、论断,前后必须一致,否则就会出现前言不搭后语、自
相矛盾的错误。例如:
\begin{equotation}
  下课铃声响了,同学们一窝蜂地走出教室。
\end{equotation}
“一窝蜂”与“走”相矛盾。“一窝蜂”是形容许多人乱哄哄同时行动或说话的样子,后面常跟
“冲”,“拥”等动词相配合。“走”既没有人多的含义,也没有“同时行动”的意味,与“一窝蜂”
前后矛盾。因此应改“走”为“拥”。

\subsubsection{否定误用}

\index{否定误用}
否定词运用得正确与否,直接关系到判断句能否准确表达的问题。错用否定词,就会造成
否定误用的语病。例如:
\begin{equotation}
  为了防止这类交通事故不再发生,我们加强了交通安全教育和管理。
\end{equotation}
“防止……不再发生”,岂不是希望“交通事故”,“发生”?删去“不”字,变成“防止这类交通事
故再发生”,表达就清楚顺畅了。

\subsubsection{关系判断不当}

如果关系内容错误,那么判断就是错误的,例如某新闻标题:
\begin{equotation}
  某地切实减轻农民不合理负担。
\end{equotation}
错误的关系判断,“负担”前加定语“不合理”,那就是说还有一些“不合理”的负担将继续“合
理”地存在下去,现在只是“减轻”而已。

\subsubsection{假言判断不当}

假言判断是断定事物情况之间条件联系的一种复合判断,如果硬把不具有任何条件联系的
两个判断放在一起,尽管两个判断可能都是真的,而整个假言判断也是假的,因为它犯了
“强加条件”的逻辑错误。例如:
\begin{equote}
  \begin{enumerate}
    \item \label{item:喜鹊叫} 喜鹊叫,喜事到。
    \item \label{item:人有多大胆} 人有多大的胆,地有多高的产。
  \end{enumerate}
\end{equote}
例~\ref{item:喜鹊叫} 和例~\ref{item:人有多大胆} 都是充分条件假言判断的省略式,
都犯了“强加条件”的逻辑错误。例~\ref{item:喜鹊叫} 中“喜鹊叫”和“喜事到”之间,
例~\ref{item:人有多大胆} 中“人的胆量”和“地的产量”之间,没有任何的条件联系。

\subsubsection{联言判断不当}

联言判断要求它的肢判断必须都是真实正确的。只要有一个肢判断不正确,这个联言判断
就是不正确的。例如:
\begin{equotation}
  医生指出:不能随地乱吐痰,但不能不吐。
\end{equotation}
这是一个联言判断,两个肢判断分别是“不能随地乱吐痰”和“但不能不吐”。根据第一个肢
判断可以推出:随地吐痰是可以,但是不能乱吐。这显然有悖“不能随地吐痰”的事实,这
个肢判断的错误导致整个联言判断错误。

\subsection{推理不当}

\subsubsection{三段论推理不当}

\index{中项两次不周延}
\paragraph{中项两次不周延}例如:

校长向来调研的领导汇报工作时说,该校绝大部分中层干部的文化程度是大学毕业。领导
问该校的学生处主任(中层)说:“你是什么大学毕业的?”主任很不好意思,因为他不是
大学毕业。这位领导这样问的原因是在他的脑子里有这样一个推理:
\begin{equotation}
  该校绝大部分中层干部是大学毕业,

  这位主任是该校的中层干部,

  所以,他是大学毕业。
\end{equotation}
领导的推理是错误的。“该校绝大部分中层干部”这个概念(中项)在两个前提中都不周延,
犯了“中项没有一次周延”的逻辑错误。

\index{四概念错误}
\paragraph{四概念错误}任何一个三段论只能有而且必须有三个名词,即三个概念。但由
于词具有多义的特点,把三段论中实际上具有的四个名词,误当作三个名词的现象时有发
生,发生推理错误,这就是所谓的“四概念”错误。例如:
\begin{equotation}
  矛盾是永远存在的,我和他有矛盾,所以我和他的矛盾不可能化解。
\end{equotation}
“矛盾”包含了两个意思。前一个是指哲学中辩证法上的概念,指客观事物和人类思想内部
各个对立面之间互相依赖而又互相排斥的关系。后一个泛指对立的事物互相排斥。同一个
名词,却表示两个概念,该三段论中实有“四概念”了。

\subsubsection{否定不足}

\index{否定不足}
使用选言推理时,在排除其他选言肢时必须否定彻底。否则,对其他选言肢虽然有否定,
但否定的理由不充分,属于“否定不足”的错误情形,这样是不能得到可靠结论的。例如:
\begin{equotation}
  在一起盗窃案中,确定了两个有作案嫌疑的内部工作人员某甲和某乙。否定了某乙作案
  的可能,理由是某乙经济比较宽裕,生活并不困难,平时工作表现积极,不会是作案者,
  由此断定某甲是本案的作案人。
\end{equotation}
这一推理运用了肯定前提式的充分条件选言推理,由于对其中一个选言肢的否定是不彻底
的,前提虚假,所以结论为假。

\subsubsection{轻率概括}

\index{轻率概括}
简单枚举归纳推理容易犯“轻率概括”的逻辑错误:轻率概括是指只根据少量重复出现的个
别事例,贸然做出一般性结论。例如我们熟悉的出自《韩非子·五蠹》中的“守株待兔”的故
事:
\begin{equotation}
  宋人有耕田者。一日,兔走触株。宋人因之释其耒,守株待兔。
\end{equotation}
这个故事说明耕田的人仅仅根据“兔走触株”这一偶然性事件,就推断出守株便能待兔的普
遍性结论,在思维方法上犯了“轻率概括”的逻辑错误。

\subsubsection{条件关系混淆}

\index{条件关系混淆}
假言推理是以假言判断为前提的,其特点也是由假言判断规定的。假言推理肯定或者否定
前、后条件与所得的结论之间有着严格的逻辑规则,不能随意推断,否则,推出的结论就
可能不可靠。
\begin{equotation}
  假如我是一个科技工作者,我就能为社会建设做贡献。我不是科技工作者,所以我不能
  为社会做贡献。
\end{equotation}
这是一个充分条件的假言推理。它的推理规则是:通过肯定前件,来肯定后件;也可通过
否定后件,从而否定前件。原句却错用了通过否定前件来否定后件的方式,所以得不出可
靠结论。因为不一定非是“科技工作者”,才能“为社会做贡献”。

\subsection{形式逻辑的基本规律使用不当}

\subsubsection{违反同一律的逻辑错误}

\index{混淆概念} \index{偷换概念}
\paragraph{混淆概念或偷换概念}即把不同概念误当作同一概念使用。例如《韩非子》中
有这样一个故事:
\begin{equotation}
  郑县有个姓卜的人,他的裤子破了个洞,叫妻子给做条新的,妻子问他做什么样子的,
  他说照原样做即可。他妻子做好新裤后,便比照旧裤破的地方,在新裤上剪了个位置、
  大小相同的洞。
\end{equotation}
“妻子”的理解违反了同一律。因她丈夫说的“照原样”是指照原裤的尺寸,而她却把“照原样”
理解成原裤破洞的地方新裤也要有一样的破洞。

鲁迅的作品《孔乙己》中孔乙己的“窃书不能算偷”的观点犯了什么逻辑错误呢?
\begin{equotation}
  孔乙己便涨红了脸,额上的青筋条条绽出,争辩道,“窃书不能算偷…… 窃书!…… 读书人
  的事,能算偷么?”
\end{equotation}
孔乙己把不同语词“偷”和“窃”表达一个概念故意说成两个概念,并偷换概念,违反了同一
律。

\index{转移论题} \index{偷换论题}
\paragraph{转移论题或偷换论题}这种错误是在论证过程中把两个不同的论题(判断或命
题)这样或那样地混淆或等同起来,从而用一个论题去代换原来所论证的论题。例如有一
篇《临渊羡鱼不如退而结网》的文章开头是这样写的:
\begin{equotation}
  如同下棋一般分为本手、妙手和俗手,很多人都追寻所谓“妙手”,从而忘却了本为基础
  的“本手”。这往往会使“妙手”变为“俗手”。

  身处于如今的快节奏生活中,若被卷入外界的浪潮中无法脱身,那么人就会沉沦下去,
  被时代的洪流所淹没,所以要找到自己的位置,摆好自己的位置。
\end{equotation}
上例中第一节话题论述“本手、妙手和俗手”之间的关系,强调“本手”的重要性,第二节论
证“位置”的重要性,两段话题不一致,而且两个话题和标题也都不符。

\subsubsection{违反矛盾律的逻辑错误}

如果对于两个互相矛盾的命题同时给予肯定,或者说,如果对同一对象同时作出两个互相
矛盾的断定,那么就必然会产生逻辑矛盾。
\begin{equotation}
  夜晚,远远望去,整个楼漆黑一团,只有一个房间还灯火辉煌。
\end{equotation}
“整个楼漆黑一团”和“只有一个房间还灯火辉煌”,自相矛盾,必有一假。

\subsubsection{违反排中律的逻辑错误}

违反排中律出现的是“两不可”和“未置可否”的逻辑错误。企图回避对原则性或实质性问题
作出明确的答复,或采取含糊其词、模棱两可的回答,都是违反排中律要求的。

\index{两不可}
\paragraph{“两不可”逻辑错误}例如:
\begin{equotation}
  警察:前天晚上你们在哪儿待着?

  甲:我在单位加班。

  乙:他不在单位加班。

  警察:都是胡说。当时你们俩都到作案现场去过。
\end{equotation}
从逻辑学的角度看,警察的说法违反了排中律。

 \index{未置可否}
\paragraph{“未置可否”逻辑错误}“未置可否”即对两个互相矛盾的判断,既不肯定,也不
否定,含糊其辞,不明确表态。

鲁迅《祝福》中,沦为乞丐,走投无路的祥林嫂和知识分子形象的“我”关于“灵魂有无”的
一段对话。
\begin{equotation}
  “就是——” 她走近两步,放低了声音,极秘密似的切切的说,“一个人死了之后,究竟有没
  有魂灵的?”

  ……“也许有罢,——我想。”我于是吞吞吐吐的说。

  ……这时我已知道自己也还是完全一个愚人,什么踌躇,什么计画,都挡不住三句问,我
  即刻胆怯起来了,便想全翻过先前的话来,“那是,……实在,我说不清……。其实,究竟有
  没有魂灵,我也说不清。”
\end{equotation}
“我”对“灵魂有无”的模棱两可凸显了“我”懦弱和逃避的形象,从逻辑学的角度看,“我”未
置可否的作答违反了排中律。

\subsubsection{违反充足理由律的逻辑错误}

如果违反充足理由律的要求,理由不真实,或者“推不出”,就难以让人信服和接受。因此,
遵守充足理由律的要求是正确思维必不可少的条件。

\index{理由虚假}
\paragraph{理由虚假}例如:
\begin{equotation}
  地球是宇宙的中心,因为日月星辰都是围绕地球转的。
\end{equotation}
理由“日月星辰都是围绕地球转”虚假。

\paragraph{推不出}例如:
\begin{equotation}
  如果一个人是运动员,那么他就要经常锻炼身体,我不是运动员,所以,我不要经常锻
  炼身体。
\end{equotation}
违背充足理由律。这一假言推理的大前提是一个充分条件的假言判断,而充分条件的假言
判断是不应当从否定前件到否定后件的。因此,尽管这个推论的理由“运动员要经常锻炼身
体”和“我不是运动员”都是真的,但由于它违反推理的规则,它的推理形式是错误的、无效
的,即推断不是从理由中有逻辑地推想出来的。

\chapter{修辞}

\index{修辞} \index{修辞格}
修辞也称“修辞格”,是为提高语言表达效果而形成的各种修饰、加工语言的特定格式。简
言之,就是透过修饰、调整语句,运用特定的表达形式以提高语言表达作用的方式或方法。
语文中主要修辞手法有:比喻、拟人、夸张、比较、排比、对偶、反复、借代、比拟、互
文、设问、引用、呼告、反问、顶真等。

\section{修辞方法辨析}

\subsection{比喻与借代的区别}

\index{修辞!比喻}
\subsubsection{比喻}

比喻是两种不同性质的事物,彼此有相似点,便用一事物来比方另一事物的一种修辞手法。
比喻构成须满足以下两个条件:甲和乙必须是两种不同类的事物,否则不能构成比喻;甲、
乙之间必须有相似点。

比喻主要有以下几种:

\begin{enumerate}
  \item 明喻。本体、喻体都出现,中间用比喻词“像”,“似”等联结。常见形式:甲像乙。
    例如:

    \begin{equotation}
      叶子出水很高,像亭亭的舞女的裙。(朱自清《荷塘月色》)
    \end{equotation}

  \item 暗喻。本体、喻体都出现,中间用比喻词“是”,“成了”,“变成”等联结。典型形
    式:甲是乙。例如:

    \begin{equotation}
      广场上是雪白的花圈的海洋,纪念碑已堆成雪白的山冈。
    \end{equotation}

  \index{修辞!借喻}
  \item 借喻。不出现本体和比喻词,直接用喻体代替本体。借喻的典型形式是甲代乙。
    例如:

    \begin{equotation}
      夕阳映照下的西湖湖面上洒满了碎银,波光粼粼,熠熠生辉。
    \end{equotation}

    “碎银”是喻体,直接替代了没有出现的本体。
\end{enumerate}

比喻的作用:化平淡为生动;化深奥为浅显;化抽象为具体,等等。

\subsubsection{借代}

\index{修辞!借代}
借代是用相关的事物来代替所要表达的事物的一种修辞手法。

借代主要有以下几种:

\begin{enumerate}
  \item 特征代本体。例如:

    红眼睛原知道他家里只有一个老娘。(鲁迅《药》)

  \item 材料代本体。例如:

    故木受绳则直,金就砺则利。(荀子《劝学》)

  \item 标志代本体。例如:

    谁料竟会落在“三道头”之类的手里呢,这岂不冤枉!(鲁迅《为了忘却的记念》)

  \item 人名代著作。例如:

    我们要多读点鲁迅。

  \item 绰号代本人。例如:

    “芦柴棒,去烧火!”(夏衍《包身工》)

  \item 专名代泛称。例如:

    一千个读者就有一千个哈姆莱特。

  \item 具体代抽象。例如:

    不拿群众一针一线。

  \item 部分代整体。例如:

    吟罢低眉无写处,月光如水照缁衣。(鲁迅《无题·惯于长夜过春时》)
\end{enumerate}

借代的作用:可以引人联想,使表达产生形象突出、特点鲜明、具体生动的效果。

\subsubsection{借代与借喻的区别}

\begin{itemize}
  \item 借代的作用是“称代”,它只代不喻;借喻的作用是“比喻”,虽然也有代替的作用,
    但总是喻中有代。

  \item 构成借代的基础是事物的相关性,即要求借体和本体有某种关系;构成借喻的基
    础是事物的相似性,即要求喻体和本体有某些方面的相似处。

\index{修辞!借喻}
  \item 借喻可以改为明喻或暗喻,借代则不能。例如:

    \begin{equote}
      \begin{enumerate}
        \item \label{item:借代} 这大艺术喷射出的大美,曾倾倒过几多王朝,也曾风
          靡过朱门绣户,蓬门茅舍;这大美曾使盖世英雄五尺刚化为绕指柔。(李存葆
          《飘逝的绝唱》)

        \item \label{item:借喻} 把事情或者意见向有关的人说明,把错误或罪行坦白

        \item 我总觉得周围有长城围绕。这长城的构成材料,是旧有的古砖和补添的新
          砖。两种东西联为一气造成了城壁,将人们包围。(鲁迅《长城》)
      \end{enumerate}
    \end{equote}

    例~\ref{item:借代} 是借代,以“朱门绣户”代指“住在朱门绣户中的贵族”,以“蓬门茅
    舍”代指“住在蓬门茅舍中的穷人”,它只代不喻,不能改成比喻,不能说“贵族像朱门
    绣户,穷人像蓬门茅舍”。

    例~\ref{item:借喻} 是借喻,用“长城”比喻某种障碍物,这里既没有比喻词,也没有出
    现本体。这句话可以改成明喻。

\end{itemize}

\subsection{比喻与通感的区别}

\subsubsection{通感}

\index{修辞!通感} \index{修辞!移觉}
通感是一种修辞手法,又称为移觉。就是把不同感官的感觉(听觉、视觉、触觉、嗅觉、
味觉等)沟通起来,以感觉写感觉,起到增强文采的艺术效果。例如:
\begin{equote}
  \begin{enumerate}
    \item \label{item:玻璃做的夏天} 突然有钟声缓缓飘上来,很重,很古老,很悠久,
      很轻柔。(陈丹燕《玻璃做的夏天》)

    \item \label{item:李凭箜篌引} 昆山玉碎凤凰叫,芙蓉泣露香兰笑。(李贺《李凭箜
      篌引》)
  \end{enumerate}
\end{equote}
例~\ref{item:玻璃做的夏天} 将把听觉转化为触觉。例~\ref{item:李凭箜篌引} 从视觉
的角度描写听觉,将梨园艺人李凭弹奏箜篌之声化为鲜花的娇美形象,以露水在残荷上滚
动、滑落的视觉感受摹写箜篌声音悲抑;而以“香兰笑”的视觉感受刻画琴声的欢快。

为什么这些感觉可以相互转化呢?是因为它们之间有相似点。例如:

\begin{equotation}
  \tolerance=500
  微风过处,送来缕缕清香,仿佛远处高楼上渺茫的歌声似的。(朱自清《荷塘月色》)
\end{equotation}
清香是嗅觉,歌声是听觉,由嗅觉向听觉转移,形象生动地写出了荷香隐隐约约、若有若
无、清淡缥缈的特点。“清香”与“歌声”同属美好的事物,两个优美的意象叠加在一起,烘
托出了环境的清幽静谧。

通感的作用:能突破语言的局限,丰富表情达意的审美情趣,起到增强文采的艺术效果。

\subsubsection{比喻与通感的区别}

\begin{enumerate}
  \item 比喻强调“相似性”,通感侧重“相通性”。

  \item 比喻离不开本体、喻体和相似点,三者是统一的。通感只有本体而没有喻体和相
    似点。其表现形式为“本体:甲感觉→乙感觉”。

  \item 比喻是两个具体相似点的事物做比,不要求两种感觉进行转换;通感必须是两种
    不同感觉的转换。

  \item 通感往往借助于比喻、比拟、夸张等修辞方法来表达,以引起人们的联想,去获
    得具体生动的形象,增加了语言表达的巧妙性,使读者感到新奇而富有情趣。例如:

    \begin{equotation}
      突然是绿茸茸的草坂,像一支充满幽情的乐曲。(刘白羽《长江三峡》)
    \end{equotation}
    把视觉上的“草坂”与听觉上的音乐或声音沟通起来,把事物的无声姿态描摹成好像有
    声音,用这种通感手法来唤起读者丰富的联想。同时,又是通感和比喻的结合。

\end{enumerate}

以下三个例子同时使用了通感和比喻的修辞手法。
\begin{equotation}
  那是一种男性的嗓音,一波三折,委婉摇曳,就像伊犁的苹果一样芳香,又像伊犁的青
  杨一样潇洒。(王蒙《夜半歌声》)

  树上黄鹂的婉转歌声,就像清凉的泉水一样。(臧克家《一首短诗的构思过程》)

  我走进北京的市场,过客的耳语像桂花飘香。(刘征《北京的市场》)
\end{equotation}

\subsection{比喻与象征的区别}

\subsubsection{象征}

\index{修辞!象征}
象征手法的使用可以使作品更加生动、形象、富有感染力,就这个角度而言,象征可以被
视作一种修辞手法。象征,就是不直接描绘事物,而根据事物之间的相互联系,借助联想,
说的是乙,叫人联想到甲。象征由“象征体”和“本体”两个部分组成,例如菊花象征着隐逸,
其中“菊花”是“象征体”,“隐逸”是“本体”。

事物的象征关系,有的来自自然界,如太阳象征光和热,松树象征坚韧;有的来自神话传
说,如银河象征夫妻分离;有的来自社会习俗,如有的民族以白色象征哀悼,有的民族以
黑色象征哀悼;有的来自历史事实,如希特勒象征法西斯统治。一般说来,来自自然界的
象征关系,带有较大的普遍性,其余的则往往因时因地而异。例如:

\begin{equotation}
  再往上仔细看时,却不觉也吃了一惊;——分明有一圈红白的花,围着那尖圆的坟顶。
  (鲁迅《药》)
\end{equotation}
象征体是具体的“花环”,它象征着一种抽象的情感:烈士永远活在人们心间,革命自有后
来人。正如鲁迅自己在《〈呐喊〉自序》中说,这是“用了曲笔”。

\begin{equotation}
  在月光下,我看见他眼睛里晶莹发亮,我也看见那条枣红色上洒满白色百合花的被子,
  这象征纯洁与感情的花,盖上了这位平常的、拖毛竹的青年的脸。(茹志鹃《百合花》)
\end{equotation}
象征体是具体的“百合花”,象征抽象的精神:通讯员和新媳妇都有一颗高尚、纯洁、美好
的心灵。

象征的作用:象征手法的运用,能使作品显得委婉含蓄,激起读者的联想。

\subsubsection{比喻与象征的区别}

\paragraph{构成不同}

比喻是借此喻彼,喻体一般应该是让人看得见、听得见或摸得着的具体事物、具体人,它
不像象征那么含蓄。喻体和本体要求形似,喻体往往是具体的,如“问君能有几多愁?恰似
一江春水向东流”这个比喻句,喻体“一江春水”就是具体而形象的。

象征是借此寓彼,不直接把意思说出来,而是通过某一特定的具体形象以表现与之相似或
相近的概念、思想和感情。象征体和本体要求神似,象征体往往是具体的,本体往往是抽
象的,如意念、精神、品质、情感、概念等,例如狮子是勇敢的象征,鸽子和橄榄枝象征
和平,大棒象征武力,而本体“勇敢”,“和平”,“武力”都是抽象的。

\paragraph{作用不同}

比喻可使事物生动形象、具体可感,或使深刻的、抽象的道理浅显而具体地表达出来。例
如:
\begin{equotation}
  谎言是一只心灵的蛀虫,将人的心蛀得面目全非。
\end{equotation}
用形象的“蛀虫”来暗喻抽象的谎言,形象地突出了谎言的可怕。

象征,一般是通过象征的事物(象征体)去理解抽象的意义,最常见的就是用具体事物象
征抽象事理。例如:
\begin{equotation}
  一个浪,一个浪,无休止地扑过来。每一个浪都在它脚下,被打成碎末,散开……它的脸
  上和身上,像刀砍过的一样,但它依然站在那里,含着微笑,看着海洋。(艾青《礁
  石》)
\end{equotation}
作者运用象征手法传达了一种在逆境中不屈服的刚强、乐观的人格理想。

\paragraph{范围大小不同}

象征对象是整篇文章,或者至少是文章中的一大段话,因此,象征在范围上远远超过了比
喻涉及的范围。

从高尔基的《海燕》、陶铸的《松树的风格》、茅盾的《白杨礼赞》等几篇文章中采用的
象征手法来看,都是融贯全篇的。海燕的“大无畏精神”,松树的“共产主义精神”,白杨树
的“斗争精神”,分别在三篇文章中的任何一部分里都有体现,并不局限在一两个句子中。

但比喻却不同,它范围比较小,一般局限于一两个句子中。例如,在峻青的《秋色赋》里
有这么一个比喻句:“山楂树上缀满了一颗颗红玛瑙似的果子。”这里作为喻体的“红玛瑙”
仅仅指的是山楂果子,而对桃、梨、香蕉就不起比喻作用了。

但有时比喻和象征又融在一起,需要仔细分辨。如舒婷的《致橡树》,单从诗的意象上看,
橡树的高大、伟岸,木棉的红硕、柔美,与恋爱中的男女形象形似,构成比喻,使意象得
到诗化,增强了语言的明丽与隽永;若从整体上看,这两个比喻优化组合在一起,进而表
达出一种抽象的爱情观,即相互独立、平等,又互相依靠。如此寓意于象,又象意交融,
构成了象征,让意蕴在神似中得到了合理的引申与升华。

\subsection{对偶与对比的区别}

\subsubsection{对偶}

\index{修辞!对偶}
对偶是用字数相等、结构相同、意义对称的一对短语或句子来表达两个相对或相近意思的
一种修辞手法。宽式对偶,不强求平仄协调,也允许有重复。如:
\begin{equotation}
  剪纸灯谜,描绘城乡风物;秧歌花鼓,传播时代精神。
\end{equotation}
上句主语“剪纸灯谜”对应句主语“秧歌花鼓”;上句谓语“描绘城乡风物”对应下句谓语“传播
时代精神”。结构相同,字数相等,意义对称。句式更整齐,富有节奏感与音乐美,富有感
染力,生动形象地描绘出新时代传统文化的新特色、新气象。

\paragraph{按内容,对偶可分为正对、反对、串对。}

\subparagraph{正对}从两个角度、两个侧面说明同一事理,表示相似、相关的关系。

\begin{equotation}
  日出江花红胜火,春来江水绿如蓝。(白居易《忆江南》)
\end{equotation}

\subparagraph{反对}上下句表示一般的相反关系或矛盾对立关系。

\begin{equotation}
  忧劳可以兴国,逸豫可以亡身。(欧阳修《伶官传序》)
\end{equotation}

\subparagraph{串对}上下句在意义上具有承接、递进、因果、假设、条件等关系的对偶形
式,也叫“流水对”,串对中上下联不能颠倒次序。

\begin{equotation}
  一着不慎,满盘皆输。

  欲穷千里目,更上一层楼。(王之涣《登鹳雀楼》)

  山重水复疑无路,柳暗花明又一村。(陆游《游山西村》)
\end{equotation}

\paragraph{按形式,对偶可分为工对和宽对。}

\subparagraph{工对}就是字数、词性、结构、平仄、用字等均按对仗要求。例如:

\begin{equotation}
  青山横北郭,白水绕东城。(李白《送友人》)
\end{equotation}
上例属于“严对”。因为这两句都是主谓结构相对;“白水”与“青山”,“东城”与“北郭”两组
名词性词组相对,对得十分工整。上句是平平平仄仄,下句则是仄仄仄平平,平仄对仗也
很严整。

\subparagraph{宽对}就是基本符合对仗要求,但某些方面稍有出入,也就是形式要求稍宽
松一点。例如:

\begin{equotation}
  谦虚使人进步,骄傲使人落后。
\end{equotation}
上例在字面上不必重复,平仄上也不讲究,属于“宽对”。

对偶的作用:对偶的运用,使得中国的骈文、律诗、对联等创造了光辉灿烂的成果。对偶
的主要作用有:

\begin{enumerate}
  \item 形式整齐,结构对称,可以收到一种均衡的美感效果。

  \item 词句凝练概括,富有表现力,能够把相关事物间的关系表现得集中鲜明;使对立
    事物间的对比强烈,褒贬分明。

  \item 节奏鲜明,音韵和谐,读来朗朗上口,便于传诵记忆。
\end{enumerate}

\subsubsection{对比}

\index{修辞!对比}
把两种不同事物或者同一事物相反或相对的两个方面放在一起相互比较,叫对比,也叫“对
照”。对比可以使客观存在的对立统一关系表达得更集中、更鲜明突出。

对比可以分成两体对比和一体两面对比两类。

\subparagraph{两体对比}把两种根本对立的事物放在一起进行对照,使好的显得更好,坏
的显得更坏;大的显得更大,小的显得更小,等等。例如:
\begin{equotation}
  看文学大师们的创作,有时用简:惜墨如金,力求数字乃至一字传神;有时使繁:用墨
  如泼,汩汩滔滔,虽十、百、千字亦在所不惜。(周先慎《简笔与繁笔》)
\end{equotation}
“简”与“繁”形成鲜明对比。

\subparagraph{一体两面对比}把同一事物的正反两个方面放在一起来说,能把事理说得更
透彻、更全面。例如:
\begin{equotation}
  时间是勤奋者的财富,创造者的宝库;

  时间是懒惰者的包袱,浪费者的坟墓。
\end{equotation}
以比喻的手法鲜明透彻地说明了时间对四种人的不同意义和效应。

\subparagraph{对比的作用}对比的修辞作用,是揭示对立意义,使事理和语言色彩鲜明。
不同类型的对比,作用又各有特点。两体对比,揭示好同坏、善同恶、美同丑的对立,使
人们在比较中鉴别。一体两面对比,揭示事物的对立面,反映事物内部既矛盾又统一的辩
证关系,使人们全面地看问题。

\subsubsection{对偶与对比的区别}

\begin{enumerate}
  \item 基本特征不同:对偶重“偶”,基本特征是对称;对比重“比”,基本特征是对立。

  \item 表达效果的侧重不同。对偶重结构形式,主要是从结构形式上说的,它要求结构
    对称、字数相等;对比重内容意义,它要求意义相反或相对。

  \item 对偶里的“反对”就意义说是对比,就形式说是对偶,如上例的“时间是勤奋者的财
    富,创造者的宝库;时间是懒惰者的包袱,浪费者的坟墓”,这是辞格的兼属现象。当
    然,对比不一定都是对偶,这要取决于它的结构形式是否对称。例如“青山有幸埋忠骨,
    白铁无辜铸佞臣”句,就含有对比和对偶两种修辞。
\end{enumerate}

\subsection{对比与映衬的区别}

\subsubsection{映衬}

\index{修辞!映衬}
为了突出主体事物,用类似的或相反的、相异的事物作陪衬,这种修辞叫映衬,也叫“衬
托”。如“蝉噪林愈静,鸟鸣山更幽”,“僧敲月下门”,“月出惊山鸟”等,都是以动衬静,属
于映衬的修辞。

映衬可分正衬和反衬两类。

\subparagraph{正衬}正衬就是利用同主体事物相类似的事物作陪衬。正衬包含以动衬动、
以静衬静、以乐衬乐、以美衬美等。例如:
\begin{equotation}
  桃花潭水深千尺,不及汪伦送我情。(李白《赠汪伦》)
\end{equotation}
用桃花潭水之深正衬汪伦对“我”的情谊之深。

\subparagraph{反衬}反衬就是从反面衬托,利用同主体事物相反或相异的事物作陪衬。反
衬包含以动衬静、以静衬动、以苦衬乐、以乐衬苦、以丑衬美、以美衬丑等。例如:
\begin{equotation}
  雨中的雪花陡然间增多了,远远近近愈加变得模模糊糊。城市寂静无声。隐约听见很远
  的地方传来一声公鸡的啼鸣,给这灰蒙蒙的天地间平添了一丝睡梦般的阴郁。(路遥
  《平凡的世界》)
\end{equotation}
“雪花”和“公鸡啼叫”的相关描写,以动衬静,衬托城市的寂静。

\subsubsection{对比与映衬的区别}

突出正面或反面或相异的事物的主体,表达强烈的思想感情,使文章的中心思想深化。有
了陪衬的事物,被陪衬的事物才会显得突出,才能得到充分的说明。

映衬为了突出主要事物,用类似的事物或反面的有差别的事物作陪衬,因此映衬有主次之
分,陪衬事物是用来突出被陪衬事物的。对比是用来突出对立事物,两种对立的事物并无
主次之分,而是相互依存的。对比的诸项一定要相反相成、相互对立、泾渭分明,形成矛
盾的统一体。

映衬的作用主要在于突出两个事物中的一个,表达强烈的思想感情,所谓“红花还须绿叶
扶”,用绿叶衬红花,使红花更红,以达到“烘云托月”的美感效果。对比的作用是使两个事
物或两个方面在对比中互相突出。例如:

\begin{enumerate}
  \item \label{item:劝学}骐骥一跃,不能十步;驽马十驾,功在不舍。(荀子《劝学》)

  \item \label{item:有的人}有的人活着/他已经死了;有的人死了/他还活着。(臧克家
    《有的人》)

  \item \label{item:红楼梦}先到了潇湘馆。一进门,只见两边翠竹夹路,土地下苍苔布
    满,中间羊肠一条石子漫的路。(曹雪芹《红楼梦》)

  \item \label{item:爱莲说}晋陶渊明独爱菊。自李唐来,世人甚爱牡丹。予独爱莲之出
    淤泥而不染,濯清涟而不妖,中通外直,不蔓不枝,香远益清,亭亭净植,可远观而
    不可亵玩焉。(周敦颐《爱莲说》)
\end{enumerate}

例~\ref{item:劝学} 是对比,“骐骥”与“驽马”进行对比,强调“不舍”的重要性;
例~\ref{item:有的人} 是对比,用两种人的价值对比,批判行尸走肉,赞扬虽死犹生;
例~\ref{item:红楼梦} 是映衬,用潇湘馆门前的“翠竹”,“苍苔”等景物正衬林黛玉孤傲的性格;
例~\ref{item:爱莲说} 是映衬,用“菊”正衬“莲”,用“牡丹”反衬“莲”,表达莲花的高洁和孤
寂。

\subsection{排比与反复的区别}

\subsubsection{排比}

\index{修辞!排比}
把三个或者三个以上结构相同或相似、意义相关、语气一致的词组或句子排列起来,形成
一个整体,使语势得到增强,感情得到加深的修辞格。例如:
\begin{equotation}
  岛拉,我的女儿,你曾多么美丽!你美丽如悬挂在弗拉山岗上的皓月,洁白如天空飘下
  来的雪花,甜蜜如芳馨的空气!(歌德《少年维特的烦恼》)
\end{equotation}
“悬挂在弗拉山岗上的皓月,洁白如天空飘下来的雪花,甜蜜如芳馨的空气”三个短句,结
构相同,意义相关,语气一致,强调女子之美,所以是排比。

\begin{equotation}
  翩若惊鸿,婉若游龙,《洛神水赋》的舞者化身洛神,或拂袖起舞,或拨裙回转,或刚
  劲有力,或娉婷袅娜,整个舞蹈使端午祈福的美好愿景与惊艳众人的视觉效果高度统一。
\end{equotation}
结构相同,都是由“或”字领出一个四字短语;意思相关,多角度描写舞者的动作;语气一
致,都有赞美之意;句式整齐,语言凝练,节奏鲜明,富有韵律美,有助于凸显舞蹈之美。

排比的各个项目之间的关系,有的是并列的,排比的项目之间的关系是平等的联合的关系。
例如:
\begin{equotation}
  水里的游鱼是沉默的,陆地上的兽类是喧闹的,空中的飞鸟是歌唱着的。但是,人类却
  兼有海里的沉默、地上的喧闹与空中的音乐。(泰戈尔《飞鸟集》)
\end{equotation}

有的是承接的,排比的项目之间的关系有逻辑上先后之分,不可以随意变动。例如:
\begin{equotation}
  古之欲明明德于天下者,先治其国;欲治其国者,先齐其家;欲齐其家者,先修其身;
  欲修其身者,先正其心;欲正其心者,先诚其意;欲诚其意者,先致其知;致知在格物。
  (《礼记·大学》)
\end{equotation}

有的是递进的,排比的项目之间有阶梯式关系。例如:
\begin{equotation}
  一年之计,莫如树谷;十年之计,莫如树木;终身之计,莫如树人。(《管子·权修》)
\end{equotation}

排比的作用:可以使文章条理清晰,显得气势磅礴;使语言畅达明快,富于节奏感,适宜
于强烈感情的抒发。例如:
\begin{equotation}
  人类是一件多么了不得的杰作!多么高贵的理性!多么伟大的力量!多么优美的仪表!
  多么文雅的举动!在行为上多么像一个天使!在智慧上多么像一个天神!宇宙的精华!
  万物的灵长!(莎士比亚《哈姆雷特》)
\end{equotation}
用四个整齐的“多么”句子,强烈地表达出莎士比亚借哈姆雷特之口对“宇宙的精华!万物的
灵长!”人类的盛赞。

\subsubsection{反复}

\index{修辞!反复}
为了突出某个意思、强调某种感情,特意重复某个词语或句子,这种辞格叫反复。

从类型上讲,反复可分为连续反复和间隔反复两类。

连续反复是接连重复相同的词语或句子,中间没有其他词语出现。例如:

\begin{equotation}
  周总理,我们的好总理,你在哪里呵,你在哪里?(柯岩《周总理,你在哪里》)
\end{equotation}

间隔反复是相同词语或句子的间隔出现,即有其他词语或句子将反复的部分隔开。例如:

\begin{equotation}
  风雪一天比一天大,人们的干劲一天比一天猛,砍下的毛竹一天比一天堆得高,为竹滑
  道修的架在两座高山之间的竹桥,也一天比一天往上去。(袁鹰《井冈翠竹》)
\end{equotation}

“一天比一天”是短语的间隔反复。

有时连续反复和间隔反复交错使用,可以表现感情由一般到强烈的发展变化。例如:

\begin{equotation}
  沉默呵!沉默呵!不在沉默中爆发,就在沉默中灭亡。(鲁迅《记念刘和珍君》)
\end{equotation}

反复的作用:用同一的语句,一再表现强烈的情思;调节音节、增强节奏。

\subsubsection{排比与反复的区别}

\begin{enumerate}
  \item 排比是为了加强语势,反复是为了突出强调某种感情。

  \item 排比是把三句或三句以上结构相同的句子连在一起,反复是把某个词语或句子重
    复两次以上。

  \item 排比中有部分提示词语相同,而反复则是词语或句子完全相同。例如:

    \begin{equotation}
      栾恩杰从导弹研究的技术员到中国探月工程首任总指挥,经历过各种各样的失败,
      大到火箭里面的特殊装置出现问题,小到一个插头插错了,这些失败意味着什么?
      意味着多少个日夜的辛苦付之一炬,意味着接下来的工作更加艰苦卓绝,意味着你
      在世界的航天格局中可能突然之间换了赛道,栾恩杰认为:失败也是在给我们上课,
      当问题一一解决的时候,成功就在我们前面。
    \end{equotation}
    三个“意味着”句式整齐,节奏感强,且有递进效果,加强语势,突出航天研发过程的
    艰难,侧面烘托出航天工作者的坚韧顽强和航天工作的重要性。

    \begin{equotation}
      红酥手,黄縢酒,满城春色宫墙柳。东风恶,欢情薄,一怀愁绪,几年离索。错!
      错!错!(陆游《钗头凤》)
    \end{equotation}
    上阕三个“错”接连反复,强烈地展现了陆游对于与唐氏婚姻的结束的无限悔恨、自责
    之情。

\end{enumerate}

\subsection{设问与反问的区别}

\subsubsection{设问}

\index{修辞!设问}
设问是无疑而问,自问自答,以引导读者或听众注意和思考问题的修辞手法。例如:

\begin{equotation}
  如果说个体建筑的宏伟壮丽主要表现为西方古建筑艺术,那么中国古建筑艺术则主要表
  现为群体建筑的博大壮观。建筑群体的组合采取的是什么形式呢?一般来讲,它采取的
  是由单幢房屋围合成的院落形式,即四合院。不同类型的建筑正是由这种最基本的四合
  院单位组合而成的。
\end{equotation}
先问“建筑群体的组合采取的是什么形式呢”,然后回答“一般来讲,它采取的是由单幢房屋
围合成的院落形式,即四合院”;通过自问的方式引起读者注意,启发读者思考。强调了句
子的主要内容,突出建筑群体组合的形式特点。设问句结构上具有承上启下的过渡作用。

设问主要有以下三种形式。

\subparagraph{一问一答}例如:

\begin{equotation}
  寂静为什么可怕?因为寂静邻于死亡,有时候也许就是死亡。身体死亡了,在尸躯本身
  无所谓可怕;看见尸躯的人也许觉得可怕,然而这只是原始的恐惧心理,仔细一想,也
  就没有什么可怕。只有身体机能还存在,而精神已经死亡了,才是真正的可怕。(叶圣
  陶《冲破那寂静》)
\end{equotation}
作者自问:“寂静为什么可怕?”然后自答:“因为寂静邻于死亡,有时也许就是死亡。”并
进一步强调,“只有身体机能还存在,而精神已经死亡了,才是真正的可怕”。

\subparagraph{几问一答}例如:

\begin{equotation}
  啊,是谁,这么早就把那亲爱的令人心醉的乡音送到我的耳畔?是谁,这么早就用他那
  吱吱哇哇的悦耳动听的音乐唤来了玫瑰色的黎明?是一个青年人。(峻青《乡音》)
\end{equotation}
用“是谁”两次提问,最后一答,表达对“青年人”的赞美。

\begin{equotation}
  是谁在乌黑的窗棂上铺展一派春意?是谁在漫天飞雪里开出一枝红梅?是谁经过剪刀轻
  灵的裁剪,给家中增添喜气洋洋的期待?——是窗花。(张金凤《窗花舞》)
\end{equotation}
运用了三个设问,引出了本文的写作对象“窗花”,引发读者对窗花作用的思考,强调了作
者对窗花的喜爱之情。

\subparagraph{不答式设问}提出的问题不需要回答,读者自会明白的设问叫不答式设问。
这种设问只是为了某种强调、提示以引起注意和联想。例如:
\begin{equotation}
  今天这里有没有特务?你站出来!是好汉的站出来!(闻一多《最后一次讲演》)
\end{equotation}
从句子语境可以推知会场是有特务的,所以此处设问不需要回答,用“今天这里有没有特
务?”设问是为了让听众思考特务的反动本质。

设问的作用:引起读者注意,启发读者思考;突出某些内容,使文章起波澜,有变化。

\subsubsection{反问}

\index{修辞!反问} \index{修辞!反诘}
反问也叫反诘,用疑问的形式表达确定的意思,无疑而问,明知故问,这种辞格叫反问,
又叫激问。反问只问不答,把要表达的确定意思包含在问句里。否定句用反问语气说出来,
就表达肯定的内容;肯定句用反问语气说出来,就表达否定的内容。例如:
\begin{equotation}
  现在我所谓希望,不也是我自己手制的偶像么?(鲁迅《故乡》)
\end{equotation}
作者在自问的同时,强调自己所谓的希望,也是自己手制的,无关社会与他人。

\paragraph{反问主要有两种}

\subparagraph{用肯定句表达否定的内容}例如:
\begin{equotation}
  人同此心,心同此理,凡属黄帝子孙,谁愿成为民族的千古罪人?
\end{equotation}

\subparagraph{用否定句表达肯定的内容}例如:
\begin{equotation}
  在这薄霭和微漪里,听着那悠然的间歇的桨声,谁能不被引入他的美梦去呢?(朱自清
  《桨声灯影里的秦淮河》)
\end{equotation}

\paragraph{反问的作用}加强语气,发人深省,激发读者的感情,加深读者的印象,增强
文章的气势和说服力。例如:
\begin{equotation}
  不是有无数人在讴歌那光芒四射的朝阳、四季常青的松柏、庄严屹立的山峰、澎湃翻腾
  的海洋吗?不是有好些人在赞美那挺拔的白杨、明亮的灯火、奔驰的列车、崭新的日历
  吗?(秦牧《土地》)
\end{equotation}
这里连用两个反问,并套用了排比,既起到强调语意、强化语势的作用,又使语言表达跌
宕有致,同时也抒发了强烈的感情。此外,还使语言富有整齐美和节奏感。

\subsubsection{设问与反问的区别}

设问不表示肯定什么或否定什么,而反问则明确地表示肯定或否定的内容。

设问主要是提出问题,引起注意,启发思考,然后自己回答;而反问主要是加强语气,用
确定的语气表明自己的思想,答在问中。例如:
\begin{equotation}
  今天我们为什么要读经典?为什么如此强调赓续文脉香火?\linebreak[1]《典籍里的中
  国》的开场白,或许可为答案,“知道我们的生命缘起何处,知道我们的脚步迈向何方”。
  那些在血脉和文脉中代代传承的文化基因,形塑了我们今天的精神世界和价值体系,影
  响着我们的思想方式和行为方式。
\end{equotation}
先用“今天我们为什么要读经典?为什么如此强调赓续文脉香火?”发问,然后回答:“知道
我们的生命缘起何处,知道我们的脚步迈向何方。”两问一答。此例中用设问修辞引发读者
思考阅读经典、传承传统文化的意义;引出后文,强调传统文化对今天人们思想和行为的
影响。

\begin{equotation}
  难道这还用解释吗,密哈益·沙维奇?难道这不是理所应当吗?如果教师骑自行车,那还
  能希望学生做出什么好事来?他们所能做的就只有倒过来,用脑袋走路了!既然政府没
  有发出通告,允许做这种事,那就做不得。(契诃夫《装在套子里的人》)
\end{equotation}
运用反问的修辞,强调这件事不用解释,是理所应当的,教师骑自行车的话,就不要希望
学生能做出什么好事。答在问中。

\subsection{比拟与比喻的区别}

\subsubsection{比拟}

\index{修辞!比拟}
根据想象把物当作人写或把人当作物写,或把甲物当作乙物来写,这种辞格叫比拟。被比
拟的事物称为“本体”,用来比拟的事物称为“拟体”。拟体一般不出现,只把适用于拟体的
词用在本体上。例如:
\begin{equotation}
  新疆属于绿洲农业区,干旱少雨,为了让棉花吃好喝好长得好,就要进行科学的水肥管
  理。
\end{equotation}
“为了让棉花吃好喝好长得好”一句使用了比拟修辞。“棉花”是“本体”,“人”是“拟体”,没
有出现。把棉花吸收足够的水分和肥料才能长得好,比拟成人吃好饭喝好水才能健康成长,
暗含作者对棉花的喜爱之情。

\begin{equotation}
  那肥大的荷叶下面,有一个人的脸,下半截身子长在水里。那不是水生吗?(孙犁《荷
  花淀》)
\end{equotation}
把水生当作植物来写,“长在水里”是荷的状态,人的下半截身子“长在水里”,跟荷梗一样,
给人以壮美的形象。这里使用了比拟的修辞,流露出作者对人物的喜爱和赞美。

比拟在一般情况下分为拟人和拟物两种。

\index{修辞!拟人}
\subparagraph{拟人}把物当作人来描写,使物具有人的动作行为、思想感情、神情样貌的
一种比拟。拟人,是人格化的手法,可以把无生命的物写得栩栩如生,也可以把有生命的
物写得可爱可憎。可分为动物拟人、植物拟人、具体事物拟人、抽象事物拟人几种。例如:
\begin{equotation}
  人面不知何处去,桃花依旧笑春风。(崔护《题都城南庄》)(把“桃花”拟作人)

  待到山花烂漫时,她在丛中笑。(毛泽东《卜算子·咏梅》)(把“山花”拟作人)

  海睡熟了。大小的岛拥抱着,偎依着,也静静地恍惚入了梦乡。星星在头上眨着慵懒的
  眼睑,也像要睡了。(鲁彦《听潮》)(把“大海”,“岛”,“星星”拟作人)
\end{equotation}

\index{修辞!拟物}
\subparagraph{拟物}一般情况下分为人物拟物和事物拟物两小类。例如:
\begin{equotation}
  我到了自家的房外,我的母亲早已迎着出来了,接着便飞出了八岁的侄儿宏儿。(鲁迅
  《故乡》)
\end{equotation}
本句以人拟物。人是不会飞的,“飞”是某些动物所具有的能力,宏儿被当作会“飞”的鸟等
动物来描写,写出了其心情的急切和动作的轻快。
\begin{equotation}
  沙漠竟已狂虐到了这样地步,它正在无情地吞噬着一座孤立的大山!(玛拉沁夫《沙漠,
  我将不再赞美你》)
\end{equotation}
本句以事物拟物。把“沙漠”当作生物来描写,所以能“狂虐”并“吞噬”大山。

比拟的作用:让读者感受到作者强烈的情感,启发读者想象,从而引发共鸣;生动形象、
神形毕现、栩栩如生。

\subsubsection{比拟与比喻的区别}

\begin{enumerate}
  \item 比拟是仿照拟体的特征“摹”写本体,重点在拟;比喻是用喻体比喻本体,重点在
    “喻”。

  \item 比拟中,本体和拟体彼此交融,浑然一体,本体必须出现,拟体一般不出现;比
    喻的本体和喻体一主一从,本体或出现或不出现,而喻体必须出现。
\end{enumerate}

总之,比喻是“以此喻彼”,其修辞特点往往体现在名词或名词性短语上,且喻体必须出现;
比拟是“拟此为彼”,其修辞特点往往体现在动词上,而拟体一般不出现。例如:

\begin{equotation}
  从未见过开得这么盛的藤萝,只见一片辉煌的淡紫色,像一条瀑布,从空中垂下。(宗
  璞《紫藤萝瀑布》)
\end{equotation}
本句把茂盛的藤萝比作瀑布,生动形象地表现了藤萝的茂盛,表达了作者对藤萝花的赞美
之情。

\begin{equotation}
  东西长安街成了喧腾的大海。(袁鹰《十月长安街》)
\end{equotation}
本句使用了暗喻的修辞手法。“长安街”为本体,“大海”为喻体。

\begin{equotation}
  我似乎打了一个寒噤;我就知道我们之间已经隔了一层可悲的厚障壁了。(鲁迅《故
  乡》)
\end{equotation}
本句使用了借喻的修辞方法,本体和比喻词没有出现,只出现了喻体“厚障壁”。

\begin{equotation}
  小草偷偷地从土里钻出来。(朱自清《春》)
\end{equotation}
本句使用了拟人的修辞手法,“本体”是“小草”。“拟体”是“人”,没有出现。本句把小草拟
作人,使小草无声无息却在眨眼间长出来的可爱姿态变得鲜明生动,也使文章更加充满趣
味,表达了对小草、对春天的喜爱和赞美之情。

\begin{equotation}
  我把青春栽种在这里,尽管时值严冬,却终于蔚然成林。(孔捷生《绿色的蜜月》)
\end{equotation}
本句使用了拟物的修辞手法,“本体”是“青春”。拟体是“植物”,没有出现,通过动词“栽种”
体现。

\subsection{对仗与对偶的区别}

\subsubsection{对仗}

\index{修辞!对仗}
古代的仪仗队是两两相对的,这是“对仗”这个术语的来历。

对仗是律诗、骈文等按照字音的平仄和字义的虚实作成对偶的语句。

由对仗的概念可以知道对仗有以下特点:

\begin{enumerate}
  \item 对仗是在格律诗出现后,融合对偶结构与格律而形成的一种修辞。例如律诗中的
    颔联(第二联)、颈联(第三联)往往要求对仗。

  \item 对仗是诗词、骈文中的对偶,一般用在骈文、诗、词、曲、对联当中。

  \item 对仗有严格的格律要求,讲究平仄,严格用词,上下句不能出现重复的字。
\end{enumerate}

\subsubsection{对仗与对偶的区别}

对仗是在对偶的结构基础上,进一步完善发展而来的,它除了具备对偶的“句式的结构相同
或基本相同,字数相等,意义上密切相关联的一对短语或句子”等特点外,还要具备以下特
点:

\begin{enumerate}
  \item 对仗一般用在骈文、诗、词、曲、对联当中,如果不是出现在这些文体中,一般
    不认为是对仗。就是说,对仗是一种格律要素。对偶的使用则不限于诗、词、骈文等
    以整句为表现形式的文字中。
  \item 对仗一般要平仄相对,上下句不能出现重字。对偶没有这两个严格的要求。
\end{enumerate}

总之,对仗的使用范围不如对偶宽,对仗的格律要求比对偶严。因此对仗不等同于对偶,
在讲诗、词、骈文的韵律时,如果对偶句式符合对仗的要求,最好称其为“对仗”。例如:
\begin{equotation}
  无边落木萧萧下,不尽长江滚滚来。(杜甫《登高》)
\end{equotation}
这两句是对仗。首先,上下两句句子结构相同,词的构成也相同。“无”对“不”,是副词;
“边”对“尽”,是形容词;“无边”对“不尽”,又都是偏正式;“落木”对“长江”,都是偏正式
名词;“萧萧”对“滚滚”,都是叠词;“下”对“来”,都是动词。其次,上下句平仄相对。“无
边落木萧萧下”为“平平仄仄平平仄”;“不尽长江滚滚来”为“仄仄平平仄仄平”。

\subsection{回环与顶真的区别}

\subsubsection{回环}

\index{修辞!回环}
把前后语句组织成穿梭一样的循环往复的形式,用以表达不同事物间的有机联系,这种辞
格叫回环。通俗地说,回环,就是重复前一句的结尾部分,作为后一句的开头部分,又回
过头来用前一句开头部分作后一句结尾部分。例如:
\begin{equotation}
  在她,工作就是游戏,游戏就是工作。(冰心《关于女人》)

  谭婶婶学习一个月回来,挟了两个卫生包,身上被单一扎,她就是产院,产院就是她,
  到处给人接生,到处宣传科学,和旧的接生婆展开斗争。(茹志鹃《静静的产院》)
\end{equotation}

在现代汉语的回环修辞文本建构中,多数情况都是不太追求形式上的严整,而只求形式上
大体首尾衔接、往复成章的意趣而已。例如:
\begin{equotation}
  远远的街灯明了,好像是闪着无数的明星。天上的明星现了,好像是点着无数的街灯。
  (郭沫若《天上的街市》)
\end{equotation}

回环修辞深受大众喜欢,我们耳熟能详的很多话语也使用了回环的修辞手法。例如:
\begin{equotation}
  疑人不用,用人不疑。

  生产促进科学,科学促进生产。
\end{equotation}

有些对联使用了回环,因而妙趣横生。例如:
\begin{equotation}
  过苦年,苦年过,过年苦,苦过年,年去年来今变古;读书好,书好读,读好书,书读
  好,书田书舍子而孙。

  上海自来水来自海上;中国出人才人出国中。
\end{equotation}

回环的作用:回环可使语句整齐匀称,在视觉上、语感上都给人以循环往复的美感。回环
能揭示事物相互依存或者相互排斥的辩证关系,使语意精辟警策,以加深读者、听者对客
观事物的认识和理解。例如:
\begin{equotation}
  我们这里只有医生和病人,病人和医生。(电影《马门教授》)
\end{equotation}
“医生和病人,病人和医生”使用了回环,“医生”和“病人”被回环强调,强化了马门教授对
“医生”身份的尊重和拒绝纳粹分子的义正词严。

\subsubsection{顶真}

\index{修辞!顶真}
用上一句结尾的词语或句子作下一句的起头,使前后的句子头尾蝉联,上递下接,这种辞
格叫顶真。顶真是汉语传统的修辞格之一,经常出现在各种文体的文章中。例如:
\begin{equotation}
  \begin{enumerate}
    \item \label{item:再别康桥} 夏虫也为我沉默,沉默是今晚的康桥。(徐志摩《再
      别康桥》)

    \item \label{item:论语} 名不正则言不顺,言不顺则事不成,事不成则礼乐不兴,
      礼乐不兴则刑罚不中。(《论语·子路》)

    \item \label{item:高干大} 咱们做的事越多,老百姓就来的越多;老百姓来的越多,
      咱们的力量就越大;咱们的力量越大,往后的事也就越多!(欧阳山《高干大》)
  \end{enumerate}
\end{equotation}
例~\ref{item:再别康桥} 是词与词顶真,例~\ref{item:论语} 是短语与短语顶真,
例~\ref{item:高干大} 是句子与句子顶真。

顶真的作用:首先,使议事说理更准确、严谨、周密。其次,语气连绵,音律流畅,给人
以流畅明快的蝉联美感。

\subsubsection{回环与顶真的区别}

顶真和回环在头尾顶接这一点上相似,但又有根本上的不同。

结构上:回环把前后语句组织成穿梭一样的循环往复的形式,其轨迹是圆周形,语言形式
是按照逆序排列,形式为“A→B,B→A”。顶真是上一句结尾的词语或句子做下一句的起头,
上递下接,其连接关系是直线型,形式为“→A,A→B,B→C……”。

一个回环模块只有两个语言片段,而顶真则不限于两个。例如:

\begin{equotation}
  他比先前并没有什么大改变,单是老了些,但也还未留胡子,一见面是寒暄,寒暄之后
  说我“胖了”,说我“胖了”之后即大骂其新党。(鲁迅《祝福》)
\end{equotation}
这是顶真修辞。“寒暄”,“说我‘胖了’,”构成“→A(寒暄),A(寒暄)→B(说我“胖了”),
B(说我“胖了”)→C……”顶真结构。

\begin{equotation}
  挽闻一多联:一个人倒下去,千万人站起来;千万人站起来,一个人倒下去。
\end{equotation}
这是回环修辞。“一个人倒下去”和“千万人站起来”构成“A→B,B→A”回环结构。

\subsection{层递和排比的区别}

\subsubsection{层递}

\index{修辞!层递}
根据事物的逻辑关系,连用内容递升或递降的语句,表达层层递进的事理,这种辞格叫层
递。例如:

\begin{equotation}
  保卫家乡,保卫黄河,保卫华北,保卫全中国。
\end{equotation}
写出了“家乡”,“黄河”,“华北”和“中国”的领属关系,范围逐渐扩大,充满保家卫国的激
情,有强烈的感染力。(《保卫黄河》歌词)

\begin{equotation}
  听说四川有一只民谣,大略是“贼来如梳,兵来如篦,官来如剃”的意思。(鲁迅先生
  《谈金圣叹》)
\end{equotation}
比喻句“贼来如梳,兵来如篦,官来如剃”,“梳”,“篦”,“剃”,词义上程度逐层加重,用
三者的差别来强调官比贼和兵更加狠毒、残酷,突出民众对封建官府的仇恨和作者对社会
黑暗的痛斥。

\begin{equotation}
  不是被什么声音吵醒,而是因为静,寂静,绝寂静。(邵燕祥《大峡谷去来》)
\end{equotation}
“静,寂静,绝寂静”,从静到寂静再到绝寂静,是程度上由浅入深的递升。

\begin{equotation}
  一鼓作气,再而衰,三而竭。(左丘明《曹刿论战》)
\end{equotation}
采用递降的层递修辞,强调了作战时勇气的重要性。

层递分为递升和递降两类。

\index{修辞!递升}
\subparagraph{递升}按照事物的发展状况排列,由小到大,由少到多,由低到高,由短到
长,等等。例如:

\begin{equotation}
  时间一天一天地过去,一月一月地过去,一年一年地过去,真理老人所撒的种子,也一
  天一天地生长,一月一月地开花,一年一年地结果。一粒种子变成一百粒,一万粒,千
  万粒……(圣野《两袋种子》)
\end{equotation}
两次用了表示时间由短到长的层递。“生长→开花→结果”,反映了种子生活的规律;“一百粒
→一万粒→千万粒”,种子数量由少到多。两个层递深刻地揭示了“真理老人撒播快乐和幸福,
最后战胜法西斯”的道理,逻辑说明关系十分紧密。

递升式层递,在古代汉语修辞中也有很多经典文本。例如:
\begin{equotation}
  知之者不如好之者,好之者不如乐之者。(《论语·雍也》)

  天时不如地利,地利不如人和。(《孟子·公孙丑下》)
\end{equotation}

\index{修辞!递降}
\subparagraph{递降}递降就是按照事物的下降状况排列,由大到小,由多到少,由高到低,
由长到短,等等。例如:
\begin{equotation}
  他一直是魂思梦想着打飞机,眼前飞过一只雁,一只麻雀,一只蝴蝶,一只蜻蜓,他都
  要拿枪瞄瞄。(郑直《激战无名川》)
\end{equotation}

使用递降的层递修辞,“他”瞄准练习的动物——“雁”,“麻雀”,“蝴蝶”,“蜻蜓”,形体由大
到小递降,这个层递修辞形象地突出了“他”苦练瞄准本领的认真精神和想打敌机的迫切心
情。

递降式层递,古代诗文中也有经典文本。例如:
\begin{equotation}
  民为贵,社稷次之,君为轻。(《孟子·尽心下》)
\end{equotation}

递升递降都是相对的,不能截然分开。有时同一例句,从某个角度看是递升,换一个角度
看也能是递降。例如:
\begin{equotation}
  少年听雨歌楼上,红烛昏罗帐。壮年听雨客舟中,江阔云低,断雁叫西风。而今听雨僧
  庐下,鬓已星星也。悲欢离合总无情,一任阶前点滴到天明。(蒋捷《虞美人·听雨》)
\end{equotation}

由少年到壮年再到老年的年龄递升,与心境由浪漫到漂泊再到凄凉的递降相形对比,突出
强调了作者心境的每况愈下和晚景的凄凉。

\subparagraph{层递的作用}层递借步步推进,意义上的逐层深入,使人的认识层层深化,
对表达的事理产生深刻的印象。

少而好学,如日出之阳;壮而好学,如日中之光;老而好学,如炳烛之明。(刘向《说苑·
建本》)

从年龄段看,是递升的层递修辞。从学习效果看,是递降的层递修辞。年龄段“少”,“壮”,
“老”的学习的成效分别像“日出之阳”,“日中之光”到“炳烛之明”,学习成效由高到低递降,
使读者认识到年龄越大,学习效果越差,从而强调“读书趁年少”的道理。

\subsubsection{层递与排比的区别}

层递着重看意义关系,着眼于意义上的等次性(级差性),构成层递的几个语句在内容上
必须是递升或递降的;排比不要求意义上的等次性,而是着眼于内容上的并列性,构成排
比的内容是一个问题的几个方面,或相关的几个问题。

层递不受语言结构的约束,在结构上不强调相同或相似,往往不用相同的词语;排比在结
构上必须相同或相似,往往要用相同的词语。例如:

\begin{equotation}
  这是未庄赛神的晚上。这晚上照例有一台戏,戏台左近,也照例有许多的赌摊。做戏的
  锣鼓,在阿Q耳朵里仿佛在十里之外;他只听得桩家的歌唱了。他赢而又赢,铜钱变成角
  洋,角洋变成大洋,大洋又成了叠。(鲁迅《阿Q正传》)
\end{equotation}
这是层递修辞,“铜钱”到“角洋”到“大洋”到“叠”,钱的单位由小到大,有明显的层递关系。

\begin{equotation}
  他们的品质是那样的纯洁和高尚,他们的意志是那样的坚韧和刚强,他们的气质是那样
  的淳朴和谦逊,他们的胸怀是那样的美丽和宽广!(魏巍《谁是最可爱的人》)
\end{equotation}
这是排比修辞,句子分别从“品质”,“意志”,“气质”,“胸怀”四个方面来赞美志愿军战士
的伟大品质。四个句子是并列关系,没有主次、轻重、大小的区别。

有观点认为既符合排比结构特点又符合层递的意义要求的句子,就是兼有排比和层递修辞,
这种一句话中兼有多种修辞的现象,就叫“兼格”。例如:

\begin{equotation}
  为了整个班,为了整个潜伏部队,为了这次战斗的胜利,邱少云像千斤巨石一般,趴在
  火堆里一动不动。(李元兴《我的战友邱少云》)
\end{equotation}

它是层递修辞,因它在内容上反映了“班→部队→胜利”这三个事物的递升关系。

它也是排比修辞,结构相同,都用了“为了”介宾短语;语气一致,赞扬了邱少云“趴在火堆
里一动不动”的崇高精神境界;意思密切关联,揭示了邱少云不怕火烧的根本原因。

\subsection{双关与反语的区别}

\subsubsection{双关}

\index{修辞!双关}
利用语音或语义的条件,有意使语句兼顾表面和内里两种意思,言在此而意在彼,这种辞
格叫双关。双关是指一个句子或一个词语有两种不同的含义,一个是表面的意思,一个是
暗含的意思,后者才是所要表达的主要意思。例如:

\begin{equotation}
  “可叹停机德,堪怜咏絮才。玉带林中挂,金簪雪里埋。”(《红楼梦》第五回)
\end{equotation}
谐音双关,“玉带林中挂”暗含谐音“林黛玉”,“金簪雪里埋”中“雪”谐音“薛宝钗”中的“薛”,
暗示了林黛玉、薛宝钗的悲惨命运。

就构成的条件看,双关可以分为谐音双关和意义双关。

\subparagraph{谐音双关}利用音同或音的条件构成谐音双关。例如:

\begin{equotation}
  “杨柳青青江水平,闻郎江上唱歌声。东边日出西边雨,道是无晴却有晴。”(刘禹锡
  《竹枝词》)
\end{equotation}
诗中利用“晴”与“情”同音构成双关,寓意男女之间绵绵的情意。

\subparagraph{意义双关}利用词语或者句子的多义性在特定语境中构成语义双关。例如:

\begin{equotation}
  周繁漪:好,你去吧!小心,现在,(望窗外,自语)风暴就要起来了!(曹禺《雷
  雨》)
\end{equotation}
此处“风暴”非指“刮大风,下暴雨”,而是指激发的矛盾,生死的搏斗。

双关的作用:形成含蓄美;拓展了想象的空间;可以制造讽刺效果。例如:

\begin{equotation}
  此夜曲中闻折柳,何人不起故园情!(李白《春夜洛城闻笛》)
\end{equotation}
笛子吹奏的是一支《折杨柳》曲,它属于汉乐府古曲,抒写离别行旅之苦。“柳”者,“留”
也。李白听着远处的笛声,不由自主地陷入了乡思。

\subsubsection{反语}

\index{修辞!反语}
故意使用与本来意思相反的词语或句子来表达本意,这种辞格叫反语,也叫反话。例如:

\begin{equotation}
  流氓欺乡下老,洋人打中国人,教育厅长冲小学生,都是善于克敌的豪杰。(鲁迅
  《冲》)
\end{equotation}
“克敌的豪杰”是反语,是对流氓、洋人、教育厅长表示强烈的讽刺与嘲弄。

反语可主要分为两类。

\subparagraph{以正当反}用正面的语句去表达反面的意思。例如:

\begin{equotation}
  当三个女子从容地转辗于文明人所发明的枪弹的攒射中的时候,这是怎样的一个惊心动
  魄的伟大呀!中国军人屠戮妇婴的伟绩,八国联军的惩创学生的武功,不幸全被这几缕
  血痕抹杀了。(鲁迅《纪念刘和珍君》)
\end{equotation}
“文明人”,“伟绩”,“武功”都是反义词。“文明人”正是“野蛮人”,“伟绩”,“武功”正是“罪
恶”。

\subparagraph{以反当正}用反面的语句去表达正面的意思。例如:

\begin{equotation}
  黛玉听了,睁开眼,起身笑道:“真真你就是我命中的‘天魔星’!请枕这一个。”说着,
  将自己枕的推与宝玉,又起身将自己的再拿了一个来,自己枕了,二人对面倒下。(曹
  雪芹《红楼梦》第十九回)
\end{equotation}
“天魔星”指给人带来磨难的人。这里的“天魔星”是黛玉对宝玉的昵称,表现宝、黛之间亲
密无间的关系和不可分离的深情。

反语的作用:达到讽刺、幽默的效果;可以表达亲切、喜爱的思想感情。例如:

\begin{equotation}
  几个女人有点失望,也有些伤心,各人在心里骂着自己的狠心贼。(孙犁《荷花淀》)
\end{equotation}
这里的“狠心贼”,并没有什么恶意,相反更能表现出几个女人对自己丈夫深沉的爱。

\subsubsection{双关与反语的区别}

表达意思上,二者都有表里两层意思,但是,反语表里意思相反;双关表里意思相关。

表达形式上,双关的形式可分为两种:一种是利用音同或音近的条件,使词句同时表达双
关的意思;另一种是利用词句的多义现象,使一个词句在特定的语言环境中形成双关。而
反语则只表达一种意思,不像双关那样,一箭双雕。

使用方法上,反语多用在批判、揭露方面,而双关的使用范围则要广泛得多。例如:

\begin{equotation}
  你想,四周围黑洞洞的,还不容易碰壁吗?(周晔《我的伯父鲁迅先生》)
\end{equotation}
本句语义双关。“黑洞洞”暗指社会的黑暗。“碰壁”暗指是与反动势力作斗争时受到的挫折
与迫害。“黑洞洞”,“碰壁”两个词语不是用与本来意思相反的词语或句子来表达本意,而
是用词语的多义性在特定语境中构成语义双关。

\begin{equotation}
  东京也无非是这样。上野的樱花烂漫的时节,望去却也像绯红的轻云。但花下也缺不了
  成群结队的“清国留学生”的速成班,头顶上盘着大辫子,顶的学生制帽的顶上高高耸起,
  形成一座富士山。也有解散辫子,盘得平平的,除下帽来,油光可鉴,宛如小姑娘的发
  髻一般,还要将脖子扭几扭。实在标致极了。(鲁迅《藤野先生》)
\end{equotation}
最后一句为反语。“实在标志极了”,用正面的语句去表达反面的意思,意思其实是“实在丑
陋极了”,表达对“清国留学生”的丑态的批判。

\section{常见修辞错用举例}

\subsection{比喻错用举例}

\subsubsection{比喻的使用要遵循的原则}

\begin{enumerate}
  \item 本体与喻体要有相似性。

  \item 本体与喻体不能同类。

  \item 新颖。

  \item 通俗易懂。
\end{enumerate}

\subsubsection{比喻错用举例}

\begin{equotation}
  无数条淙淙流淌的小河就像大地上的脉搏一样在不停地流动着、跳动着。
\end{equotation}
脉搏只会跳动,不会流动,本体“流动的小河”和喻体“不会流动的脉搏”没有相似性。

\begin{equotation}
  这团烟雾黑得出奇,好像一群大喷泉突然喷水时张成的一个大扇面。
\end{equotation}
把“黑雾”比喻成“一群大喷泉突然喷水时张成的一个大扇面”,本体和喻体没有相似性。

\begin{equotation}
  长辫子姑娘成了舞会上灿烂的明星,所有人的目光都围绕着她转,宛如电子围绕着原子
  核不停地旋转一样。
\end{equotation}
将人们对姑娘的关注比作“电子围绕着原子核不停地旋转一样”不恰当,“电子围绕着原子核
不停地旋转”是机械运动,而长辫子姑娘的舞蹈应该是优美的。此句修辞运用不当。

\begin{equotation}
  东方渐明,天空中出现了万道霞光,犹如一朵含苞欲放的鲜花。
\end{equotation}
霞光的“放射性”和鲜花的“含苞”没有相似性,比喻不当,最后一句可以改成“犹如绽放的烟
花”。

\begin{equotation}
  这篇文章的结构十分严密,就像神经节和神经网的关系一样。
\end{equotation}
喻体“神经网”是谁也没有看见过的东西,难免让读者觉得生涩难懂,违反了“通俗易懂”这
一原则。

\begin{equotation}
  景区灯光璀璨,数万游人像蚂蚁一样,挤成一团,游览十里古韵不夜河,尽享灯影美景
  浪漫情。
\end{equotation}
把欢乐的人们比作“一团蚂蚁”,本体“欢乐的游人”和喻体“蚂蚁”没有相似性,且含有贬义,
感情色彩也不当。

\subsubsection{“像”字句不一定是比喻句}

“像”含义丰富,“像”字句不一定是比喻句,具备了“本体”,“比喻词”和“喻体”结构的“像”
字句才是比喻。

以下“像”字句都不是比喻。
\begin{equotation}
  深秋的太阳没遮拦地照在身上,煦暖得像阳春三月。(比较)

  这天黑沉沉的,好像要下雨了。(猜测)

  每当看到这条红领巾,我就像置身于天真烂漫的少年时代。(想象)

  阿Q哥,像我们这样的穷朋友是不要紧的。(举例)

  为此目的,就要像马克思所说的详细地占有材料,加以科学的分析和综合的研究。(指
  示)
\end{equotation}

\subsection{排比错用举例}

\subsubsection{排比使用要求}

\begin{enumerate}
  \item 由三个或三个以上短语或句子。

  \item 结构相同或相似。

  \item 内容相关、意义相近、语气一致。
\end{enumerate}

\subsubsection{排比错用举例}

\begin{equotation}
  平易的话语,幽默的口吻,宣讲内容接地气,宣讲员精彩的宣讲使得收看直播的群众既
  听得进又记得牢。
\end{equotation}

“平易的话语,幽默的口吻”是偏正结构,“宣讲内容接地气”是动宾结构,三句结构不相同,
不能构成排比,可改成:

\begin{equotation}
  平易的话语,幽默的口吻,接地气的宣讲内容,宣讲员精彩的宣讲使得收看直播的群众
  既听得进又记得牢。
\end{equotation}

\begin{equotation}
  个人的自学,个人的努力,个人的独立钻研,是主要的;但是适当的讨论,相互的讨论,
  集体的讨论也是必要的。
\end{equotation}

“不合逻辑”病句。自学已经包含了个人努力,个人努力包含了独立钻研,三者不能并列。
“适当”是讲程度,“相互”和“集体”说的是讨论的形式,三者也不能并列。

\subsection{夸张错用举例}

夸张是为达到某种表达效果的需要,故意对事物的形象、特征、作用、程度等方面加以扩
大、缩小或超前描述的一种修辞手法。夸张可分为三类:扩大夸张(如“蜀道之难,难于上
青天”),缩小夸张(如“心眼小得像针鼻儿”),超前夸张(如“李医生给人看病,药方没
开,病就好了三分”)。

\subsubsection{运用夸张的注意点}

尊重事实,合理合度夸大。既不能夸张过度,违背事实,也不能和事实距离过近,否则会
分不清是在说事实还是在夸张。就夸张的“合理合度夸大”,鲁迅先生有过精辟的论述:“,
‘燕山雪花大如席’是夸张,但燕山究竟有雪花,就含着一点诚实在里面,使我们立刻知道
燕山原来有这么冷。如果说‘广州雪花大如席’,那就变成笑话了。”

为了和文体特征相符,在科技说明文等文体中可少用甚至不用夸张,以免歪曲事实。

\subsubsection{夸张错用举例}

\begin{equotation}
  玉米稻子密又浓,铺天盖地不透风,就是卫星掉下来,也要弹回半空中。
\end{equotation}
过分夸大了种植物的密度,脱离了生活的基础和根据,这是浮夸。

\begin{equotation}
  蚂蚁身躯虽小,但力量很大,一只蚂蚁可搬动一粒米,一群蚂蚁不就可以翻江倒海了吗?
\end{equotation}
用“翻江倒海”形容蚂蚁的力量,夸张过度。

\begin{equotation}
  他跑得像自行车一样快。
\end{equotation}
人在快跑时可以达到自行车的速度,“像自行车一样快”就没有超过人快跑时可能的程度,
夸大的程度较低,夸张不当,可改成“他跑得像脱弦的箭一样快”。

\begin{equotation}
  会议室里静得连一根针落地都听不到。
\end{equotation}
安静应该能听到针落地的声音。此句与事实不符。

\subsection{借代错用举例}

借代错用主要表现为以下两个方面:一是不能准确识别借代,尤其是不能精准区分借代和
借喻,误把借代判为借喻;二是不能识别词语的借代义或含有借代的本义。

\subsubsection{误判借代}

区别借代与借喻的有效方法之一是借代不能改成明喻。另外,借喻是比喻,喻体一定会出
现,而且与喻体相关的词语要契合比喻修辞。例如:

\begin{equotation}
  风吹墙头草,左右两边倒,摇过来不是本心,摇过去才是本心。
\end{equotation}
本句是暗喻,只出现了喻体“墙头草”,改成明喻就是“某种类型的人像墙头草……”。还要注
意的是,句子中还出现了“风吹”,“两边倒”,“摇”等描写喻体相关的词语。借代则不需要
出现与“代体”相关意义的描述。

\begin{equotation}
  我要五块钱,他嫌贵。你嫌贵,我还嫌你胖呢。胖的像条大白熊,别压坏我的驴。讲来
  讲去,大白熊答应我的价钱,骑着驴逛了半天,欢欢喜喜照数付了脚钱。(杨朔《雪浪
  花》)
\end{equotation}
“胖的像条大白熊”显然是比喻中的明喻,下文有“别压坏我的驴”对喻体“大白熊”特征进行
描述。“大白熊答应我的价钱”是借代,用人的特征指代人。为什么不是比喻呢?因为“大白
熊答应我的价钱”一句后没有与“大白熊”特征相似性的词语描述,如果改成明喻“他像大白
熊一样答应我”显然不合逻辑。

\subsubsection{代体不当}

\begin{equotation}
  三十多颗心,就在这块新耕耘的土地上踏出了一条新路。
\end{equotation}
用“心”代替“人”不仅缺乏明确性和代表性,且和后文“踏出了一条新路”搭配不当。

\subsection{比拟错用举例}

使用比拟时,首先要注意本体的特点和拟体的特点要适切,符合事理逻辑;其次是比拟修
辞要能和整个语段的语境、语意、词句的感情色彩相符。

\subsubsection{拟体不当,不合逻辑}

例如:
\begin{equotation}
  蒲公英柔软的茎上顶着小黄伞,雄赳赳地守卫在道路两旁。
\end{equotation}
本句使用拟人的修辞方式,但前半句说蒲公英“柔软”,后半句将其拟作“雄赳赳的卫士”不
当。

\begin{equotation}
  面对风暴的欺凌,松柏挺胸不屈,杨柳弯腰逢迎,江河寂然无语,高山昂然抗争。
\end{equotation}
“风暴”下的“江河”应是波浪起伏,而非“寂然无语”。

\subsubsection{比拟句语境失恰,不合逻辑}

例如:

\begin{equotation}
  秋雨跳着欢快的舞,一下就是几天,什么活也干不了,真闷死人。
\end{equotation}
把“秋雨”拟作“人”,借赞扬秋雨的欢快,表达人的欢快之情,这与句末对秋雨的埋怨矛盾。

\begin{equotation}
  晚上,我们坐在院子里乘凉,听爷爷讲《聊斋》故事。他讲到紧张的时候,大家都屏住
  气;讲到轻松的时候,大家都拍手大笑。这时连月亮也探头探脑,冷眼瞧着我们。
\end{equotation}
描述爷爷讲故事,氛围欢快,拟人句“月亮也探头探脑,冷眼瞧着我们”渲染了清冷诡异的
气氛,前褒后贬,句意矛盾。

\subsection{对偶错用举例}

\subsubsection{运用对偶的注意点}

\begin{enumerate}
  \item 要合乎格式要求。用字数相等、结构相同、意义对称的一对短语或句子来表达两
    个相对或相近的意思。

  \item 要合乎表达规范,没有语病。

  \item 要合乎文体风格和整体文本的语言风格。
\end{enumerate}

\subsubsection{对偶错用举例}

\begin{equotation}
  当下一些人总感觉陷入了精神内耗,对自己信心全无,患得患失;对社会满腹牢骚,怨
  天怨地。
\end{equotation}
“信心全无”是主谓结构,“满腹牢骚”是偏正结构,“对自己信心全无,患得患失”和“对社会
满腹牢骚,怨天怨地”,结构不同,不能构成对偶。

\begin{equotation}
  公园里,人群喧闹,小孩嬉笑;百花争艳,桃李争娇,一派春意盎然的景象。
\end{equotation}
“人群”包含“小孩”,“百花”包含“桃李”,意义有包容关系,句子前后形成不了对偶。

\begin{equotation}
  他脚蹬起跑器,两手撑地,收腹、弓背,凝神谛听“鸣枪”,前方跑道上,他的身影里,
  一只蚂蚁在横穿跑道,慢腾腾气定神闲貌似旅游者在欣赏自然风景,晃悠悠摇头晃脑好
  像哲学家在思考人生难题。
\end{equotation}
整个文段多用短句描写“他”起跑前的动作,强调起跑前的紧张,而末句的对偶用长句子描
写蚂蚁横穿跑道的情形,与全段语言风格不相符。

\subsection{设问错用举例}

\subsubsection{运用设问的注意点}

精准判断“不答式”设问,我们通常认为设问修辞要自问自答。“不答式”设问没有回答,对
这类设问要能够准确判断。例如:

\begin{equotation}
  问苍茫大地,谁主沉浮?(毛泽东《沁园春·长沙》)
\end{equotation}
上例是不是设问句?

首先可以排除使用反问修辞,因为它不是“反着问”的。它也不是一般疑问句,因为答案就
在其中。其实它属于“不答式”设问修辞,因为答案已在其中,不需要回答,意在引起读者
思考。

使用“几问一答式”设问时,要注意几个问句的排列顺序,问句间不能存在“不合逻辑”的语
病。

\subsubsection{设问错用举例}

\begin{equotation}
  每年的4月23日是世界阅读日,这一天总会让人思考阅读的意义。阅读是为了滋养精神诗
  意的栖居吗?是为了在考试中考出高分吗?是为了自我充电终身学习吗?还是为了炫耀
  消遣?诺贝尔文学奖获得者德国作家黑塞在《获得教养的途径》一文中回答了这个问题:
  阅读经典“要帮助我们将自己的人生变得越来越充实、高尚,越来越有意义”。
\end{equotation}
“阅读是为了滋养精神诗意的栖居吗?是为了在考试中考出高分吗?是为了自我充电终身学
习吗?还是为了炫耀消遣?”四个句子间在意义上排列无序。

\subsection{反问错用举例}

\subsubsection{使用反问的注意点}

反问用疑问的形式表达肯定的意思,使用时要避免否定失当,准确表情达意。

反问多用来加强语气,强化表达者的态度,使用时要避免语言暴力,力争表达得体。

\subsubsection{反问错用举例}

\begin{equotation}
  雷锋精神当然要赋予它新的内涵,但谁又能否认现在就不需要学习雷锋了呢?
\end{equotation}
“谁能否认……呢”,意思是“谁都得承认……”,这与要表达的意思相反。末句可以改成“但谁又
能否认现在需要学习雷锋呢?”

\begin{equotation}
  每个人都应该追求独立的生命价值,难道你能否认你不应该追求独立的生活吗?
\end{equotation}
“反问”加“双重否定”等于“肯定”。上例改成陈述句是“您要承认你不应该追求独立的生活”,
这就与句意相反。末句可以改成“难道你能否认你应该追求独立的生活吗?”

\begin{equotation}
  谁也不会否认,鲁迅先生不是中国新文学的奠基人。
\end{equotation}
应把后半句中的“不”去掉。

\begin{equotation}
  难道谁能否认你脑子转得慢呢?(同学之间)

  谁能说你这不是偷懒的行为呢?(上下级之间)

  难道马路是你家修的吗?想怎么开就怎么开!(汽车司机之间)

  你的服务态度难道不能好一点吗?(顾客商家之间)

  你怎么能和成绩好的牛娃比呢?(家长孩子之间)
\end{equotation}
这些反问句都有一些责备、傲慢的语气,往往能激化矛盾,不利于事情的沟通和解决。应
该把这些反问句换成平和得体的表达方式。

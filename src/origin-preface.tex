\chapter[原前言]{《语法修辞逻辑精讲》前言}

二十多年前,我和几位同事合编了《古代文化知识精讲》《文学常识精讲》《古代名句选
讲》三本小册子,由于讲的都是常识,坚持面向大众,因而受到了普遍欢迎,一度成了畅
销书。今年,我又和同事合作,推出三本新的小册子。这是三本什么书,又为什么要编呢?

先看学校里。不少中学生“一怕文言文,二怕写作文”。这种状况应该改变,这需要从学好
有关常识做起。

再说社会上。眼下的信息时代,人们日益追求快速、高效,不愿看稍长一点的文章,不肯
稍作一点冷静的思考,以致经常出现用词不当、表意不明、言不及义、花里胡哨的表达。
其中一个重要原因是忘记了常识。难怪几年前就有人说,当代人不缺知识,缺的是常识。

为了帮助中学生和社会上的普通读者初步具备读懂浅近文言文的能力和基本的写作能力,
我们编写了《文言文常识精讲》《写作常识精讲》。而要有基本的写作能力,首先必须解
决好正确表达的问题。语法能使表达通顺,修辞能使表达生动,逻辑能使表达严密,为此
我们编了《语法修辞逻辑精讲》,以有助于表达的规范。

因为这是一套面向大众的普及性读物,所以我们在编写时坚持两条:一是通俗性,不玩概
念,让读者一看就懂;二是资料性,少说大道理,多附对读者有用的资料,如容易用错的
成语100例,常见的语、修、逻错误,常见的文言虚词20例,应试作文临考应注意的问题,
增强语言吸引力的方法,等等。

说到常识,我想起了从报上看到的一则短文。有人问一位事业成功者有什么经验,答曰:
“没什么经验,我只是喜欢按常识去认真做事而已。”这句普通得不能再普通的话告诉我们:
如果肯在学常识、按常识做事上下功夫,把平常的事做到极致,把低级错误降到最少,那
效率、成功就一定会随之而来。当然,随着科技的进步和时代的发展,常识也得不断丰富、
完善,但它对实践的指导功能永远不会过时。

关于常识,我想再说几句。从某种意义上说,强调常识,就是尊重规律,就是符合情理,
就是坚守底线。实践证明:抛弃常识,目标往往落空;尊重常识,事情常常成功。对每个
普通人来说,想多多少少办成一点事,就会深切地感到:常识始终有用,常识伴人一生。

编写这三本小册子,再次体现了我们对“普及常识,面向大众”这一原则的坚守。

我们不敢奢望这三本小册子对中学生和普通读者学好文言、写好文章能有多大作用,只要
能对“学到常识,减少失误”有所助益,我们就很欣慰了。

参加本套丛书编写的都是金陵中学语文教研组的优秀中青年教师。为了编好本套丛书,他
们在繁忙的日常教学之余,广泛搜集资料,细心加以筛选,认真进行编写,付出了辛勤的
劳动。尽管如此,不足仍恐难免,真诚欢迎语文专家和广大读者批评指正。

\begin{flushright}
  喻旭初\\
  2023年8月
\end{flushright}

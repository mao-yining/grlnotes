\chapter{语法}

\subsubsection{语法单位}

语法单位主要有四级:语素、词、短语、句子。它们都是语言中的音义结合体。

\index{语素}
语素是语言中最小的音义结合体。语素可以组合成合成词,有的可单独成词。语素可以分
为单音节(如“水”)、双音节(如“蟋蟀”)和多音节(如“吐鲁番”)三类。

\index{词}
词是最小的能够独立运用的语言单位,是组织短语和句子的备用单位。一部分词加上句调
可以单独成句。

\index{短语}
短语是由词组成的、没有句调的语言单位,是造句的备用单位。大多数短语可以加上句调
成为句子。

\index{句子}
句子是具有一个句调、能够表达一个相对完整的意思的语言单位,句子前后有隔离性停顿。

\section{词法}

\subsection{词的构成}

\subsubsection{【单纯词】}

\index{单纯词}
由一个语素构成的词是单纯词,不存在内部构造问题,但可以根据构词语素的音
节特征来划分类型。根据音节的数量,单纯词可以分为单音节、双音节和多音节三类。

单纯词分类如表~\ref{tab:单纯词分类}:

\begin{table}[htbp]
  \index{传承词}
  \index{拟声词}
  \index{音译词}
  \index{双声词}
  \index{叠韵词}
  \index{叠音词}
  \index{音译词}
  \index{拟声词}
  \centering \caption{单纯词分类} \label{tab:单纯词分类}
  \begin{tabular}{|c|c|l|}
    \hline
    分类 & \multicolumn{2}{c|}{例词} \\ \hline
    \multirow{3}*{单音节单纯词} & 传承词 & 水、吃、讲 \\ \cline{2-3}
                                & 音译词 & 钵、硼、氢 \\ \cline{2-3}
                                & 拟声词 & 嗖、哇、嘶 \\ \hline
    \multirow{6}*{双音节单纯词} & \multirow{3}*{联绵词} & 双声词:伶俐、参差、弥漫 \\ \cline{3-3}
                                &                       & 叠韵词:从容、窈窕、烂漫 \\ \cline{3-3}
                                &                       & 非双声叠韵词:芙蓉、蝙蝠、垃圾 \\ \cline{2-3}
                                & 叠音词  & 姥姥、蝠蝠、蝙蝠 \\ \cline{2-3}
                                & 音译词  & 克隆、咖啡、葡萄 \\ \cline{2-3}
                                & 拟声词  & 哗啦、吧唧、哧溜 \\ \hline
    \multirow{2}*{多音节单纯词} & 音译词 & 马赛克、法西斯、麦克风 \\ \cline{2-3}
                                & 拟声词 & 淅淅沥沥、稀里哗啦、叽里咕噜 \\
                                \hline
  \end{tabular}
\end{table}

\subsubsection{【合成词】}

\index{合成词}
由两个或两个以上语素构成的词。

合成词分类如表~\ref{tab:合成词分类}。

\begin{table}[htbp]
  \centering \caption{合成词分类} \label{tab:合成词分类}
  \begin{tabular}{|c|c|l|}
    \hline
    \multicolumn{2}{|c|}{合成词分类} & \multicolumn{1}{c|}{例词} \\ \hline
    \multirow{9}{6em}{复合词:不同词根组合而成}
                                     & \multirow{4}*{联合型} & 相近关系:道路、教授、泥土 \\ \cline{3-3}
                                     & & 相关关系:口舌、骨肉、笔墨 \\ \cline{3-3}
                                     & & 相反关系:始终、反正、开关 \\ \cline{3-3}
                                     & & 偏义关系:国家、窗户、动静 \\ \cline{2-3}
                                     & \multirow{2}*{偏正型} & 定中关系:圆球、方桌、卧铺 \\ \cline{3-3}
                                     & & 状中关系:笔谈、函授、热爱 \\ \cline{2-3}
                                     & 述宾型                  & 毕业、注意、伤心 \\ \cline{2-3}
                                     & 补充型                  & 削弱、促进、信件 \\ \cline{2-3}
                                     & 主谓型                  & 地震、口红、眼花 \\ \hline
                                     \multirow{3}{6em}{派生词:由词根和词缀组合而成}
                                     & 前缀型 & 老师、老板、老婆 \\ \cline{2-3}
                                     & 后缀型 & 男子、孩子、妻子 \\ \cline{2-3}
                                     & 中缀型 & 土里土气、古里古怪、糊里糊涂 \\ \hline
    \multirow{2}*{重叠词}          & AA型 & 叔叔、星星、常常 \\ \cline{2-3}
                                   & AABB型 & 花花绿绿、断断续续、熙熙攘攘 \\ \hline
  \end{tabular}
\end{table}

\subsection{词的类别}

\subsubsection{【实词】}

\index{实词}
能够单独充当句法成分,意义实在具体,即有词汇意义和语法意义的是实词。实词主要包
括名词、动词、形容词、区别词、数词、量词、副词、代词,以及特殊实词如拟声词、叹
词等。

\subsubsection{【虚词】}

\index{虚词}
不能充当句法成分、只有语法意义的就是虚词。虚词可细分为介词、连词、助词、语气词。

\subsubsection{【短语】}

\index{短语}
短语是由语法上能够搭配的词组合起来的没有句调的语言单位,又叫词组。它是大于词而
又不成句的语法单位。简单短语可以充当复杂短语的句法成分,短语加上句调可以成为句
子。

\paragraph{短语的结构类型}

\begin{description}
  \index{短语!主谓短语}
  \item[主谓短语]由有陈述关系的两个成分组成,前面被陈述部分是主语,表示要说的是
    谁或什么;后面陈述的部分是谓语,说明主语怎么样或是什么。如:
    \begin{equotation}
      全剧终、心理健康、凤凰涅槃
    \end{equotation}

    \index{短语!动宾短语}
  \item[动宾短语]由有支配、涉及关系的两个成分组成,前面起支配作用的部分
    是动语,表示动作行为;后面被动作支配的部分是宾语,表示做什么、是什么。
    如:
    \begin{equotation}
      认识自我、蹭热点、清扫死角
    \end{equotation}

    \index{短语!偏正短语}
  \item[偏正短语]由有修饰关系的两部分组成,修饰部分在前面,叫修饰语,被
    修饰部分在后面,叫中心语。分为定中短语(由定语和名词性中心语组成)和状
    中短语(由状语和动词、形容词性中心语组成)两种。如:
    \begin{equotation}
      (定中短语)世外桃源、青春之歌、清澈的爱

      (状中短语)独立思考、十分幽僻、明天出发
    \end{equotation}

    \index{短语!中补短语}
  \item[中补短语]由有补充关系的两个成分组成,前面被补充部分是中心语,由
    谓词充当;后面补充部分是补语,也由谓词充当,起述说的作用,能回答“怎么样”
    的问题。如:
    \begin{equotation}
      喊得山响、熟透了、痛快极了
    \end{equotation}

    \index{短语!联合短语}
  \item[联合短语]由语法地位平等的两项或几项组成,彼此间是联合关系,可细
    分为并列、递进、选择等关系。如:
    \begin{equotation}
      风花雪月、傲慢与偏见、红玫瑰与白玫瑰
    \end{equotation}

    \index{短语!连谓短语}
  \item[连谓短语]由多项谓词性词语连用,谓词性词语之间没有语音停顿。如:
    \begin{equotation}
      到教室看书、拉住我不放、摸着石头过河
    \end{equotation}

\index{短语!兼语短语}
  \item[兼语短语]由前一动语的宾语兼作后一谓语的主语,即动宾短语的宾语和
    主谓短语的主语套叠,合二为一,形成有宾语兼主语双重身份的一个“兼语”。直
    接包含兼语的短语叫兼语短语。如:
    \begin{equotation}
      选他当代表、请君入瓮、只剩下柳树随风起舞
    \end{equotation}

\index{短语!同位短语}
  \item[同位短语]多由两项组成,前项和后项的词语不同,所指是同一事物。如:
    \begin{equotation}
      我们渔民、小说家巴金、父子二人
    \end{equotation}

\index{短语!方位短语}
  \item[方位短语]由方位词直接附在名词性或谓词性词语后面组成,主要表示处
    所、范围或时间,具有名词性。如:
    \begin{equotation}
      三天前、月光下、开会以前
    \end{equotation}

\index{短语!量词短语}
  \item[量词短语]由数词或指示代词加上量词组成。如:
    \begin{equotation}
      三次、这回、哪件
    \end{equotation}

\index{短语!介词短语}
  \item[介词短语]由介词附着在名词等词语前面组成。如:
    \begin{equotation}
      关于课堂纪律问题、当黎明到来的时候、为大家
    \end{equotation}

\index{短语!助词短语}
  \item[助词短语]由助词附着在词语上组成,包括“的”字短语、比况短语和“所”
    字短语等。如:
    \begin{equotation}
      扑入你视野的、炸雷似的、所见
    \end{equotation}

\end{description}

\paragraph{短语和词的区别}

短语是词和词结合起来构成的,可是词和词联在一起并非都是短语。比如,“铁”是一
个词,“路”也是一个词,“铁路”却不是短语,而是另一个词。通常用“扩展法”区别短
语和词的界限。所谓扩展法,就是把可疑单位拆开,插一个或几个词,造成一个较复
杂的短语形式。经扩展后,说来能成话的,那么这个单位应属于由两个词组成的短语。
如果经扩展不成话的,就不是短语,而是词。例如:“眼红”不能扩展,就是词;“眼睛
红”可以扩展,就是短语。

\subsection{词义的感情色彩}

词义的感情色彩是指附着在词的概念义之上,表达人或语境所赋予的主观情感和态度。

感情色彩主要包括以下三种:

\begin{description}
  \index{褒义词}
  \item[褒义词]在表示意义的时候,带有赞美、喜爱、肯定的感情色彩。如:
    \begin{equotation}
      热爱、坚强、大方、奉献
    \end{equotation}

    \index{贬义词}
  \item[贬义词]在表示意义的时候,带有贬斥、厌恶、否定、轻蔑的感情色彩。如:
    \begin{equotation}
      无耻、怂恿、虚伪、推诿
    \end{equotation}

    \index{中性词}
  \item[中性词]指不带有褒贬感情色彩的词。如:
    \begin{equotation}
      河流、结论、士兵、跳
    \end{equotation}

    中性词根据语言表达的需要,可以用于褒或贬。例如:
    \begin{equotation}
      邹荻帆《乡音》:我看见蝗虫遮天蔽日,向农民争夺粮食。
    \end{equotation}
    “蝗虫”原本是指一种昆虫,中性词,在这里用“蝗虫”比喻那些不劳而获的地主,带上
    了明显的贬义色彩。

\end{description}

\subsection{同义词、反义词}

\subsubsection{同义词}

\index{同义词}
指意义相同、相近的一组词。例如:

坚持、保持、维持、教、诲、训、诫

\begin{enumerate}
  \item 同义词的辨析

    可以从以下方面进行辨析。

    意义方面。例如:“轻视”和“蔑视”,“轻视”语义轻,“蔑视”语义重。

    感情色彩。例如:“成果”,“结果”和“后果”,分别是褒义词、中性词和贬义词。

    用法方面。例如:“突然”和“猛然”,“突然”可以作谓语、定语、状语和宾语,“猛然”
    往往作状语。

  \item 同义词的作用

    同义词的作用主要有:表达准确;表达生动;表达委婉;表达富有变化;加强表达气
    势。例如:

    \begin{equotation}
      孔乙己便涨红了脸,额上的青筋条条绽出,争辩道,“窃书不能算偷……窃书!
      ……读书人的事,能算偷么?”(鲁迅《孔乙己》)
    \end{equotation}
    “窃”和“偷”是同义词,孔乙己的狡辩生动地展示了孔乙己深受科举制度毒害的迂腐形
    象。

    \begin{equotation}
      真的猛士,敢于直面惨淡的人生,敢于正视淋漓的鲜血。(鲁迅《纪念刘和珍君》)
    \end{equotation}
    “直面”和“正视”是同义词,同义对用,使行文富有变化。
\end{enumerate}

\subsubsection{反义词}

\index{反义词}
意义相反或相对的两个词构成反义词。例如:
\begin{equotation}
  我曾远离祖国几年。那些日子,我对祖国真的说不出有多么的怀念。这怀念是痛苦而又
  是幸福的。(黄药眠《祖国山川颂》)
\end{equotation}
“痛苦”和“幸福”是反义词。
\begin{equotation}
  子在川上曰:“逝者如斯夫,不舍昼夜。”(《论语》)
\end{equotation}
“昼”与“夜”是反义词。

\paragraph{反义词的作用:}

\begin{enumerate}
  \item 通过意思的鲜明对照,深刻揭示事物特点。

  \item 多组反义词连用,加强语气,使语言深刻有力。例如:
    \begin{equotation}
      有的人活着/他已经死了;有的人死了/他还活着。(臧克家《有的人》)
    \end{equotation}
    使用“活”和“死”、“死”和“活”两组反义词,把人生的伟大与丑恶论述得淋漓尽致、入
    木三分。
\end{enumerate}

\subsubsection{常见近义词辨析100例}

\begin{enumerate}
  \item 诞生·诞辰

    诞生,意为“出生”,作动词,是中性词,对任何人都可使用;\\
    诞辰,意为“生日”,作名词,是褒义词,多用于值得尊敬的人,除“诞辰”外,还有“华
    诞”,“寿诞”,都用于值得尊敬的人。

  \item 配置·安置

    配置,配备布置;安置,使人和事物有着落。

  \item 申辩·申诉

    申辩,对受人指责的事申述理由,加以解释;\\
    申诉,诉讼当事人或其他公民对已发生效力的判决不服时,向有关部门提出重新处理
    的要求。

  \item 脆弱·软弱

    脆弱,禁不起挫折;软弱,不坚持。

  \item 委屈·委曲

    委屈,受到不应该有的指责或待遇,心里难过;\\
    委曲,事情的底细和原委。

  \item 抚养·扶养

    抚养,指爱护并养育,用于长辈对晚辈的教养;\\
    扶养,指养活,用于法律关系,可用于平辈。

  \item 消失·消逝·消释

    消失,表示事物从存在到不存在,强调过程;\\
    消逝,含有一个事物随时间的过去而不复存在的意思,强调结果;\\
    消释,指疑虑、嫌怨、痛苦等消失解除。

  \item 氛围·气氛

    氛围,周围的气氛和情调;\\
    气氛,周围环境中的一种精神表现或景象,不包括情调。

  \item 反应·反映

    反应,指事情所引起的意见、态度或行动;\\
    反映,指向上汇报或通过文艺作品表现客观事物的实质。

  \item 刻画·刻写

    刻画,雕刻采画器物;用文字描写或用其他艺术手段表现人物形象。

    刻写,把蜡纸铺在誊写钢板上用铁笔书写。

  \item 退化·蜕化

    退化,生物体在进化过程中某一部分器官变小,构造简化,功能减退甚至消失。如鲸、
    海豚等的四肢成鳍状。泛指事物由优变劣,由好变坏。

    蜕化,虫类脱皮,比喻腐化堕落。

  \item 间隙·间歇

    间隙,指空隙,空闲的时间,事物间的空间或时间距离;\\
    间歇,指停止、中止,时断时续,两段时间之间的间隔或运动、变化等隔一段时间就
    停一会儿。

  \item 侵害·侵犯

    侵害,指侵入而损害;侵犯,是非法干涉别人,损害其权利。

  \item 变迁·变革

    变迁,指情况或阶段的变化转移,多用于人事、时代;\\
    变革,指改变事物本质,多用于社会制度。

  \item 体验·体会

    体验,指在生活实践中的亲身经历、感受,偏重于感情;\\
    体会,透过现象对事物的精神实质的领悟,偏重于理性。

  \item 品位·口味

    品位,是指文艺作品所达到的水平,作名词;\\
    品味,指仔细体会,常作动词,有时也指物品的品质风味。

  \item 工夫·功夫

    工夫,有三层意思:
    \begin{enumerate}
      \item 表示占用的时间,如“一会儿工夫就完成”,“用了两年工夫写成一本书”;
      \item 表示空闲时间,如“我现在没工夫”,“双休日大家都有工夫外出旅游了”;
      \item 表示时候,如“刚解放那工夫,我还是个孩子”。
    \end{enumerate}

    功夫,主要指人的本领怎样、造诣如何。如“他的表演真有功夫”,“演员都得练功夫”。

    这两个词的区别在于“工夫”表示时间,“功夫”不表示时间,指人的本领。

  \item 交代·交待

    交代,主要有三层意思:
    \begin{enumerate}
      \item 移交、接替,如“交代工作”;
      \item 嘱咐、吩咐, 如“领导一再交代我们要按政策办事”;
      \item \label{item:explain}把事情或者意见向有关的人说明,把错误或罪行坦白
        出来,如“交代问题、交代政策”。
    \end{enumerate}

    交待,主要有两层意思:
    \begin{enumerate}
      \item 同~\ref{item:explain},不过习惯用“交代”;
      \item 完结(指结局不如意的,含诙谐意),如“要是飞机出了事,这条命也就交待
        了”。
    \end{enumerate}

  \item 做客·作客

    做客,指访问别人,自己当客人;\\
    作客,是指寄居在别处,是常用的书面语。

  \item 成见·意见

    成见,着重指先入为主、不愿改变的固定看法,多用于人,适用范围较小,含贬义;\\
    意见,泛指一定的看法,包括正确的、错误的、不满的,用于人也用于事物,适用范
    围较广,是中性词。

  \item 袒护·偏向

    两词都有“(对某一方)无原则支持或者袒护”的意思,就此含义而言都是贬义词,表
    示出于私心不公正地向着某一方。根据日常生活中这两个词的使用情况看,这两个词
    的细微差别在于:袒护,适用于书面语体,语意较重;偏向,适用于口头语体,语意
    较轻。

  \item 增殖·增值

    增殖,是增生和繁殖的意思,繁殖是指生物产生新的个体的意思,如“增殖耕牛”,“提
    高牲畜增殖力”等;\\
    增值,是价值增加的意思,如“小麦变成面粉,价值就增高了,实现了增值”。“增值税”
    是指增加数值应缴纳的税。这里的“值”是指数值,与生物繁殖不是一回事,不能用“增
    殖税”。

  \item 侦查·侦察

    关键在于把“查”与“察”的区别搞清楚。查,包括检查和调查的意思,如“检查”,“审
    查”,“查究”,“查阅”等;\\
    察,包含仔细观察和调查的意思,如“观察”,“考察”,“察访”,“察觉”等。“查访”和
    “察访”,“查看”和“察看”有时可以通用,但表示的意思有细微差别。查访、查看的
    “查”,主要指检查、察访;察看的“察”是指观察、仔细看的意思。

    侦查,公安机关、国家安全机关和监察机关在刑事案件中,为了确定犯罪事实和证实
    犯罪嫌疑人、被告人确实有罪而进行调查及采取有关的强制措施,如侦查案情。

    侦察,指为了弄清敌情、地形及其他有关作战的情况而进行活动,如“侦察飞行”。

    侦查,是司法用语,主要指调查和检查;侦察,是军事用语,主要指观察和察看,在
    词义和用法上都有区别。

  \item 度过·渡过

    度过,主要是时间上的经过,如“这个五一假期在海边度过”;渡过主要指的是由此岸
    到彼岸,是空间的经过,首先是涉水的空间转移,如“渡过重洋”,“渡过黄河”。另外,
    “渡过”也用于比喻义的通过,如“渡过艰难险阻”,“渡过难关”。

  \item 权力·权利

    权力,指政治上的强制力量,也指职责范围内的支配力量;\\
    权利,词义侧重于利,多用来指公民或法人依法行使的权力和享受的利益(跟义务相
    对)。

  \item 抑制·克制

    抑制,主要有两个含义。

    \begin{enumerate}
      \item 大脑皮层的基本神经活动之一,是在外部或者内部刺激下产生的,作用是阻止
        皮质的兴奋,减弱器官的活动,如“睡眠就是大脑皮质全部处于抑制的现象”;
      \item 压下去,控制,如“他抑制不住内心的喜悦”。
    \end{enumerate}

    克制,指克服、制服,多指抑制情感,如“他很能克制自己的情感,冷静地处理问题”。

  \item 筹办·筹措

    筹办,指筹划办理,对象是事情;\\
    筹措,指设法弄到,对象是财物等东西,如款子、粮食等。

  \item 披阅·批阅

    披阅,指披览,阅读;\\
    批阅,指阅读并加以批示或批改。

  \item 考察·考查

    考察,指实地观察了解、调查研究,也指细致深刻的观察;\\
    考查和检查差不多,强调用一定的标准来衡量(行动、行为)。

  \item 开辟·开拓

    开辟,指打开通路,创立(从无到有);\\
    开拓,指开辟、扩展(从小到大)。

  \item 宽慰·安慰

    宽慰,指宽解、安慰;\\
    安慰,形容心情安适(或用作使动)。

  \item 困苦·痛苦

    困苦,指生活上艰难痛苦;\\
    痛苦指身体或精神感到非常难受。

  \item 机体·肌体

    机体,是生命个体的总称,如加速机体的新陈代谢;\\
    肌体,指身体,常用来比喻组织机构。

  \item 既而·继而

    既而,时间副词,着重指前后两件事发生的时间相隔不久,一般单用;\\
    继而,关联副词,前后两事紧紧相连,常与“始而”,“先是”搭配。

  \item 校正·矫正·教正

    校正,校对更正文字、位置上的偏差和错误;\\
    矫正,纠正生理毛病和错误偏差;\\
    教正,客套话,让人指教。

  \item 误解·曲解

    误解,错误地理解。

    曲解,错误解释客观事实或别人的原意(多指故意的),如“你这样分析课文,实际上
    是曲解了作者的创作意图”。

  \item 辣手·棘手

    辣手,指手段厉害或毒辣;\\
    棘手,指形容事情难办。

  \item 界限·界线

    界限,指不同性质事物的分界、限度、尽头等,意思笼统;\\
    界线,指分界的线(具体的)。

  \item 急躁·暴躁

    急躁,侧重于“急”,有两种意思:

    \begin{enumerate}
      \item 碰到不称心的事情马上激动不安;
      \item 想马上达到目的,不做好准备就行动。
    \end{enumerate}

    暴躁,侧重于“暴”,指遇事好发急,不能控制情绪。

  \item 经历·阅历

    经历,指亲身见过、做过或遭遇过(的事)。如“他一生经历过两次世界大战”,“生活
    经历”。

    阅历,指:
    \begin{enumerate}
      \item 亲身见过、听过或做过;
      \item 由经历得来的知识。常用作名词,如“他阅历很浅”。
    \end{enumerate}

  \item 鉴赏·欣赏

    鉴赏,指鉴定和欣赏(艺术品、文物);\\
    欣赏,指享受美好的事物,领略其中的趣味。

  \item 局面·场面

    局面,一个事情内事情的状态,范围大,较抽象,如“生动活泼的政治局面”;

    场面,一定场合下的情境,范围小,较具体,如“场面壮观”。

  \item 尽管·无论

    尽管,作连词时,表姑且承认某种事实,下文往往有“但是”,“然而”等表转折的连词
    呼应;\\
    无论,连词,表条件不同而结果不变。

  \item 节余·结余

    侧重点不同。节余,是因节约而剩下;\\
    结余,是结算后剩下。

  \item 简洁·简截

    简洁,(说话、行文等)简明扼要,没有多余的内容;\\
    简截,同“简捷”,直截了当。

  \item 究竟·毕竟

    都含有“到底”的意思。究竟,表示追根到底,用于疑问句,语气不肯定;可兼作名词,
    表示原因和结果。毕竟,表示追根到底所得的结论,有加强语气的作用。

    在表示肯定语气时,毕竟和究竟可以通用。同二者词义相近的词有“到底”,“终归”,“终
    究” 。

  \item 精密·周密

    精密,侧重于“精”,意为精确细密,多指研究或制作的精确程度;\\
    周密,侧重于“周”,形容做事周到、全面、细密。

  \item 截止·截至

    截止,动词,不带宾语但可带补语,表示(到一定期限)停止;\\
    截至,一般作介词,与宾语组成介词短语作状语,通常用于尚未结束的过程,截止到。

  \item 艰苦·坚苦

    艰苦,艰难困苦,适用于环境、生活、岁月等客观条件;\\
    坚苦,坚毅刻苦,适用于主观精神、工作作风等。

  \item 艰辛·艰难

    艰辛,强调办事的艰难而辛苦;艰难,强调事物或行为的困难。

  \item 坚忍·坚韧

    坚忍,坚持而不动摇;坚韧,顽强而有韧性。

  \item 痕迹·踪迹

    痕迹:物体留下的印记;残存的迹象。

    踪迹,指行动所留下的痕迹,重在行动后留下的。

  \item 合计·核计

    合计:盘算,商量;合在一起计算。

    核计,指核算(成本)。

  \item 焕发·激发

    焕发:光彩四射;振作。

    激发,指刺激使兴奋。

  \item 豁然·霍然

    豁然,形容开阔或通达。

    霍然:副词,突然;形容词,疾病迅速消除(书面语)。

  \item 轰然·哄然

    轰然,大声;哄然,许多人同时发出声音。

  \item 宏大·洪大

    宏大,侧重于规模大,常用于建筑物、队伍、场面、理想;\\
    洪大,声音大而响亮。

  \item 化装·化妆

    化装有两个意思:
    \begin{enumerate}
      \item 假扮;
      \item 指演员为了适合所扮演的角色形象而修饰容貌。
    \end{enumerate}

    化妆的含义:

    \begin{enumerate}
      \item 特指艺术范畴,适用对象是指特定的表演者,有通过修饰、打扮而改变原来
        面貌的意思(该词义与“化装”通用);
      \item 指生活化妆,有用脂粉等妆饰品修饰容颜,使容貌美丽的意思。
    \end{enumerate}

    化装侧重于装扮;化妆侧重于打扮。

  \item 淡泊·淡薄

    淡泊,不追求名利(书面语);\\
    淡薄,(云雾等)密度小,(味道)不浓,(感情、兴趣等)不浓厚,(印象)因淡
    忘而模糊。

  \item 大义·大意

    大义,大道理,如“微言大义”;\\
    大意,“主要的”或“大概的”意思。

  \item 惦记·思念

    惦记,指(对人或事)心里老想着,放心不下,多用于口语;\\
    思念,指对景仰的人、离别的人或环境不能忘怀,希望见到,多用于书面语。

  \item 电讯·电信

    电讯,指用电话、电报或无线电设备等传送的信息;\\
    电信,指用电话、电报或无线电设备等传送信息的通信方式,旧称电讯。

  \item 典雅·高雅

    典雅,指优美而不粗俗;\\
    高雅,指高尚而不粗俗。

  \item 对于·关于

    都是介词。

    对于,引进对象或事物的关系者。

    关于:
    \begin{enumerate}
      \item 引进某种行为的关系者,组成介宾作状语;
      \item 引进某种事物的关系者,组成介宾作定语,后面要加“的”。
    \end{enumerate}

    注意:表关涉,用“关于”不用“对于”;指出对象,用“对于”不用“关于”;兼有两种情
    况时可以互用;“关于”有提示性质的意义,用“关于”组成的介宾,可以单独作标题,
    用“对于”组成的介宾,只有跟名词组成偏正短语才能作标题,如“对于政策的认识”。

  \item 凋敝·凋零

    凋敝,生活事业衰败,如“民生凋敝”。

    凋零:
    \begin{enumerate}
      \item 草木凋谢零落,如“秋风扫过,万木凋零”;
      \item 衰落,如“家道凋零”。
    \end{enumerate}

  \item 恩惠·恩赐

    恩惠,名词,给予或受到的好处;\\
    恩赐,动词,泛指因怜悯而施舍。

  \item 发奋·发愤

    奋,指鸟振翅飞翔,后来引申为振作、鼓动;\\
    愤,指因为不满意而感情激动。

    发奋,指振作起来,如“发奋努力”,“发奋有为”等;\\
    发愤,指决心努力,如“发愤忘食”,“发愤图强”等。

    发奋,强调精神振作;发愤,突出精神受到刺激而产生向上的内动力。

    发奋,使用的范围要比“发愤”大,可以指个人,也可以指群体或国家;“发愤”一般指个
    人。

    使用上,发奋可以说“奋发”,而发愤则不能说“愤发”。

  \item 法制·法治

    法制,指有关的法律制度;法治,表示根据法律来治理国家。

  \item 凡响·反响

    凡响,指平凡的音乐;反响,指事物所引起的回响,反应。

  \item 妨害·妨碍

    妨害,指有害于事物发展(程度重),使受损害,如“妨害健康”,“妨害要表达的义理”;\\
    妨碍,指使事物不能顺利进行,如“妨碍交通”,“妨碍政策的实施”。

  \item 抚育·哺育·抚恤

    抚育,指照料、教育儿童或照管动植物;\\
    哺育,指喂养,比喻培育;抚恤,指(国家或组织)对因公受伤、牺牲或残废人员的
    家属进行安慰并给以物质帮助。

  \item 肤浅·浮浅

    肤浅,(学识)浅,理解不深;浮浅,(思想作风、文章风格)浅薄、不切实。

  \item 伏法·服法

    伏法,依法处以死刑;服法,认罪。

  \item 伏帖·服帖

    伏帖,心里舒服、顺从。服帖,顺从、妥当,常用 AABB 式重叠。

    表示顺从、驯服时,服帖、伏帖通用。但表示舒坦时用“伏帖”,表示妥当时用“服帖”,
    如“把事情办服帖”。

  \item 赋予·付与

    赋予,指(上对下)交给,是特殊用法;付与,指拿出、交给,是一般用法。

  \item 富裕·富余

    富裕,指财物充足;富余,指足够而有剩余。

  \item 分辨·分辩

    分辨,指把两个以上的人或事物区分开,有分析辨别的意思;\\
    分辩,指为消除所受的指责而进行解释、说明,与“辩解”意思相同。

  \item 风气·风俗·风尚

    风气,指社会上或某个集体流行的爱好和习惯;\\
    风俗,指社会上长期形成的风尚、礼节、习惯等的总和,范围较大;\\
    风尚,指在一定时期中社会上流行的风气和习惯。

  \item 废除·废黜·解除

    废除,指取消、废止(法令、制度、条约等);\\
    废黜,指罢免、革除(官职),现多指废除特权;\\
    解除,指去掉、破除(警报、顾虑、武装、职务等)。

  \item 沟通·勾通

    两个词都有相通连的意思,但“勾通”为贬义词,指暗中串通、勾结。

  \item 公然·公开

    公然,指公开的,毫无顾忌的(贬义);\\
    公开,指(与秘密相对)不加隐蔽的。

  \item 灌注·贯注

    灌注,指用液体浇灌;贯注,指精力集中,有“贯穿下去”的意思。

  \item 光临·惠顾

    光临,是敬辞,称宾客的来到;惠顾,惠临,多用于商店对顾客。

  \item 贯穿·贯串

    贯穿,指穿过、通过(较具体的事物);\\
    贯串,指从头到尾穿过一个或一系列事物。

  \item 国事·国是

    国事,指国家大事;\\
    国是,指国家大计,国家的大政方针;用“国是”的地方一般也可用“国事”,但反之却
    不一定,比如较具体的事务,就不能用“国是”。

  \item 给予·给以

    给予,书面语,也作“给与”;\\
    给以,所带宾语只说所给的事物,不说接受的人,并且多为抽象事物,如“应当给以帮
    助”。

  \item 安详·慈祥·祥和

    安详,指神态平静、从容稳重;\\
    “祥”指吉利,如“祥云”,“发祥”。慈祥,形容老年人的态度神色和蔼安详。祥和,指
    气氛而言。

  \item 把戏·伎俩

    把戏,指花招,蒙蔽人的手法;\\
    伎俩,指不正当的手段(贬义)。

  \item 包含·包涵

    包含,包容、含有;包涵,原谅、宽恕。

  \item 暴发·爆发

    都是动词,都含有“突然发作”的意思。暴发,强调突然性;爆发,强调爆炸性。爆发
    的使用范围比暴发宽。

    暴发:

    \begin{enumerate}
      \item 指突然发财或得势,多含有贬义;
      \item 指突然发作,多用于山洪、大水或疾病等具
        体事物。
    \end{enumerate}

    爆发:
    \begin{enumerate}
      \item 指因爆炸而迅猛发生,多用于具体事物,如“火山爆发”;
      \item 指像爆炸那样突然地发生,多用于抽象事物,如革命、起义、运动等重大事
        变,再如力量、情绪等。
    \end{enumerate}

  \item 本义·本意

    本义,词的本来意义,与引申义、比喻义相对;
    本意,指心里本来的想法、目的。

  \item 不至(不至于)·不致

    不至(不至于),不会达到某种程度,如“决不至于不知道”;\\
    不致,不会引发某种后果,如“决不致犯错误”。

  \item 不止·不只

    不止,副词,不停止或超出某个数目或范围,句中一般带有表数量的词;\\
    不只,用于表递进关系的关联词,常同“还有、甚至”等连用,表示递进关系。

  \item 抱怨·报怨

    抱怨,心中不满,数说别人不对;报怨,对怨恨的人做出反应。

  \item 甄别·鉴别

    甄别,审查鉴定(优劣、真伪),考核鉴定(能力、品质等),如“近几年我国出土了
    大量先秦时期的典籍,使我们有可能对过去被判为伪书的作品重新加以甄别”。

    鉴别,一般用来指判别事物的好坏,如“在选择读书时,我们首先要鉴别书的好坏”。

  \item 草率·轻率

    草率,指(做事)不认真,敷衍行事;\\
    轻率,(说话做事)随随便便,不经过慎重考虑。

  \item 苍茫·苍莽

    苍茫,多指夜色、水域、大地等旷远、迷茫,引申为模糊不清;\\
    苍莽,多指树林、山岭、大地等广阔无边,引申为意境心胸开阔。

  \item 长年·常年

    长年,一年到头,整年,如“长年在野外工作”。

    常年:

    \begin{enumerate}
      \item 终年、长期,如“常年坚持体育活动”;
      \item 平常年份,如“常年产量不过200斤”。
    \end{enumerate}

  \item 呈现·浮现

    呈现,露出的事物较清楚,持续的时间长,多是直接看到的(不是想象的)多在事物
    本身,有时在人的眼前。对象多是现实的事物,如颜色、景色、神情、气氛等。

    浮现,往往是影影绰绰的,持续的时间较短,多是想象的,有时是直接看到的,多在
    脑中、眼前、脸上等。对象多是人的形象、印象、往事、表情等。

  \item 嗤笑·耻笑

    都含有“取笑”的意思,但是程度不同。耻笑的程度重,嗤笑的程度轻。

    嗤笑,以为可笑而讥讽嘲笑,如“他是因为有心理障碍才口吃的,不要嗤笑他”。

    耻笑,以为可耻而鄙视嘲笑,如“日本政府至今也不肯面对过去的侵略历史,这种欺世
    行为怎么能不遭到世人的耻笑!”

\end{enumerate}

\subsubsection{常见反义词100例}

\begin{multicols}{4}
  \centering

  \textbf{A}\\
  安心---担心\\
  安慰---责备\\
  按照---违背\\
  昂首---俯首\\
  昂扬---低落\\

  \textbf{B}\\
  包围---突围\\
  抱怨---体谅\\
  必然---偶然\\
  表面---本质\\
  别致---普通\\

  \textbf{C}\\
  灿烂---暗淡\\
  仓促---从容\\
  常常---偶尔\\
  陈腐---新奇\\
  承担---推脱\\
  充实---空虚\\
  抽象---具体\\
  踌躇---果断\\
  纯粹---混杂\\
  粗鲁---斯文\\

  \textbf{D}\\
  大概---确切\\
  呆板---灵活\\
  得意---失意\\
  低沉---高亢\\
  典雅---粗俗\\

  \textbf{F}\\
  发奋---气馁\\
  乏味---有趣\\
  放肆---收敛\\
  非凡---平庸\\
  肤浅---深刻\\

  \textbf{G}\\
  尴尬---自然\\
  高亢---低沉\\
  固执---变通\\
  故意---无意\\
  过度---适度\\

  \textbf{H}\\
  含蓄---直率\\
  浩瀚---渺小\\
  宏观---微观\\
  滑稽---庄重\\
  荒谬---合理\\

  \textbf{J}\\
  简洁---烦冗\\
  解放---束缚\\
  介意---释怀\\
  经常---偶尔\\
  绝对---相对\\

  \textbf{K}\\
  开明---顽固\\
  肯定---否定\\
  宽厚---刻薄\\
  宽裕---拮据\\
  扩充---缩减\\

  \textbf{L}\\
  冷静---冲动\\
  临时---正式\\
  流传---失传\\
  笼统---具体\\
  沦陷---收复\\

  \textbf{M}\\
  朦胧---清晰\\
  梦幻---现实\\
  腼腆---大方\\
  敏捷---迟钝\\
  明显---隐晦\\

  \textbf{N}\\
  内疚---心安\\
  凝固---溶化\\
  浓艳---淡雅\\
  虐待---优待\\
  鸟瞰---仰望\\

  \textbf{P}\\
  平常---特别\\
  平凡---非凡\\
  普通---特别\\
  漂泊---安稳\\
  迫切---从容\\

  \textbf{Q}\\
  牵挂---忘怀\\
  强迫---自愿\\
  全部---局部\\
  全面---片面\\
  蜷缩---伸直\\

  \textbf{S}\\
  洒脱---拘束\\
  散漫---严整\\
  神奇---平常\\
  实质---表象\\
  怂恿---劝阻\\

  \textbf{W}\\
  挖苦---奉承\\
  微观---宏观\\
  围拢---散开\\
  维持---改变\\
  无私---自私\\

  \textbf{X}\\
  昔日---今朝\\
  详情---概况\\
  幸福---痛苦\\
  幸运---倒霉\\
  寻常---特别\\

  \textbf{Y}\\
  压抑---舒展\\
  野蛮---文明\\
  一致---分歧\\
  依附---独立\\
  渊博---浅薄\\

  \textbf{Z}\\
  展望---回顾\\
  真理---谬论\\
  中断---连续\\
  庄重---随便\\
  自在---拘束\\

\end{multicols}

\subsection{成语}

\index{成语}
成语是人们从历史上传承下来并长期使用的结构凝固、语义精辟且具有民族特点的熟语。

\subsubsection{成语的特点}

\begin{enumerate}
  \item 历史的传承性。大多数成语沿袭自神话寓言(如“愚公移山”)、历史故事(如“四
    面楚歌”)、诗文语句(如“学而不厌”)或口头俗语(如“鸡毛蒜皮”),具有历史的传
    承性。

  \item 结构的凝固性。成语结构的凝固性主要表现在成语字数、构成成分和结构关系的
    固定性上。绝大部分成语由四字构成,结构关系和次序不能随便改变,例如“满城风雨”
    不能改成“风雨满城”。

  \item 意义的整体性。大多数成语的整体意义是比喻义。例如“为虎作伥”,本义是为老
    虎引路的鬼。比喻义是给坏人做帮凶,为坏人效劳。
\end{enumerate}

\subsubsection{成语辨析技巧}

\begin{enumerate}
  \item 精准理解词义,避免望文生义。例如:
    \begin{equotation}
      围棋等棋类游戏能很好地培养人的统筹意识和战略眼光,因为如果目无全牛,就很可
      能因顾此失彼而落败。
    \end{equotation}
    “目无全牛”语出《庄子·养生主》:“三年之后,未尝见全牛也。”形容技艺已达到十分
    纯熟的地步。

    再列举一些常因望文生义而用错的成语:

    \begin{description}
      \item[差强人意] 大体上能让人满意。不能理解为“不满意”。

      \item[火中取栗] 比喻冒险给别人出力,自己却上了大当,一无所得。也指冒险行
        事,使自己蒙受损失。不能理解为“冒危险为自己捞取好处”。

      \item[不瘟不火] 指戏曲不沉闷乏味,也不急促,恰到好处。不能理解为“不火爆”。

      \item[首当其冲] 指首先受到攻击或遭遇灾难。不能理解为“冲在最前面或首要的”。

      \item[久假不归] 长期借用,不归还。不能将“假”理解为“放假”。

      \item[当仁不让] 泛指遇到应该做的事,积极主动去做,不推让。不能理解为“理所
        当然”。

      \item[不足为训] 指不能当作范例或法则。不能理解为“不值得作为教训”。

      \item[七月流火] 天气转凉。不能理解为“天气很热”。

      \item[瓜田李下] 经过瓜田,不弯下身来提鞋,免得人家怀疑摘瓜;走过李树下面,
        不举起手来整理帽子,免得人家怀疑摘李子。泛指容易引起嫌疑的地方。不能理
        解为“瓜田边,李子树下”。

      \item[不赞一词] 原指文章写得很好,别人不能再添一句话,现在也指一言不发。
        不能只关注“赞”。

      \item[明日黄花] 原指重阳节过后菊花日渐枯萎,没什么好玩赏的了。比喻过时或
        无意义的事物。后多比喻已失去新闻价值的报道或已失去应时作用的事物。不能
        理解错写成“昨日黄花”。

    \end{description}

  \item 明确感情色彩,避免褒贬误用。例如:
    \begin{equotation}
      他性格比较内向,平时沉默寡言,但是一到课堂上就变得振振有词,滔滔不绝,所以
      他的课很受学生欢迎。
    \end{equotation}
    “振振有词”形容理由似乎很充分,说个不休。多用于贬义,本句语境是褒义。
    \begin{equotation}
      少年强则国强,一旦一个国家的青少年群体普遍沉溺于玩乐,胸无城府,不思进取,
      不关心国家大事,那么这个国家的未来则让人担忧。
    \end{equotation}
    “胸无城府”形容人坦率真诚,没有心机。褒义词。用在本句中不恰当,可改为“胸无大
    志”。

    再列举一些褒贬易误用的成语:

    \begin{description}

      \item[倾巢而出] 比喻敌人出动全部兵力侵扰。

      \item[面目全非] 事物的样子改变得很厉害。

      \item[上下其手] 指玩弄手法,暗中作弊。

      \item[坐而论道] 原指坐着议论政事,后泛指空谈大道理。

      \item[一团和气] 原指和蔼可亲,现多指态度温和而缺乏原则。

      \item[无独有偶] 虽然罕见,但是不止一个,还有一个可以成对儿(多用于贬义)。

      \item[炙手可热] 炙,烤。指手一靠近就觉得热;比喻气焰权势之盛。

      \item[弹冠相庆] 指一人当了官或升了官,他的同伙也互相庆贺将有官可做(含贬义)。

      \item[过江之鲫] 现在形容赶时髦的人很多、连续不断。

      \item[满城风雨] 形容事情传遍各处,到处都在议论着(多指坏事)。

    \end{description}

    (以上为贬义成语)

    \begin{description}

      \item[惨淡经营] 费尽心思辛辛苦苦地经营筹划。后指在困难的境况中艰苦地从事某种事业。

      \item[苦心孤诣] 费尽心思钻研或经营,达到别人达不到的境地。

      \item[凤毛麟角] 比喻稀少而可贵的人或事物。

      \item[叹为观止] 指赞美看到的事物好到极点。

      \item[蔚然成风] 形容一种事物逐渐发展流行,形成风气。

      \item[毁家纾难] 捐献全部家产,帮助国家缓解危难。

      \item[雨后春笋] 比喻新事物大量出现。

      \item[侃侃而谈] 形容说话理直气壮,从容不迫。

      \item[死得其所] 形容死得有意义、有价值。

      \item[特立独行] 指有操守、有见识,不随波逐流。
    \end{description}

    (以上为褒义成语)

  \item 明确适用范围,避免用错对象。例如:
    \begin{equotation}
      小庄从小就对机器人玩具特别感兴趣,上学后喜欢收集机器人模型,通过各种途径得
      到的模型已经汗牛充栋,摆满了整整一间屋。
    \end{equotation}
    “汗牛充栋”形容书籍极多。句中用来形容“模型”多,属于错用。
    \begin{equotation}
      屏幕中的剧情风生水起,扣人心弦。
    \end{equotation}
    “风生水起”指风从水面吹过,水面掀起波澜。形容事情做得有生气,蓬勃兴旺。不能
    用来形容“剧情”。

    再举一些易用错对象的成语:

    \begin{description}

      \item[巧夺天工] 精巧的人工胜过天然,形容技艺极其精巧。该词的适用范围仅限
        于人工产品,不能用于形容自然美景。

      \item[浩如烟海] 形容文献、资料等极为丰富。不能用于形容“地方宽阔”。

      \item[扣人心弦] 形容诗文、表演等有感染力,使人心情激动。不能用于形容灾情、
        局势等。

      \item[倚马可待] 形容文思敏捷,写文章快。不能用于形容做事速度快。

      \item[豆蔻年华] 指女子十三四岁的年纪。不能用于男子。

      \item[草长莺飞] 形容江南春天的景色。只能用于形容江南的景色,不能用于形容
        北方草原。

      \item[间不容发] 两物中间容不下一根头发,形容事物之间距离极小,也形容与灾
        祸相距极近,情势极其危急。不能用来形容关系亲密。

      \item[积重难返] 长期形成的不良的风俗、习惯不易改变。也指长期积累的问题不
        易解决。不能用来形容具体的事物。

      \item[活灵活现] 形容描述或模仿的人或事物生动逼真,使用对象是描述的形象或
        假物。不能用于真实的事物。

      \item[三人成虎] 比喻流言惑众,使人以假为真,使用对象是流言、讹传等。不能
        用于表示团结。

    \end{description}

  \item 把握谦辞、敬辞之分,避免谦敬错位。例如:
    \begin{equotation}
      在群众路线教育实践活动中,领导们大多都摒弃了过去在主席台上大讲特讲的习惯,
      而是深入到群众中去座谈,让群众洗耳恭听。
    \end{equotation}
    “洗耳恭听”指专心地听(请人讲话时说的客气话),一般用作敬辞。而此句中说“让群
    众洗耳恭听”,谦敬失当。
    \begin{equotation}
      小王同学站起来说道:“陈教授刚才那番话抛砖引玉,我下面要讲的只能算是狗尾续
      貂。”
    \end{equotation}
    “抛砖引玉”比喻用粗浅的、不成熟的意见引出别人高明的、成熟的意见。是谦辞,用
    于“陈教授”有失恭敬。

    再列举一些易谦敬错位的成语:

    \begin{description}

      \item[蓬荜生辉] 表示由于别人到自己家里来或张挂别人给自己题赠的字画等而使
        自己非常光荣。

      \item[敝帚自珍] 一个破扫把,自己也十分珍惜。比喻自己的东西再不好也值得珍
        惜。

      \item[敬谢不敏] 敬,恭敬;谢,推辞;不敏,不聪明,没有才能。表示推辞做某
        事的客气话。

      \item[忝列门墙] 忝,辱没他人,自己有愧。表示自己愧在师门。

      \item[才疏学浅] 才能低,学识浅。

      \item[挂一漏万] 挂,列举;漏,遗漏。形容列举不全,遗漏很多。

      \item[不情之请] 客套话,不合情理的请求(向人求助时称自己的请求)。

      \item[姑妄言之] 姑且说说(对于自己不能深信不疑的事情,说给别人时常用此语
        以示保留)。

      \item[雕虫小技] 比喻微不足道的技能(多指文字技巧)。

      \item[管窥蠡测] 管,竹管;窥,从小孔或缝隙里看;蠡,瓢。从竹管里看天,用
        瓢来量大海。比喻眼光狭窄,见识短浅。

    \end{description}

    (以上成语易谦辞错位)

    \begin{description}

      \item[高抬贵手] 客套话,多用于请求对方饶恕或通融。

      \item[不吝赐教] 敬辞,用于自己向别人征求意见或请教问题。

      \item[鼎力相助] 敬辞,大力相助(表示请托或感谢时用)。

      \item[洗耳恭听] 专心地听(请人讲话时说的客气话)。

      \item[大材小用] 大的材料用在小处。多指人事安排上不恰当,屈才。

      \item[率先垂范] 带头给下级或晚辈做示范。

      \item[虚怀若谷] 谦虚的胸怀像山谷一样空旷深广。形容非常谦虚。

      \item[虚左以待] 虚,空着;左,古时以左位为尊。空着左边的位置等待客人,表
        示尊敬。也泛指留出位置恭候他人。

    \end{description}

    (以上成语易敬辞错位)

  \item 整体理解句段,避免语意重复。例如:
    \begin{equotation}
      他们或娓娓道来地讲述文物的历史,或扮成古人演绎国宝故事。
    \end{equotation}
    “娓娓道来”指连续不断地说,生动地谈论,与后面的“讲述”语意重复。
    \begin{equotation}
      要解决愈演愈烈的医患矛盾,既需要运用法律武器制止违法行为,更需要从根本上釜
      底抽薪,进一步推进医药卫生体制改革。
    \end{equotation}
    “釜底抽薪”比喻从根本上解决问题,和前面的“从根本上”语意重复。

    再列举一些易语意重复的成语:

    \begin{description}

      \item[难言之隐] 隐,隐情。难于说出口的藏在内心深处的事情。不能说“难言之隐
        的苦衷”。

      \item[相形见绌] 跟另一人或事物比较起来显得远远不如。不能说“显得相形见绌”。

      \item[哀鸿遍野] 比喻呻吟呼号,流离失所的灾民到处都是。不能说“灾民哀鸿遍
        野”。

      \item[如芒在背] 像芒和刺扎在背上一样,形容坐立不安。不能说“好像如芒在背”。

      \item[遍体鳞伤] 遍,全部;鳞,鱼鳞,这里指伤痕布满全身,像鱼鳞一样密。形
        容伤势非常重。不能说“浑身被打得遍体鳞伤”。

      \item[当务之急] 当前急切应办的事。不能说“目前的当务之急”。

      \item[生灵涂炭] 形容人民处于极端困苦的境地。不能说“人民/百姓生灵涂炭”。

      \item[妄自菲薄] 过分看轻自己。形容自卑。不能说“妄自菲薄自己”。

      \item[莘莘学子] 众多的学子。不能说“众多莘莘学子”。

      \item[贻笑大方] 指让内行人笑话。不能说“让人/使人/被人贻笑大方”。

      \item[芸芸众生] 泛指众多的平常人。不能说“许多芸芸众生”。

    \end{description}

  \item 明确语法规则,避免语法错误。例如:
    \begin{equotation}
      有些领导漠不关心人民群众的疾苦。
    \end{equotation}
    “漠不关心”不能带宾语。

    再列举一些易犯语法错误的成语:

    \begin{description}

      \item[含英咀华] 比喻琢磨和领会诗文的要点和精神。动词性成语,易误用作形容词。

      \item[不谋而合] 没有事先商量而彼此的见解或行动完全一致。该成语一般充当谓语,
        不充当状语。

      \item[念念有词] 旧时迷信的人小声念咒语或说祈祷的话,也指人不停地自言自语。
        该成语不能充当状语来修饰“说”。

      \item[水到渠成] 水流到的地方自然成渠,比喻条件成熟,事情自然成功。该成语不
        能充当状语,比如“水到渠成地推进”。

      \item[街谈巷议] 大街小巷里人们的谈论。名词性成语,易误用作动词。

      \item[正襟危坐] 整理好衣襟端端正正地坐着,形容严肃或拘谨的样子。后面不能带
        表示处所的介宾短语。

      \item[萍水相逢] 比喻向来不认识的人偶然相遇。后面不能带表示处所的介宾短语。

      \item[望其项背] 赶得上或比得上。往往用在否定句中,例如“不能望其项背”。

      \item[视为儿戏] 比喻不当一回事,极不重视。常常用在疑问句中,例如“怎可视为儿
        戏”。

      \item[耳濡目染] 形容见得多听得多了之后,容易受到影响。不能带宾语,不能说“孩
        子很容易耳濡目染父母的言行”。

    \end{description}

  \item 理解句段情境,避免不合语境。例如:
    \begin{equotation}
      比赛过后,教练希望大家重整旗鼓,继续以高昂的士气、振奋的精神、最佳的竞技状
      态,在下一届赛事中再创佳绩。
    \end{equotation}
    “重整旗鼓”指失败之后,重新集合力量再干(摇旗和击鼓是古代进军的号令)。该句
    说“再创佳绩”,表明本次比赛并未失利。
    \begin{equotation}
      正在悠闲散步的外科主任王教授,突然接到护士电话说有个病人情况危急,他立刻安
      步当车向医院跑去。
    \end{equotation}
    “安步当车”指慢慢地步行,就当作是坐车。语境强调王教授听说病人情况危急,立刻
    向医院赶去。

\end{enumerate}

\subsubsection{近义成语使用辨析}

\begin{enumerate}
  \item 根据语意侧重点
    \begin{equotation}
      要知道生之可贵,但不可;要知道死不足惧,但不可轻易言死。(苟且偷生~苟且偷
      安)
    \end{equotation}
    “苟且偷生”与“苟且偷安”都有“只顾眼前的安乐,不顾长远的利益”的意思。但“苟且偷
    生”指得过且过,勉强地生存下去,重在生存;“苟且偷安”指只图眼前安逸,得过且过,
    重在安逸。依据“生之可贵”的语境,应选“苟且偷生”。

  \item 根据语气与语意程度轻重
    \begin{equotation}
      上天在惩治一个人的罪孽之前,会先让他得意一时,过上一段太平时日,这样在他,
      最终得到报应时,才会有切肤之痛。(咎由自取~罪有应得)
    \end{equotation}
    “咎由自取”:遭受责备、惩处或祸害是自己造成的。“罪有应得”:干了坏事或犯了罪
    得到应得的惩罚。两者都有“惩罚是自己招来的,应该的”的意思,但“咎由自取”语义
    比较轻,“罪有应得”的语义比较重。横线处应选“罪有应得”。

  \item 根据适用范围大小
    \begin{equotation}
      任何人,任何事物,只要与我的身体利益有关,就不能真正占据我的心。我只有忘
      掉自己,才能地进行沉思和遐想。(津津乐道~津津有味)
    \end{equotation}
    “津津乐道”与“津津有味”都可以形容讲话时兴致勃勃,但“津津乐道”仅指很感兴趣地
    谈论,适用范围较窄;“津津有味”不仅指谈论,也可指有兴趣地看着、听着或吃得很
    有滋味,适用范围较广。应选“津津有味”。

  \item 根据适用对象
    \begin{equotation}
      人人都会渴求自己具有一种的精神状态,期望获得充满活力与效能的心理机能;人
      总是希望最高程度地发展自己的智力与才能,从而能为社会做出卓有成效的贡献。
      (生机勃勃~生意盎然)
    \end{equotation}
    “生机勃勃”与“生意盎然”都有“生命力旺盛”的意思,多用来形容有生气、充满活力。
    其区别主要在适用对象上,“生机勃勃”的适用对象既可以是草木、自然景象,也可以
    是人、社会气象;“生意盎然”的适用对象只能是草木、自然景象,不能用来形容人,
    也不能用来形容社会气象。因此,应选“生机勃勃”。

  \item 根据语法特征
    \begin{equotation}
      挑选优质机芯,注重售后服务,注意手表的外观、启动性能、拨针、走时精度等,
      学会了这些技巧,就能挑选一款的腕表。(如愿以偿~称心如意)
    \end{equotation}
    “如愿以偿”与“称心如意”都有“遂心如愿”的意思,但“如愿以偿”多用作谓语,也作状
    语,一般不作定语;“称心如意”多用作定语,也常用作谓语、状语。例句待填词语用
    作“腕表”的定语,应选“称心如意”。

  \item 根据感情色彩
    \begin{equotation}
      帮助王东一家的两年来,刘老师自己也搭了不少钱,但她始终认为,毕竟这是她曾
      经教过的学生,现在学生需要帮助了,她作为启蒙老师不能。(冷眼旁观~袖手旁观)
    \end{equotation}
    “袖手旁观”与“冷眼旁观”都有“从旁观看,置身事外”的意思,但“袖手旁观”偏重于不
    过问,不协助,应该给予帮助而不帮助,含贬义;“冷眼旁观”偏重不热情、不关心,
    指可以参与而不愿意参与。指用冷淡的态度从旁观看时,含有贬义(但当指用冷静的
    态度从旁观察时,并不含贬义)。依据“需要帮助”和“不能……”的语境,应选“袖手旁
    观”。

\end{enumerate}

\subsubsection{容易写错的成语50例}

\begin{multicols}{3}
  \begin{enumerate}
    \item 按部(步)就班
    \item 卑躬屈(曲)膝
    \item 变本加厉(利)
    \item 不徇(循)私情
    \item 不知所措(错)
    \item 痴心妄(忘)想
    \item 独出心(新)裁
    \item 翻天覆(复)地
    \item 风尘仆仆(扑)
    \item 耿耿(梗)于怀
    \item 归根结蒂(底)
    \item 汗流浃(夹)背
    \item 黄粱(梁)一梦
    \item 诲(悔)人不倦
    \item 积重难返(反)
    \item 集(积)思广益
    \item 既往不咎(究)
    \item 矫(娇)揉造作
    \item 噤(禁)若寒蝉
    \item 开源节(截)流
    \item 寥寥(廖)无几
    \item 漫(慢)不经心
    \item 明辨(辩)是非
    \item 莫名(明)其妙
    \item 墨(默)守成规
    \item 弄巧成拙(绌)
    \item 奴颜婢(卑)膝
    \item 呕(沤)心沥血
    \item 轻歌曼(慢)舞
    \item 屈(曲)指可数
    \item 惹是(事)生非
    \item 融会(汇)贯通
    \item 矢(失)口否认
    \item 水泄(泻)不通
    \item 耸(怂)人听闻
    \item 铤(挺)而走险
    \item 乌烟瘴(障)气
    \item 瑕不掩瑜(玉)
    \item 相辅相成(承)
    \item 相形见绌(拙)
    \item 销(消)声匿迹
    \item 休(修)养生息
    \item 要言不烦(凡)
    \item 一气呵(哈)成
    \item 贻(遗)笑大方
    \item 有恃(侍)无恐
    \item 原形毕(必)露
    \item 针砭(贬)时弊
    \item 谆谆告诫(戒)
    \item 走投(头)无路
  \end{enumerate}
\end{multicols}

\subsubsection{容易用错的成语100例}

\begin{enumerate}
  \item 安之若素~对困窘的遭遇毫不介意,心情平静得像往常一样。现在也指对于错误的
    言论或事物不闻不问。

    误用:真是好事多磨,在经历了许多挫折后,他的公司终于正式成立了,他也可以安
    之若素了。

  \item 八面玲珑~原指窗户宽敞明亮,后用来形容人处世圆滑,不得罪任何一方。

    误用:这块天然宝石,晶莹剔透,八面玲珑,光彩夺目,可称世间极品。

  \item 白头如新~相交虽久而并不知己,像新知一样,指交友互不知心。

    误用:老王和老李曾非常要好,20多年前,两人产生了矛盾,一直互不理睬。退休后,
    一件偶然的事,消除了他们多年的隔阂,两人和好如初,白头如新,大家也为之高兴。

  \item 坂上走丸~坂(bǎn),山坡,斜坡。形容事情发展很快。

    误用:从某种意义上讲,搞财务工作犹如坂上走丸,有一定的风险,只有精通业务,
    严于律己,才能“化险为夷”。

  \item 抱残守缺~守着残缺的东西不放。形容思想保守,不肯接受新事物。不能用于对某
    种旧物有感情。

    误用:一只发黑的藤编书箱,一把破旧的竹躺椅,放在哪儿都碍事,可祖父说跟它们
    有“乡情”,不肯扔掉。我想,这是老年人特有的抱残守缺的心理,应当理解。(可用
    “敝帚自珍”。)

  \item 比翼双飞~比翼,即比翼鸟,比喻夫妻。如比翼鸟双宿双飞,形影不离。比喻夫妻
    恩爱。不能用于两个男女学生。

    误用:一家报纸曾刊载一篇题为“比翼双飞两‘状元’,”的文章,报道某市男女两位高
    考状元的事。

  \item 不耻下问~不以向地位、学问较自己低的人请教为可耻。

    误用:老李从小就养成了勤学好问的良好习惯,遇到问题,总是不耻下问,及时向同
    事、亲朋好友甚至左邻右舍请教。

  \item 不孚众望~孚,使人信服。不能使大家信服。

    误用:在球迷的呼吁下,教练使用了巴乔,他在世界杯上果然不孚众望,多次挽救了
    意大利队。(句中应用“不负众望”。)

  \item 不假思索~用不着想。形容说话做事迅速。

    误用:同志们都认为,他这个人办事向来深思熟虑,计划周密,不假思索。

  \item 不绝如缕~像细线一样连着,差一点就要断了。多用来形容局势危急或声音细微悠
    长。

    误用:五一长假,来黄山旅游的人络绎不绝,不绝如缕。

  \item 不刊之论~刊,消除。古代把字写在竹简上,有错误就削去;不刊,比喻不能改动
    或不可磨灭。形容不能改动或不可磨灭的言论。

    误用:这篇文章写得太差,真是不刊之论。

  \item 不可开交~开交,结束,解决。无法摆脱或结束。

    误用:新春佳节将至,这家著名的大商场人头攒动,挤得不可开交。

  \item 不可理喻~喻,使明白。不能够用道理使他明白。

    误用:对外国人来讲,京剧舞台上那种木头刀枪稀松一碰,口中一吆喝就打了一仗,
    简直不可理喻。

  \item 不名一文~名,指占有。形容穷到极点,连一文钱也没有。

    误用:王宝森之流贪污腐化,挥霍浪费国家财产,他的人格可说是不名一文。(句中
    可用“不值一钱”。)

  \item 不能自已~已,停止。指无法控制自己,使激动的情绪平静下来。

    误用:课堂上我一时不能自已,竟然趴在桌子上睡着了,结果被老师提到办公室狠批
    了一顿。

  \item 不情之请~不合情理的要求。常用作对人有所请托的客套话。

    误用:有的人向人民伸手要官要权,这种不情之请绝不答应。

  \item 不忍卒读~不忍心读完,多形容文章悲惨动人。

    误用:随着出版业的市场化和多元化,类型多样、题材丰富的作品大量涌现,其中也
    有一些作品粗制滥造,令人不忍卒读。

  \item 不容置喙(huì)不容别人插嘴。喙,嘴。

    误用:“权钱交易”,“权权交易”等时下的腐败病症,在这个领域里虽不能说样样俱全,
    但该领域遭受“感染”却是不容置喙的事实。

  \item 不胜其烦~烦琐得使人受不了的意思。

    误用:马大嫂为人热情,工作兢兢业业,总是不胜其烦地为小区居民做好每一件事。

  \item 不以为然~然,对。不认为是对的,表示不同意。

    误用:开始,人家送礼他都不收,时间长了,他就认为是小事一桩,犯不着太认真,
    也就不以为然了。

  \item 城下之盟~因敌军兵临城下而被迫签订的屈辱性的和约。多误解为签订下合同。

    误用:两个贪婪的家伙以为法院没有掌握他们鲸吞公款的罪行,利用机会订立城下之
    盟。

  \item 充耳不闻~充,堵塞。塞住耳朵不听。形容存心不听别人的话。

    误用:这个世纪是信息时代,一个人如果整日充耳不闻,不去了解大千世界的变化,
    是不可能有大作为的。

  \item 春秋笔法~孔子修订《春秋》语句中含有褒贬。后人就指文笔曲折而意含褒贬的文
    字为“春秋笔法”。

    误用:何先生西洋油画的功底非常深厚,对中国画的春秋笔法也十分熟稔,寥寥几笔,
    一个鲜活的形象便跃然纸上。

  \item 大方之家~大方,大道理。懂得大道理的人。后泛指见识广博或学有专长的人。

    误用:告别时,他非要送我几块高档衣料不可,真是大方之家。(句中把“大方”误解
    为“不吝啬”。)

  \item 大快人心~指坏人受到惩罚或打击,使大家非常痛快。同“拍手称快”。

    误用:改革开放政策富裕了老百姓,真是大快人心。

  \item 弹冠相庆~因即将做官而互相庆贺,多用于贬义。

    误用:1993年,森达皮鞋就卫冕了中国“鞋业大王”称号,没容他的“农民军团”弹冠相
    庆,朱相桂已开始深思了。

  \item 得意忘言~既已领会其意旨,则不再需要表意之言词。后亦引申为彼此默喻,心照
    不宣。

    误用:听到女儿考上重点大学的好消息,老李竟高兴得手舞足蹈,得意忘言。

  \item 登堂入室~堂、室,古代宫室,前面是堂,后面是室。登上厅堂,进入内室。比喻
    人在学问或技艺方面由浅入深,循序渐进,达到更高的水平。

    误用:朱光潜幼年常蹲在教室窗下听父亲讲课,一次被父亲无意间发现,便让他登堂
    入室,他成了父亲最年幼的学生。

  \item 独到之处~指与众不同的见解。

    误用:塑料有不受酸碱腐蚀的独到之处,这是钢铁所不及的。

  \item 耳提面命~拉着耳朵当面指导,形容教诲恳切,要求严格。

    误用:教育学生要讲究方式方法,不能总是耳提面命,摆家长作风。

  \item 翻云覆雨~比喻耍手段,弄权术,反复无常。贬义。

    误用:辛弃疾继承并发扬了苏东坡的豪放风格,以翻云覆雨的笔力,激昂跌宕的气势,
    抒情言志,针砭现实,形成南宋词坛一大流派。

  \item 犯而不校~校,计较。别人触犯了自己也不计较。

    误用:一个人在工作中难免有一些缺点和错误,只要认真改正就行,不能犯而不校。
    (可用“一意孤行”,指不接受别人的劝告,顽固地按照自己的主观想法去做。)

  \item 粉墨登场~粉墨,指化妆用品。化妆登台演戏,也用于讽刺某些人登上政治舞台。

    误用:王帆竞选班长一职成功,就职演说那天,他精心准备后粉墨登场。

  \item 风起云涌~常比喻多种力量或事物并起,发展迅速,声势浩大。

    误用:中国人民解放军胜利地渡过长江,以风起云涌之势迅速地歼灭了盘踞在江南的
    国民党军队。(可用“风卷残云”。)

  \item 风声鹤唳~唳,鹤叫。常与“草木皆兵”配合使用,形容惊慌失措或自相惊扰。

    误用:近年来这份报纸引起的报业大战,杀得人仰马翻、天昏地暗、风声鹤唳。

  \item 风雨飘摇~形容形势很不稳定。

    误用:流浪的人快回来吧,不要再过这种风雨飘摇的日子了。

  \item 凤毛麟角~凤凰的毛,麒麟的角。比喻稀少而可贵的人或事。不是指稀少。

    误用:在今年的“排队推动日”活动中,虽仍有凤毛麟角的几个“不自觉者”,但广大市
    民不论乘车还是购物都能自觉排队。

  \item 俯拾即是~俯下身子就能拾到。形容数量极多,随处可得。

    误用:珠宝专卖店的柜子里,各种式样的名贵宝石俯拾即是,吸引了许多顾客。(可
    用“琳琅满目”。)

  \item 付之一笑~一笑了之,表示毫不介意。

    误用:他待人态度谦和,不论遇到谁,都付之一笑。(句中误解为对人笑脸相迎。)

  \item 改头换面~比喻只改形式,不改内容。贬义。

    误用:在市场经济条件下,这家小店不仅装饰一新,而且转变了服务作风,改头换面,
    焕然一新。

  \item 改弦易辙~易,更换;辙,车轮轧下的痕迹,这里指道路。乐器换掉弦,车子改换
    道路。比喻变更方向、计划或做法。

    误用:虽然这个店的招牌几易其名,改弦易辙,但因其服务质量差,顾客仍然很少。

  \item 肝脑涂地~原来形容惨死,后来表示竭尽忠诚,不惜任何牺牲。

    误用:西山村伏击战中,日寇被八路军打得横尸阡陌、肝脑涂地。

  \item 高屋建瓴~建,倾倒;瓴,水瓶。把水瓶从高屋脊上向下倾倒。比喻居高临下,其
    势不可阻挡。

    误用:这座度假村建在山的最高处,面对着一望无际的大海,远远望去,的确给人以
    高屋建瓴之感。

  \item 歌功颂德~意思是颂扬功绩和德行。多用于贬义。

    误用:他们的火一般的激情,为这些好人好事歌功颂德。

  \item 后起之秀~秀,特别优异的。后出现的或新成长起来的优秀人物。

    误用:刚一起跑,高三(1)班的夏曦就滑倒了。他爬起来奋力追赶,离终点20米时终
    于成为后起之秀,夺得3000米跑的第一名。(并非指开始时落后,后来超越他人的人,
    可用“后来居上”。)

  \item 涣然冰释~涣然,流散的样子。流散,消失得像冰块消融一样。一般比喻疑团解除。

    误用:这部轻喜剧逗得大家哈哈大笑,人们所有的烦恼都涣然冰释了。

  \item 祸起萧墙~萧墙:古代宫室内当门的小墙(照壁),比喻为内部。祸害起于内部。

    误用:没想到,由于楼房的工程质量不过关,结果造成严重事故,真是祸起萧墙。
    (句中误把“萧墙”视为楼房的墙壁了。)

  \item 既往不咎~意思原指已经做完或做过的事,就不必再责怪了;现指对以往的过错不
    再责备。

    误用:日本军国主义分子抱着既往不咎的态度,恶意篡改侵华历史,这是中国人民绝
    不能接受的。

  \item 江河日下~江河的水逐日流向下游。比喻事物日衰,景象日非。

    误用:近几年,黄河、岷江的部分河段多次出现断流现象,面对这江河日下的情况,
    人们开始冷静地思考环保问题。

  \item 孑然一身~孑,单独。孤零零的一个人。

    误用:月明星稀,夜深人静,王小晓独自孑然一身地匆匆穿过小巷,闪进了巷口的一
    个漆黑的大门。

  \item 津津乐道~津津,兴趣浓厚的样子;乐道,喜欢谈论。指很有兴趣地去谈论。

    误用:这件事情非常有趣,满脸笑容的他讲得津津乐道。(句中“津津乐道”与“讲得”
    重复,可删去“讲得”。)

  \item 泾渭分明~比喻界限清楚。

    误用:这是一个很敏感的问题,一提到这个问题,就会公说公有理,婆说婆有理,很
    难泾渭分明。

  \item 绝无仅有~形容极其少有。

    误用:可以断言,所有大大小小的知识分子,没有得到过这位“不说话的老师”(指各
    类辞书)指教的,绝无仅有。(不是“绝对没有”的意思。)

  \item 慷慨解囊~解囊,打开钱袋。形容豪爽大方地在经济上给人帮助。

    误用:仅仅靠一双脚板、一块块地搜集,很难收到较多的奇石。为了充实自己的“奇石
    王国”,他常常慷慨解囊,上门求购别人珍藏的奇石。

  \item 良莠不齐~莠,类似谷子的野草。好苗和野草混杂在一起。常比喻好人和坏人难以
    区分。

    误用:参赛歌手的素质良莠不齐,在过“文化关”时,着实让电视机前的观众大跌眼镜。

  \item 淋漓尽致~淋漓,渗透了水的样子,比喻尽情、酣畅;尽致,达到极点。形容文章
    或说话表达得详尽、透彻,也指暴露得很彻底。

    误用:这篇文章把敌人的反动论点批驳得淋漓尽致。(可改为“体无完肤”。)

  \item 另眼相看~用另一种眼光去看待,多指看待某个人不同于一般。

    误用:进入高三以来,一向学习成绩平平的陈浩一直特别刻苦,进步很大,真叫人不
    得不另眼相看。(句中可用“刮目相看”。)

  \item 屡试不爽~爽:差错。经过多次试验都没有差错。

    误用:他一连几次都没考好,真是屡试不爽,因而心情十分沉重。(与“没有考好”矛
    盾。)

  \item 门可罗雀~门前可以张网捕雀,形容门庭冷落,宾客稀少。

    误用:可是好运不长,餐馆逐渐由门可罗雀到无人问津,终于关门大吉。

  \item 南辕北辙~比喻行动和目的相反。

    误用:今天还有不少教师通过给学生留大量的作业以达到取得高分的目的,显然与素
    质教育南辕北辙。

  \item 披肝沥胆~披,披露;沥,往下滴。比喻开诚相见,也形容极尽忠诚。

    误用:目前,全国公安系统展开了一场轰轰烈烈的“打拐”斗争,令犯罪分子披肝沥胆,
    闻风而逃。(句中应为“胆战心惊”。)

  \item 七手八脚~形容好几个人一起动手,也指人手多,干活忙乱的样子。

    误用:集合的号声已经响了,他还在七手八脚地收拾着行李。

  \item 期期艾艾~形容口吃的人吐词重复,说话不流利。

    误用:愿一切生命不致因飘落在缝间而期期艾艾。(应用“自怨自艾”,原意是悔恨自
    己的错误,自己改正。现在只指悔恨自己的错误。)

  \item 起死回生~把将要死的人医活,形容医术高明。不用于病人。

    误用:一个月后他的病逐渐好起来,这是他第四次起死回生了。(可改为“死而复
    生”。)

  \item 巧夺天工~人工的精巧胜过自然。

    误用:山上的石头奇形怪状,有的像猴子嬉戏,有的像双龙衔珠,有的似莲花盛开……
    真是巧夺天工。(应用“鬼斧神工”,形容自然造化的神奇,也可形容艺术技巧高超,
    不是人力所能达到的。)

  \item 琴瑟失调~比喻夫妇不和。

    误用:外援和主教练在转会费和出场费等问题上意见不合,终于琴瑟失调,不得不分
    手。

  \item 求全责备~责,要求;备,全。对人对事要求完美无缺。

    误用:有的人生前尽量为自己树碑立传,文过饰非,很少像秋白同志这样坦荡无私,
    光明磊落,求全责备自己。(句中把“责备”误解为批评指责。)

  \item 趋之若鹜~像鸭子一样成群地争先恐后地跑去。贬义。多比喻许多人争着去追逐不
    好的事物。

    误用:齐白石画展在美术馆开幕了,国画研究院的画家竞相观摩,艺术爱好者也趋之
    若鹜。

  \item 忍俊不禁~忍俊,含笑;不禁,不能自制。忍不住要发笑。

    误用:看到他这种滑稽的表情,坐在身旁的一名外国记者忍俊不禁扑哧一声笑起来。

  \item 身无长(cháng)物~没有多余的东西。形容穷困或俭朴。

    误用:近年很多名牌大学毕业生,除了书本知识外便身无长物,被认为缺乏一技之长
    而在现代职场中难以立足。

  \item 十室九空~室,人家。十家人家,九家空虚。形容因灾荒、战乱和横征暴敛致使百
    姓破产或流亡的惨象。

    误用:目前,全国空置房的总量已逾8000万平方米,国家已下决心从海南开始清理数
    千亿银行占用资金。在这种形势下,面对十室九空的楼盘,该董事会无奈决定低价拍
    卖空关近5年的50套商品房。

  \item 是可忍孰不可忍~指对人的重大罪行不可容忍,极度愤慨。

    误用:今夏洪水肆虐,淹没无数的城镇和大片的良田,是可忍孰不可忍,我们必须精
    诚团结,战胜洪魔。

  \item 通宵达旦~通宵,通夜、整夜。一直到天亮。

    误用:他站在门口微笑着说:“李师傅,你这样几天几夜通宵达旦地忙活,可要注意身
    体啊!”

  \item 玩火自焚~比喻干冒险或害人的勾当,最后受害的还是自己。

    误用:环境污染日趋严重,人类这种玩火自焚的行为如不停止,将自毁生存空间。
    (可用“作茧自缚”。)

  \item 万籁俱寂~万籁:自然界万物放出的各种声响。形容周围环境十分宁静。

    误用:号声一响,一连长一声“立正”,如潮似浪、热火朝天的操场,顿时万籁俱寂。

  \item 万人空巷~空巷,指街、巷的居民都走出来了。形容盛大集会或新奇事物轰动一时
    的情景。

    误用:这部精彩的电视剧播出时,几乎万人空巷,人们在家里守着荧屏,街上显得静
    悄悄的。

  \item 文不加点~点,涂改。文章一气呵成,无需修改。

    误用:只见他奋笔疾书,一气呵成,不消片刻,一篇佳作便展现在大家面前。只是文
    不加点,难以断句,不能不说是白璧微瑕。(句中误为不加标点。)

  \item 无可厚非~没有可以过分指责非议的。用于有一定小问题的人或事。

    误用:这部小说的构思又精巧又严密,真是无可厚非。(可改为“无懈可击”。)

  \item 无隙可乘~隙,裂缝,空子。没有空子可钻。

    误用:这部60万字的长篇小说,构思精巧,叙述严密,简直无隙可乘。(可用“无懈可
    击”:没有一点弱点可以让人攻击。形容十分严密,找不到一点漏洞,没有一点破绽让
    人攻击。)

  \item 舞文弄墨~形容玩弄文字技巧。多用于贬义。

    误用:他本来就喜欢舞文弄墨,再加上这两年的苦练,如今成了闻名的“笔杆子”。

  \item 下里巴人~本指古代楚国通俗歌曲,后泛指通俗的文学艺术。

    误用:白居易在地方为官时很注意接近民众,不管是乡间农妇还是下里巴人,他都谈
    得来,从他们那里得到了很多创作素材。

  \item 莘莘学子~莘莘,众多的样子。众多的学子。

    误用:那是一张两人的合影,左边是一位英俊的解放军战士,右边是一位文弱的莘莘
    学子。

  \item 信笔涂鸦~信,听凭,随意;涂鸦,比喻字写得很拙劣,随便乱涂乱画。贬义。

    误用:在“纪念周恩来诞辰一百周年”大会期间,著名画家石坚先生即席作画,他信笔
    涂鸦,似有神功,在三勾两画之中,一只展翅高飞的雄鹰便跃然纸上。

  \item 胸有成竹~比喻处理事情心里先有主意,有成算。

    误用:下乡前两天,党委又组织参加扶贫的干部认真学习了有关文件,使大家进一步
    明确政策,做到胸有成竹。

  \item 休戚相关~休,喜;戚,悲伤,不幸。彼此之间的忧喜、祸福都互相关联。形容彼
    此利害一致。一般用于人与人或人与集体之间,而不用于事物。

    误用:二十年的经历告诉我们,这场改革与我们国家的前途、民族的命运是休戚相关
    的。

  \item 休养生息~生息,人口繁殖。指在战乱之后,减轻人民负担,安定生活,发展生产,
    恢复元气。

    误用:刚刚打完这场比赛,队员们又赶往下一个赛场,有人趁乘车时间小憩一会儿,
    以休养生息。

  \item 言不由衷~衷,内心。说的话没有通过内心。形容虚伪敷衍,说的不是真心话。

    误用:我这人心直口快,刚才对你说的话也是想到就说,言不由衷,请你千万不要介
    意。(与“心直口快”等矛盾。)

  \item 言近旨远~旨,含义。话说得浅近,含义很深远。褒义。

    误用:说话写文章,第一要简明扼要,做到有的放矢;第二要朴素、自然,防止言近
    旨远。

  \item 一团和气~态度温和,没有原则。贬义。

    误用:这个医疗小分队,每到一处,跟当地群众都是一团和气,不摆架子。

  \item 义不容辞~道义上不允许推辞。

    误用:教育孩子不仅仅是学校的事,家长也有义不容辞的责任。(应用“责无旁贷”,
    自己的责任,不能推卸给别人。)

  \item 饮鸩止渴~鸩:浸过鸩鸟羽毛的毒酒。喝毒酒解渴,比喻只图解决眼前困难而不顾
    后患。

    误用:如果我们把缺点、错误掩盖起来,装作看不见,那无异于饮鸩止渴。(句中“把
    缺点、错误掩盖起来”,无解决眼前困难的意思。)

  \item 鱼目混珠~比喻拿假的东西冒充真的。

    误用:这些打着私营企业家旗号的人,也是鱼目混珠,其中有真正想干一番事业的,
    也不乏骗子。(应用“鱼龙混杂”,比喻好人和坏人混杂在一起。)

  \item 与人为善~与,和,跟。跟人一同做好事。现在泛指善意帮助人。

    误用:文明礼貌,和气待人,这种与人为善的美德,不仅商业活动中需要提倡,其他
    行业活动中也应该提倡。

  \item 曾几何时~才有多少时候。指时间过去没有多久。不能误为“曾经”。

    误用:今天的野狼峪,沟壑纵横,曾几何时,就将“天堑变通途”。

  \item 振聋发聩~聩,耳聋。发出很大的声响,使耳聋的人也能听见。比喻言论能使糊涂
    麻木的人清醒。

    误用:今天,天津体育馆内万余名观众的掌声经久不息,振聋发聩,淹没了馆外的惊
    雷。

  \item 炙手可热~炙,烤,烧。手一接触就感觉到热得烫人,比喻气焰盛,权势大。

    误用:家用电器降价刺激了市民消费欲的增长,原本趋于滞销的彩电,现在一下子成
    了炙手可热的商品。

  \item 置之度外~不(把生死、利害等)放在心上。多指执着追求而不考虑过多。

    误用:有的领导干部经不住奢华生活方式的诱惑,生活腐化,把人民的期望置之度外。

  \item 捉襟见肘~捉襟,整顿衣襟;见,同“现”,露出来。整一下衣襟,胳膊肘就露出来
    了。比喻顾此失彼,无法应付。

    误用:运动会上,他的一身衣服很不合身,真是捉襟见肘。

  \item 罪不容诛~罪大恶极,处死都不能抵偿。

    误用:他多次小偷小摸,罪不容诛,但公安机关最终释放了他。

  \item 左右逢源~比喻做事得心应手,怎样进行都很顺利。也比喻办事圆滑。

    误用:谈起电脑、互联网,这个孩子竟然说得头头是道,左右逢源,使在场的专家也
    惊叹不已。
\end{enumerate}

\subsubsection{近义成语辨析100例}

\begin{enumerate}
  \item 爱憎分明~泾渭分明

    同:有“界限清楚”的意思。

    异:专指思想感情上的爱与恨;多指人或事的好坏分得很清楚。

  \item 安分守己~循规蹈矩

    同:有“规矩老实”的意思。

    异:偏重守本分,不胡来;偏重拘泥成规,不敢改变。

  \item 安之若素~随遇而安

    同:有“对任何遭遇都不在意”的意思。

    异:多指处于困境,仍能跟往常一样;强调能适应任何环境。

  \item 按部就班~循序渐进

    同:有“遵循一定程序”的意思。

    异:强调按一定步骤和规矩;强调逐渐深入或提高。

  \item 暗箭伤人~含沙射影

    同:比喻暗中诽谤、攻击或陷害别人。

    异:使用范围包括语言、行动,程度较后者重;使用范围只包括语言,并有影射某人、
    某事的意思。

  \item 八面玲珑~面面俱到

    同:有“对各方面应付得很周到”的意思。

    异:多含贬义,偏重处事手腕圆滑;中性词,偏重应付得十分周到。

  \item 半斤八两~势均力敌

    同:彼此一样,不分上下。

    异:强调水平相等,多含贬义;偏重力量相等。

  \item 抱残守缺~敝帚自珍

    同:有“守着旧东西”的意思。

    异:贬义词,形容思想守旧,不肯接受新鲜事物;褒义词,谦辞,比喻自己的东西虽
    不好,可是自己珍视。

  \item 本末倒置~舍本逐末

    同:有“主次关系处理不当”的意思。

    异:强调把主次关系颠倒了;偏重舍弃主要的,追求次要的。

  \item 别具一格~别开生面

    同:给人以新的印象、新的感觉。

    异:偏重“格”,表示风格、样子与众不同,多用于文学创作和某些事物;偏重“生面”,
    表示新的局面或形式。

  \item 病入膏肓~不可救药

    同:表示病情严重,无法医治。

    异:偏重“病重”,病情严重到了无法医治的地步。比喻事情严重到了不可挽救的程度。
    偏重“救药”,强调无药可救。比喻人或事物坏到无法挽救的地步。

  \item 博闻强识~见多识广

    同:有“见识广”的意思。

    异:偏重见闻广博、知识面宽、记忆力强,只用于书面语;偏重阅历深,经验丰富,
    多用于口语。

  \item 捕风捉影~无中生有

    同:有“凭空捏造”的意思。

    异:偏重没有事实依据;偏重本来没有,语气较重。

  \item 不刊之论~不易之论

    同:有“不能改变”的意思。

    异:强调不可磨灭,不可更改;偏重论断正确,不可改变。

  \item 不堪设想~不可思议

    同:指不能想象。

    异:适用于严重的、不良的后果;一般适用于奇妙的、深奥的、不可理解的事情或道
    理。

  \item 不求甚解~囫囵吞枣

    同:有掌握知识不透彻,或对情况不够了解的意思。

    异:表示想懂个大概,不求彻底了解,偏重在态度上,是中性词;多指在学术上食而
    不化,不加分析思考地笼统接受,偏重在方法上,是贬义词。

  \item 不闻不问~漠不关心

    同:有“冷漠、不关心”的意思。

    异:偏重行动;偏重态度。

  \item 陈词滥调~老生常谈

    同:指讲惯听厌了的。

    异:谈的内容既陈旧又空泛(滥:空泛,不合实际),含贬义;谈的虽然是老话,但
    不一定没有现实意义,属中性词。

  \item 出尔反尔~反复无常

    同:经常变卦。

    异:偏重语言上前后矛盾;偏重表现上变化无常。

  \item 出神入化~炉火纯青

    同:指达到的境界很高。

    异:形容技艺高超、神妙;还可用于学术、修养方面。

  \item 处心积虑~殚精竭虑

    同:有“费尽心思”的意思。

    异:强调蓄谋已久,含贬义;强调用尽精力,费尽心思,偏褒义。

  \item 唇齿相依~唇亡齿寒

    同:比喻关系密切,互相依存。

    异:强调相互依存;强调利害相关,一方遭难,另一方也跟着遭难。

  \item 大庭广众~众目睽睽

    同:表示有许多人的场合。

    异:指聚集了很多人的公开场合;指很多人注目的场合(睽睽:睁大眼睛注视的样
    子)。

  \item 大张旗鼓~雷厉风行

    同:有“公开做事,声势浩大”的意思。

    异:强调声势和规模很大;形容执行政策法令等严格而迅速,也强调声势大而行动快。

  \item 顶礼膜拜~五体投地

    同:表示崇拜之意。

    异:偏重崇拜;偏重敬佩。

  \item 咄咄逼人~盛气凌人

    同:形容气势汹汹,使人难堪。

    异:应用范围广,不限于人,还可用于气势、形势、命令等;只用于人,并含有傲慢
    自大的意思。

  \item 阿谀奉承~趋炎附势

    同:比喻奉承、依附有权势的人。

    异:偏重“阿谀”,用好听的话讨好人;偏重“趋炎”,迎合权势(炎、势:指权势)。

  \item 耳濡目染~潜移默化

    同:有“不知不觉受到影响”的意思。

    异:所说的对象只是耳朵听到的、眼睛看到的;主要是思想或性格方面起了变化。

  \item 防患未然~未雨绸缪

    同:表示事前做好准备。

    异:偏重预防;偏重准备。

  \item 匪夷所思~不可思议

    同:有“不可理解”的意思。

    异:指言谈行动超出常情,不是一般人所能想象的;指难以想象,不能理解。

  \item 风言风语~流言蜚语

    同:表示没有根据的话。

    异:多指无意传说,传说者多出于无知、怀疑和猜测;多指有意传说,传说者往往出
    于险恶用心。

  \item 锋芒毕露~崭露头角

    同:有“才能显露出来”的意思。

    异:指锐气和才干全部表现出来,还可比喻骄傲自负;比喻突出地显露出才能和本领,
    不含骄傲自负之意。

  \item 浮光掠影~走马观花

    同:表示印象不深。

    异:比喻事物留下的印象不深;多指观察事物不仔细。

  \item 改邪归正~弃暗投明

    同:从坏的方面转向好的方面。

    异:偏重不再做坏事;偏重在政治上脱离黑暗势力,投向进步势力。

  \item 苟且偷生~得过且过

    同:形容只图眼前,不顾将来。

    异:偏重贪图眼前的安逸;偏重胸无大志,工作马虎,不负责任。

  \item 狗尾续貂~画蛇添足

    同:有“所做事情不当”的意思。

    异:指拿不好的东西续在好的东西后面,显得好坏不相称,多指文学作品;比喻做多
    余的事,反而不恰当。

  \item 孤注一掷~破釜沉舟

    同:有“最后拼一下以求胜利”的意思。

    异:偏重尽所有力量作最后一次冒险;偏重下决心决一胜负,含褒义。

  \item 故步自封~墨守成规

    同:因循守旧,不求进步或革新,都含贬义。

    异:偏重不求上进;偏重固执守旧,不肯改进。

  \item 光明磊落~光明正大

    同:心地光明的意思,用于人及其言行。

    异:偏重人的精神品质,指襟怀坦荡,没有私心;指人的行为正当、正派。

  \item 骇人听闻~耸人听闻

    同:有“使人吃惊”的意思。

    异:结果使人吃惊害怕;指故意说夸大或惊奇的话,使人震惊。

  \item 含糊其词~闪烁其词

    同:有“说话不清楚、不明确”的意思。

    异:偏重说得含糊不清;偏重说话故意遮遮掩掩、躲躲闪闪,不肯说出事情的真相和
    要害。

  \item 花天酒地~醉生梦死

    同:形容腐朽糜烂的享乐生活。

    异:偏重迷恋酒色;偏重浑浑噩噩、糊里糊涂地生活。

  \item 画饼充饥~望梅止渴

    同:比喻用空想来安慰自己,常可以通用。

    异:有“画饼”的行动;只表示“空等”,“空望”。

  \item 涣然冰释~烟消云散

    同:有“消失”的意思。

    异:指消除嫌疑或误解;指消除情绪或思想。

  \item 挥金如土~一掷千金

    同:形容极度挥霍。

    异:偏重对钱财的轻视;偏重一次花钱之多。

  \item 疾恶如仇~深恶痛绝

    同:有“厌恶、憎恨”的意思。

    异:如仇,如同仇敌;痛恨坏人坏事像痛恨仇敌一样;厌恶、憎恨到了极点,语意较
    重。

  \item 见利忘义~利令智昏

    同:表示为私利而做坏事。

    异:忘义,不顾道义;智昏,头脑发昏。

  \item 洁白无瑕~完美无缺

    同:指没有一丁点缺点和错误。

    异:适用于人和物,不适用于事情;多指人和事。

  \item 洁身自好~明哲保身

    同:指怕招惹是非。

    异:用作褒义时,偏重不与世俗同流合污;用作贬义时,多指怕惹是非,只顾自己好,
    不关心公众事情。用作褒义时,偏重处世待人十分明智;用作贬义时,多指怕犯错误
    或怕得罪人。

  \item 空前绝后~凤毛麟角

    同:有“稀少、少有”的意思。

    异:指以前没有过,以后也不会有;比喻稀少且可贵的人或事物。

  \item 口蜜腹剑~笑里藏刀

    同:形容阴险狡诈。

    异:偏重嘴甜;偏重外表和气。

  \item 历历在目~记忆犹新

    同:表示清楚地记得往事。

    异:偏重过去情景的再现;偏重记忆像新的一样。

  \item 恋恋不舍~流连忘返

    同:有“舍不得离开”的意思。

    异:语意范围广,指对一切人、景物等的留恋;偏重对景物的留恋。

  \item 两全其美~一举两得

    同:做一件事情在两方面有好处。

    异:指做一件事情,顾及两个方面,使两方面都很好(做事情之前就考虑到了);指
    做一件事情,得到两种收获(注意说的是“收获”)。

  \item 另眼相看~刮目相看

    同:都指在看待对方上发生了变化。

    异:强调用另一种眼光,表示特别重视;强调用新的眼光来看待,表示跟过去不一样。

  \item 六神无主~心惊肉跳

    同:形容惊惧不安。

    异:偏重心情慌乱,不知怎么办才好;偏重心神不宁、不安,害怕不好的事临头。

  \item 美不胜收~琳琅满目

    同:形容美好的事物很多。

    异:偏重来不及看,来不及一一欣赏;偏重满眼都是。

  \item 莫衷一是~无所适从

    同:有“不知道”的意思。

    异:指不能判断哪个是对的,形容意见有分歧,不能得出一致的结论;指不知道跟从
    谁好,形容不知怎么办才好。

  \item 秣马厉兵~严阵以待

    同:有“做好战斗准备”的意思。

    异:偏重人员的行动;偏重整个军队排好阵势,等待敌人的来临。

  \item 目光如豆~鼠目寸光

    同:比喻目光短浅,看不到远处、大处。

    异:偏重眼光小,强调看不到全局;偏重眼光近,看不到将来。

  \item 恰到好处~恰如其分

    同:表示说话、做事达到适当的程度,可通用。

    异:偏重恰巧达到最好的地步;偏重正合分寸。

  \item 轻车熟路~得心应手

    同:有“做起来容易”的意思。

    异:侧重对情况熟悉;侧重行动上运用自如。

  \item 丧尽天良~丧心病狂

    同:有“心肠坏、做事凶狠”的意思。

    异:偏重心肠坏;偏重言行荒谬、凶狠残忍。

  \item 殊途同归~异曲同工

    同:有“用不同的方法,得到同样的结果”的意思。

    异:通过不同的道路,走到同一个目的地,比喻采取不同的方法而得到相同的结果,
    不强调结果的好坏;不同的曲调演得同样好,比喻不同的人的辞章或言论同样精彩,
    或者不同的做法收到同样好的效果,偏重于收到好的效果。

  \item 谈笑风生~谈笑自若

    同:有“谈话时有说有笑”的意思。

    异:强调谈话时兴致勃勃,气氛活跃,多指平时说话;强调不变常态,多用于紧张、
    情势严重时的讲话。

  \item 外强中干~色厉内荏

    同:有“表面看起来厉害,内在其实不是这样”的意思,贬义词。

    异:指外表看起来很强大,实际上很空虚,可以形容人的体质、经济能力或国家实力
    等,适用范围较广;指外表强硬而内心怯懦,多用于形容人的态度。

  \item 妄自尊大~夜郎自大

    同:有“骄傲自大”的意思。

    异:偏重狂妄;偏重无知。

  \item 望风而逃~闻风丧胆

    同:有“听到一点风声就害怕”的意思。

    异:偏重吓得连忙逃跑;偏重吓破了胆,丧失了勇气。

  \item 无微不至~无所不至

    同:有“没有一处不到”的意思。

    异:形容待人处世细致周到,体贴入微,含褒义;除了“没有达不到的地方”之外,还
    可指什么事都干得出来,多含贬义。

  \item 瑕不掩瑜~瑕瑜互见

    同:表示同时具有优点和缺点。

    异:表示缺点遮不住优点;表示有优点也有缺点,没有主次之分。

  \item 休戚与共~休戚相关

    同:有“利害一致”的意思。

    异:形容彼此共同承受幸福与灾祸,有同甘共苦的意思;形容彼此间祸福相互关联,
    关系密切,但无“同甘共苦”的意思。

  \item 徇私舞弊~营私舞弊

    同:指为私利而玩弄手段,干违法乱纪的事。

    异:指为了私情,照顾私人关系而舞弊;指为自己谋求私利而舞弊。

  \item 洋洋洒洒~纷纷扬扬

    同:形容多而连续不断的状态。

    异:多形容文章或谈话内容丰富,连续不断;一般用来形容雪、花、叶等具体事物飘
    洒的样子。

  \item 咬文嚼字~字斟句酌

    同:有“重视字句”的意思。

    异:形容过分地斟酌字句,多指死抠字眼儿而不注重实质内容,略带贬义;指对字句
    反复推敲琢磨,多用于称赞人讲话或写文章时在语言上认真推敲,也可以用于形容读
    文章时对语言的仔细品味。

  \item 一笔勾销~一笔抹杀

    同:有“全部销去”的意思。

    异:勾销,全部取消,不再计较;抹杀,涂抹掉,表示全盘否定,多指优点、成绩等
    被轻率地全盘否定。

  \item 义不容辞~责无旁贷

    同:有“应当承担、不能推辞”的意思。

    异:偏重道义上不允许推辞;偏重责任上不可推卸(贷:推卸)。

  \item 置若罔闻~熟视无睹

    同:表示“不重视,不关心”。

    异:是说放在一边不管,好像没听见一样,不重视,不在乎;是说虽然经常看见,但
    跟没看见一样,强调的是看,对于应关心的事物漠不关心。

  \item 跋山涉水~风尘仆仆~风餐露宿

    同:有“旅途辛苦”的意思。

    异:重在强调远行艰辛;重在强调长途奔波劳累;重在强调野外食宿艰难。

  \item 趁火打劫~浑水摸鱼~顺手牵羊

    同:有“趁机拿走东西”的意思。

    异:强调趁紧张危急的时候侵犯别人的权益;强调趁混乱的时机捞取利益;强调顺便
    拿走人家的东西。

  \item 承前启后~承上启下~继往开来

    同:有“接上启下”的意思。

    异:继承前代的并启发后代的,多用于学问、事业等方面;接续上面的并引起下面的,
    多用于写作等方面;多用于继承事业,开辟道路。

  \item 洞若观火~了如指掌~明察秋毫

    同:有“对事物非常清楚”的意思。

    异:强调观察认识事物透彻、深刻;强调对情况非常清楚;强调对事物观察敏锐、细
    致,任何小问题都看得很清楚。

  \item 独断专行~专横跋扈~一意孤行

    同:有“不考虑别人意见,办事主观蛮干”的意思。

    异:强调专断、霸道,语意较重;现多强调任意妄为,不讲理;不听劝告,固执地照
    自己的意思行事,语意较轻。

  \item 耳濡目染~耳闻目睹~耳熟能详

    同:有“耳朵听到”的意思。

    异:形容听得多、见得多了之后,无形之中受到影响,强调听、见的结果;强调亲耳
    听见,亲眼看见;听的次数多了,也就能详尽地说出来,只有听,没有看。

  \item 功亏一篑~功败垂成~前功尽弃

    同:有“没有成功”的意思。

    异:比喻一件大事只差最后一点人力物力而不能成功,含有惋惜之意;强调快要成功
    的时候遭到失败,含惋惜之意;强调以前的努力完全白费。

  \item 钩心斗角~明争暗斗~尔虞我诈

    同:有“争斗”的意思。

    异:强调各用心机,互相排挤;强调明里暗里都在进行争斗;强调彼此猜疑,互相欺
    骗。

  \item 绘声绘色~惟妙惟肖~栩栩如生

    同:有“生动逼真,很像”的意思。

    异:强调叙述、描写生动逼真;强调形似,模仿得像真的一样;强调艺术形象非常逼
    真,如同活的一样。

  \item 匠心独运~巧夺天工~鬼斧神工

    同:有“技艺高超、精巧”的意思。

    异:强调在文学、艺术等方面独创性地运用巧妙的心思;强调精巧的人工胜过天然;
    像是鬼神制作出来的。形容自然造化的神奇,也可形容建筑、雕塑等技艺的精巧高超。

  \item 牢不可破~颠扑不破~坚不可摧

    同:有“牢固不可摧毁”的意思。

    异:强调坚固得不可摧毁,多指抽象事物,如友谊;强调永远不会被推翻,多指理论、
    道理等;强调非常坚固,摧毁不了,多用于意志、精神等。

  \item 呕心沥血~处心积虑~挖空心思

    同:有“用尽心思”的意思。

    异:为褒义词,形容费尽心血;为贬义词,千方百计地盘算;多含贬义,形容费尽心
    计。

  \item 锲而不舍~旷日持久~持之以恒

    同:有“用时很长”的意思。

    异:强调做事情能坚持到底,也形容有恒心,有毅力;强调多费时日,拖得很久;强
    调长久地坚持下去。

  \item 煞费苦心~惨淡经营~殚精竭虑

    同:有“费尽心思做事”的意思。

    异:强调费尽心思,带有目的性;原意指苦心构思,后来强调在困境中艰苦地从事某
    种事业;强调用尽精力,指一种状态。

  \item 拭目以待~指日可待~倚马可待

    同:有“可以等待”的意思。

    异:形容殷切期望或密切关注事态的动向及结果;强调为期不远,(事情、希望等)
    不久就可以实现;形容文思敏捷,文章写得快。

  \item 深谋远虑~深思熟虑~老谋深算

    同:有“深入、周密考虑”的意思。

    异:强调周密地计划,往长远里考虑;强调深入细致地考虑;多形容人办事精明老练。

  \item 束手无策~无能为力~手足无措

    同:有“没有办法”的意思。

    异:强调没有任何解决的办法;强调没有能力或能力达不到;强调举动慌乱或没有办
    法应付。

  \item 望洋兴叹~望其项背~望尘莫及

    同:有“望到”的意思。

    异:强调因力不胜任或没有条件而感到无可奈何;赶得上或比得上,多用于否定式;
    强调远远落在后面。

  \item 一诺千金~一言九鼎~一字千金

    同:有“话少但分量重、价值高”的意思。

    异:强调说话算数,所许诺言信实可靠;强调所说的话分量很重,作用很大;用来称
    赞诗文精妙,价值极高,也指书法作品的珍贵。

  \item 一视同仁~等量齐观~相提并论

    同:有“同样看待”的意思。

    异:意为同样看待,不分亲疏厚薄,多指人;意为不管事物间的差异,同等看待,多
    指物;意为把不同的人或事物混在一起来谈论或看待。

  \item 以身作则~身体力行~身先士卒

    同:有“亲自去做”的意思。

    异:偏重用自己的行动做榜样;偏重亲自体验,努力实行;偏重职位高的人首先走在
    前面,带头去做。

  \item 犹豫不决~举棋不定~畏首畏尾

    同:有“迟疑、犹豫,拿不定主意”的意思。

    异:强调下不了决心;比喻临事拿不定主意;形容怕这怕那,疑虑过多。

  \item 作茧自缚~自食其果~玩火自焚

    同:有“做了某事结果使自己受害”的意思。

    异:比喻做了某事,结果反而使自己受困;指做了坏事,结果害了自己,自作自受;
    比喻干冒险或害人的勾当,最终受害的还是自己,语意较重。

\end{enumerate}

\section{句法}

\subsection{单句}

\subsubsection{单句的概念}

\index{单句}
根据内部结构的不同,句子可分为单句和复句。其中,单句是由短语或词充当的、有特定
的语调、能独立表达一个相对完整的意思的语言单位。单句可以根据不同的标准来划句型
和句类。

和复句相比,单句一般只有一套句子成分,往往只有一个主谓结构。例如:
\begin{equotation}
  这对于一班见异思迁的人,对于一班鄙薄技术工作以为不足道、以为无出路的人,也是一
  个极好的教训。
\end{equotation}
这是单句,句子主干是:这是教训。
\begin{equotation}
  这种桥不但形式优美,而且结构坚固。
\end{equotation}
这是复句,表递进。

\subsubsection{单句的成分}

\index{单句的成分}
单句的成分有六种,包括主干成分主语、谓语、宾语和枝叶成分定语、状语、补语。

为了便于划分,它们有一套约定俗成的符号:

\begin{enumerate}
  \item \uuline{主语},双行线。

  \item \uline{谓语},单行线,主语和谓语之间用双竖线‖隔开。

  \item \uwave{宾语},波浪线。

  \item 定语,小括号(~)。

  \item 状语,中括号[~]。

  \item 补语,单书名号〈~〉。
\end{enumerate}

在一个完整的、典型的句子中,各种成分之间的关系、顺序一般是:

\begin{table*}[htbp]
  \centering
  \begin{tabular}[c]{@{ }c@{ }c@{ }c@{ }c@{ }c@{ }c@{ }c@{ }c@{ }}
    (多愁善感的)& \uuline{邱天同学} & [居然] & \uline{喜欢} & 〈上了〉 & (豪放派诗人苏轼的)& \uwave{词章} \\
    \small 定语 &
    \small 主语 &
    \small 状语 &
    \small 谓语 &
    \small 补语 &
    \small 定语 &
    \small 宾语
  \end{tabular}
\end{table*}

\begin{description}
  \index{主语}
  \item[主语] 主语是句子中的陈述对象,说明是谁或什么,经常由名词、代词、数词、
    名词性短语充当。例如:
    \begin{equotation}
      \uuline{他}肥胖的身子向左微倾。(朱自清《背影》)
    \end{equotation}

    \index{谓语}
  \item[谓语] 用来说明陈述主语,经常由动词、形容词充当,一般表示主语“怎么样”或
    “是什么”。例如:
    \begin{equotation}
      他又蹬蹬蹬地自个向前\uline{走}了。(茹志鹃《百合花》)
    \end{equotation}

    \index{宾语}
  \item[宾语] 表示谓语动词的涉及对象的语言单位,经常由名词、代词、名词性短语充
    当,一般表示谓语“怎么样”或“是什么”。能愿动词,如“希望、想、可以”等词后面的
    一般都作宾语处理。例如:
    \begin{equotation}
      野菱角开着四瓣的\uwave{小白花}。(汪曾祺《受戒》)
    \end{equotation}

    \index{定语}
  \item[定语] 用在主语和宾语前面,起修饰和限制作用的语言单位。经常由名词、形容
    词、动词、代词充当,一般定语与中心词之间有“的”字连接。例如:
    \begin{equotation}
      (松散)、(柔软)的荒草抚弄着(她)的裤脚。(铁凝《哦,香雪》)
    \end{equotation}

    \index{状语}
  \item[状语] 一般用在动词、形容词谓语前,起修饰和限制作用的语言单位,经常由副
    词、形容词、动词、表示处所及时间的名词和方位词充当,一般状语与中心词之间有
    “地”字连接。例如:
    \begin{equotation}
      她[分明][已经][纯乎]是一个乞丐了。(鲁迅《祝福》)

      他一个人[呆呆]地坐在禾场边上。(路遥《平凡的世界》)
    \end{equotation}

    \index{补语}
  \item[补语] 谓语后面的附加成分,对谓语起补充说明作用,回答“怎么样”,“多久”,
    “多少”(时间、处所、结果)之类问题的语言单位,经常由动词、形容词、副词充当,
    一般补语与中心词之间有“得”字连接。例如:
    \begin{equotation}
      在这讽刺般的笑声中,我头一次感到自己傻得〈可怜〉。
    \end{equotation}

\end{description}

\subsubsection{六种句子成分位置的简单口诀}

\begin{center}
  主谓宾、定状补,主干枝叶分清楚;

  基本成分主谓宾,附加成分定状补;

  定语必居主宾前,谓前为状谓后补;

  “的”定“地”状“得”后补,结构助词分清楚。
\end{center}

\subsubsection{陈述句、疑问句、祈使句、感叹句}

句子根据语气可以分为四种类型,即陈述句、疑问句、祈使句和感叹句。

\begin{description}
  \index{陈述句}
  \item[陈述句] 叙述或说明事实、带有陈述语气的句子叫陈述句。它可带的语气词有
    “了、的、嘛、呢、罢了、啊”等。有肯定和否定两种形式。例如:
    \begin{equotation}
      他会去的。

      他不会去的。
    \end{equotation}

    \index{疑问句}
  \item[疑问句] 提出问题、具有疑问语气的句子叫疑问句,句末用问号。疑问句根据
    提问的手段和语义情况,可以分为四类:是非问、特指问、选择问、正反问。

    \begin{enumerate}
      \item 是非问。只对整个命题作肯定或否定回答。例如:
        \begin{equotation}
          中国最长的河流是长江吗?
        \end{equotation}

      \item 特指问。用疑问代词(如“谁、什么、怎样”等)或由它组成的短语(如“为什
        么、什么事、做什么”等)来表明疑问点。例如:
        \begin{equotation}
          孔门四教具体是指什么?
        \end{equotation}

      \item 选择问。提出不止一种看法供对方选择。例如:
        \begin{equotation}
          您是喝杜松子酒,还是威士忌酒,还是啤酒?
        \end{equotation}

      \item 正反问。提出正反两项,希望对方从中选择一项回答,也可以说是一种特殊
        的选择问。例如:
        \begin{equotation}
          你是不是也想去呢?
        \end{equotation}

    \end{enumerate}

    \index{祈使句}
  \item[祈使句] 要求对方做或不要做某事、具有祈使语气的句子叫祈使句。它可分为
    两大类:一类是命令、禁止,一类是请求、劝阻。例如:
    \begin{equotation}
      不得随地吐痰。

      您还是请进里面休息一下吧。
    \end{equotation}

    \index{感叹句}
  \item[感叹句] 带有浓厚的感情,具有感叹语气的句子叫感叹句。例如:
    \begin{equotation}
      美丽的草原真让人陶醉!
    \end{equotation}

\end{description}

\subsubsection{连谓句、兼语句}

\begin{description}
  \index{连谓句}
  \item[连谓句] 连谓短语充当谓语或独立成句的句子叫连谓句。连谓句中,往往两个谓
    词性成分相连,并且是同一个主语。例如:
    \begin{equotation}
      余德利抬头发现李冬宝的目光很慌乱。

      大家热烈鼓掌表示祝贺。

      老爷爷拄着拐棍过马路。

      这件事想起来心烦。
    \end{equotation}

    \index{兼语句}
  \item[兼语句] 兼语短语充当谓语或独立成句的句子叫兼语句。通俗地说,一个结构里
    边有两个动词,第一个动词后边的宾语同时又是第二个动词的主语,这个同时有两种
    作用的成分叫兼语,这样的句子叫兼语句。例如:
    \begin{equotation}
      地震和海啸令这个国家的经济倒退了十年。(使令式)

      他埋怨我没给他办成这件事。(爱恨式)

      村民选他当村长。(选定式)

      小河岸上的煤屑路上有人在走。(“有”字式)
    \end{equotation}

\end{description}

\subsection{复句}

\subsubsection{复句的概念}

\index{复句}
复句是由两个或两个以上意义上相关、结构上互不作句法成分的分句加上贯通全句的句调
构成的。复句前后有隔离性停顿,书面用句号、问号或叹号表示。复句的各分句间一般有
句中停顿,书面上用逗号、分号或冒号表示。

\subsubsection{复句类型}

如表~\ref{tab:复句类型分类}。根据分句间的意义关系划分,复句可以分为联合复句、偏
正复句两大类。复句内各分句间意义上平等、无主从之分的叫联合复句;复句内各分句间
意义上有主有从,也就是有正句、偏句之分的叫偏正复句,又叫主从复句。正句即主句,
是句子的正意所在;偏句是从句,意义从属于正句。

\setlength{\bigstrutjot}{1.25em}
\begin{longtable}{|p{1em}|c|c|c|p{8em}|}
  \caption{复句类型分类表}\label{tab:复句类型分类}\\
  \hline \multicolumn{2}{|c|}{分类} & 分句间关系 & \multicolumn{1}{c|}{主要关联词语} & \multicolumn{1}{c|}{例句} \\ \hline
  \endfirsthead
  %
  \multicolumn{5}{r}{(续上表)} \\
  \hline \multicolumn{2}{|c|}{分类} & 分句间关系 & \multicolumn{1}{c|}{主要关联词语} & \multicolumn{1}{c|}{例句} \\ \hline
  \endhead
  %
  \hline \multicolumn{5}{r}{接下页\dots} \\
  \endfoot
  %
  \hline
  \endlastfoot
  %
  \index{复句!联合复句} \index{复句!并列复句}
  \multirow{24}={联合复句} & \multirow{2}[tb4]*{并列复句} & \multirow{2}[tb4]*{平列、对举} & \multirow{1}[tb3]*{既……又……} & \multirow{2}[tb4]={绿既是美的标志,又是科学、富足的标志。} \\* \cline{4-4}
                           & & & \makecell{一方面……\\一方面……} & \\* \cline{2-5}
                           \index{复句!顺承复句}
                           & \multirow{2}[tb4]*{顺承复句}& \multirow{2}[tb4]*{先后相承} & 首先……然后…… \bigstrut & \multirow{2}[tb4]={只有几个赤膊的人翻,翻了一阵,都进去了,接着走出一个小旦来,咿咿呀呀的唱。}  \\*\cline{4-4}
                           & & & 便、就、又、再 \bigstrut & \\*\cline{2-5}
                           \index{复句!解说复句}
                           & \multirow{9}*{解说复句} & \multirow{9}*{解释、总分} & \multirow{4}*{一般不用关联词语} & 对自己,“学而不厌”,对人家,“诲人不倦”,我们应取这种态度。\\* \cline{4-5}
                           & & & \multirow{5}*{即、就是说} & 我们正处于这样一个历史阶段,即我们相信两国的关系能提高到一个新的高度。 \\*\cline{2-5}
                           & & & 或者……或者…… \rule[-0.7em]{0pt}{2.1em} & \multirow{3}[tb3]={在他们的内心深处,与其说盼望着回家,毋宁说更害怕回家。} \\* \cline{4-4}
                           \index{复句!解说复句}
                           & 选择复句 & 选择 & 不是……就是……\rule[-0.7em]{0pt}{2.1em}& \\* \cline{4-4}
                           & & & 与其……毋宁……\rule[-0.7em]{0pt}{2.1em}& \\* \cline{2-5}
                           \index{复句!递进复句}
                           & \multirow{4}*{递进复句} & \multirow{4}*{意思更进一层} & \multirow{4}*{不是……就是……} & 这十多个少年,委实没有一个不会凫水的,而且两三个还是弄潮的好手。 \\
                           \index{复句!偏正复句}
                           \index{复句!条件复句}
  \multirow{17}={偏正复句} & \multirow{2}[tb4]*{条件复句} & \multirow{2}[tb4]*{条件和结果} & 只要……就…… \bigstrut& \multirow{2}[tb4]={平凡的工作只要和远大的理想结合起来,便会产生极大的乐趣。} \\*\cline{4-4}
                           & & &只有……才……\bigstrut& \\*\cline{2-5}
                           \index{复句!假设复句}
                           & \multirow{2}[tb3]*{假设复句} & \multirow{2}[tb3]*{假设和结果} & 如果……就…… \rule[-1em]{0pt}{2.5em} & \multirow{2}[tb3]={我们若能这样追问,一切虚妄的学说便不攻自破了。} \\*\cline{4-4}
                           & & & 即使……也……\rule[-1em]{0pt}{2.5em} & \\*\cline{2-5}
                           \index{复句!因果复句}
                           & \multirow{2}[tb4]*{因果复句} & \multirow{2}[tb4]*{原因和结果} & 因为……所以……\bigstrut & \multirow{2}[tb4]={真正有理想的人,必定珍惜一分一秒,因为每一瞬间的奋斗都关系着目标的实现。} \\*\cline{4-4}
                           & & & 既然……那么……\bigstrut & \\*\cline{2-5}
                           \index{复句!目的复句}
                           & \multirow{3}*{目的复句} & \multirow{3}*{行为和目的} & \multirow{3}*{以、以便、为的是} & 我在这里吃雪,正是为了我们祖国的人民不吃雪。 \\*\cline{2-5}
                           \index{复句!转折复句}
                           & \multirow{4}*{转折复句} & \multirow{4}*{相反、相对} & \multirow{4}*{虽然……但是……} & 虽然我一见便知道是闰土,但又不是我这记忆上的闰土了。 \\
\end{longtable}

\subsection{常见病句类型举例}

\index{病句}
所谓病句,是指不合规范的句子。所谓规范,一是要符合语法组合规则,二是要符合语义
的搭配要求,三是要符合语用的表达习惯。寻找产生病句的原因,可能涉及语法、语义和
语用多个层面。

句子的常见语病类型主要有六种:语序不当、搭配不当、成分残缺或者赘余、结构混乱、
表意不明、不合逻辑。

\subsubsection{语序不当}

\begin{itemize}

  \item 定语和中心语错位。例如:
    \begin{equotation}
    近年来,随着教育教学改革的不断深化,高校学生的培养深受社会广大用人单位的欢
    迎,就业率明显提高。
    \end{equotation}
    深受用人单位欢迎的应该是“高校培养的学生”,而不是“高校学生的培养”。

  \item 定语、状语错位。例如:
    \begin{equotation}
      夜深人静,想起今天一连串发生的事情,我怎么也睡不着。
    \end{equotation}
    “一连串”应该移到“事情”之前,作它的定语。

  \item 多层定语语序错位。

    多项定语的排列顺序一般是:
    \begin{enumerate}
      \item 表示领属或时间、处所;
      \item 指称代词或数量词;
      \item 动词或动词性短语;
      \item 形容词或形容词性短语;
      \item 名词或名词性短语(带“的”的定语应放在不带“的”的定语之前)。
    \end{enumerate}

    \begin{equotation}
      她是一位优秀的有 20 多年教学经验的国家队的篮球女教练。
    \end{equotation}
    正确次序应是:她是国家队的(领属性的)一位(数量)有20多年教学经验的(动词
    短语)优秀的(形容)篮球(名词)女教练。

  \item 多层状语语序不当。

    复杂状语的排列顺序一般是:
    \begin{enumerate}
      \item 表目的或原因的介宾短语;
      \item 表时间的名词或介宾短语;
      \item 表处所的名词或介宾短语;
      \item 表范围、频率等的副词;
      \item 表情态的形容词;
      \item 表对象的介宾短语。
    \end{enumerate}

    \begin{equotation}
      在休息室里许多老师昨天都同他热情地交谈。
    \end{equotation}
    正确次序应是:许多老师昨天(时间)在休息室里(处所)都(范围)热情地(情态)
    同他(对象)交谈。

  \item 虚词位置不当。

    最常见的是关联词语语序不当。两个分句主语相同时,关联词语在主语之后;主语不
    同时,关联词语在主语之前。

    \begin{equotation}
      诚信教育已成为我国公民道德建设的重要内容,因为不仅诚信关系到国家的整体形
      象,而且体现了公民的基本道德素质。
    \end{equotation}
    后两个分句是同一个主语“诚信”,因此关联词语“不仅”放在“诚信”后。

  \item 关联词位置不当。

    \begin{equotation}
      由于技术水平太低,这些产品质量不是比沿海地区的同类产品低,就是成本比沿海
      的高。
    \end{equotation}
    两个分句陈述的对象分别为“这些产品质量”,“成本”,陈述的对象不同,也即主语
    不一致,那么关联词“不是”就应放在“质量”之前,这样语义才连贯。

  \item 并列词语排列顺序不当。

    并列词语的各项,要注意其轻重、先后、大小的关系,否则容易出现错误。
    \begin{equotation}
      一种观念只有被人们普遍接受、理解和掌握并转化为整个社会的群体意识,才能成
      为人们自觉遵守和奉行的准则。
    \end{equotation}

    按照认识事物的一般规律,应该是先理解,再接受,最后掌握。

\end{itemize}

\subsubsection{搭配不当}

\begin{itemize}
  \item 主语和谓语搭配不当。例如:

    \begin{equotation}
      虽然精彩绝伦的篮球总决赛已落下帷幕,但那跌宕起伏的过程、充满血性的身影、
      顽强拼搏的精神依然闪现在人们的脑海中,挥之不去。
    \end{equotation}

    此句中“那跌宕起伏的过程、充满血性的身影、顽强拼搏的精神”共用一个谓语动词“闪
    现”,但“跌宕起伏的过程”,“顽强拼搏的精神”与“闪现”不搭配。可以改成“但那跌宕
    起伏的过程依然印在人们的脑海里,充满血性的身影依然闪现在人们的脑海中,顽强
    拼搏的精神依然长存在人们的脑海中,挥之不去”。

  \item 谓语和宾语搭配不当。例如:

    \begin{equotation}
      经过几代航天人的艰苦奋斗,中国的航天事业开创了以“两弹一星”、载人航天、月
      球探测为代表的辉煌成就。
    \end{equotation}

    谓语“开创”与宾语“成就”搭配不当,应将“开创”改为“取得”。

  \item 定语、状语、补语与中心语搭配不当。例如:

    \begin{equotation}
      你知道每500克蜂蜜中包含蜜蜂的多少劳动吗?据科学家统计,蜜蜂每酿造500克蜜,
      大约要采集50万朵的花粉。
    \end{equotation}

    定语“50万朵”与中心词“花粉”搭配不当,可以改成“50万朵花的花粉”。

    \begin{equotation}
      同学们把教室打扫得干干净净,整整齐齐。
    \end{equotation}

    谓语“打扫”与补语“干干净净,整整齐齐”中的“整整齐齐”不搭配。可以改成“同学们把
    教室打扫得干干净净,把桌椅摆放得整整齐齐”。

  \item 主语和宾语不能搭配。例如:

    \begin{equotation}
      京剧是中国独有的表演艺术,它的审美情趣和艺术品位,是中国文化的形象代言之
      一,是世界艺术之林的奇葩。
    \end{equotation}

    “它的审美情趣和艺术品位”,“是”,“奇葩”,分句间的主宾搭配不当。可以改成“京剧
    是中国独有的表演艺术,极具审美情趣和艺术品位,是中国文化的形象代言之一,是
    世界艺术之林的奇葩”。

  \item 联合词组不能和某成分同时搭配。例如:

    \begin{equotation}
      他积极支持这一建议,并召开常委会进行研究,统一安排了现场会的内容、时间和
      参观人员,以及会议中应该注意的问题。
    \end{equotation}

    “会议中应该注意的问题”与“安排”不搭配,可以改为“……
    统一安排了现场会的内容、时间和参观人员,并且提出了会议中应该注意的问题”。

  \item 一面与两面搭配不当。例如:

    \begin{equotation}
      面对突然发生的灾难,一个地方抗灾能力的强弱既取决于当地经济实力的雄厚,更
      取决于政府的应急机制和领导人的智慧。
    \end{equotation}

    一面对两面。前面“一个地方抗灾能力的强弱”,说的是两面,后面说的是“当地经济实
    力的雄厚”,只是一面。可以改成“面对突然发生的灾难,一个地方抗灾能力的强弱既
    取决于当地经济实力的厚薄,更取决于政府的应急机制和领导人的智慧”。

  \item 关联词语搭配不当。例如:

    \begin{equotation}
      应用这种罗盘,无论在阴云密布以及早晚看不到太阳的时候,也不会迷失方向。
    \end{equotation}

    “无论”与“也”不搭配,应将“无论”改为“即使”。

\end{itemize}

\subsubsection{成分残缺或赘余}

\paragraph{【成分残缺】}

\index{成分残缺}
\begin{enumerate}
  \item 主语残缺。由于暗换主语或者滥用介词和“介词…… 方位词”的格式造成主语残缺。
    例如:

    \begin{equotation}
      由于文人士大夫参与到印章的创作中,使这门从前主要由工匠传承的技艺,增加了
      人文意味。
    \end{equotation}

    “由于…… 中,使”造成主语缺失。可以把“由于”去掉。

  \item 谓语残缺。例如:

    \begin{equotation}
      近年来,一批精良艺术品质和积极价值取向的文艺作品受到观众广泛认可,这充分
      证明过硬品质是新时代文艺实现文化引领的基本条件。
    \end{equotation}

    谓语残缺,“精良艺术品质”前应加“具有”。

  \item 宾语残缺。例如:

    \begin{equotation}
      目前,无论是国内建筑界,还是一般的知识阶层和社会大众,人们对建筑批评似乎
      还是保持着一种漠然,建筑艺术远未如其他艺术那样,形成活跃而健康的批评氛围。
    \end{equotation}

    “保持”没有和它搭配的宾语,在“漠然”后加“的态度”。

  \item 附加成分残缺。例如:

    \begin{equotation}
      科学工作者需要开阔的心胸,就是和自己学术观点不一样的同行也应坦诚相待,精
      诚合作。
    \end{equotation}

    成分残缺,在“和自己”前应该加介词“对”。
\end{enumerate}

\paragraph{【成分赘余】}

\index{成分赘余}
\begin{itemize}
  \item 语意重复造成主语、谓语、宾语等成分重复。例如:

    \begin{equotation}
      荞麦具有降低毛细血管脆性、改善微循环、增加免疫力的作用,可用于高血压、高
      血脂、冠心病、中风发作等疾病的辅助治疗。
    \end{equotation}

    “高血压…… 中风”这些疾病,如有症状,就是“发作”了,所以不必再有“发作”,谓语赘
    余。

  \item 否定词赘余。例如:

    \begin{equotation}
      艾滋病有性传播、血液传播、母婴传播等三大传播途径,我们需要采取紧急行动制
      止它的传播,否则不采取紧急行动,将会迅速蔓延,给人类健康带来巨大的威胁。
    \end{equotation}

    “否则”意为“如果不是这样”,此句中,“否则”就是“不采取紧急措施”的意思,应删去
    “不采取紧急措施”。

  \item 关联词赘余。例如:

    \begin{equotation}
      我这次考不好的原因是因为我没有按老师要求仔细审题。
    \end{equotation}

    去掉“原因”或者“因为我”。

  \item 介词赘余。例如:

    \begin{equotation}
      娱乐明星们不雅的形象常常见诸于荧屏,这对于喜爱模仿的少年儿童是有负面影响
      的。
    \end{equotation}

    “诸”解释为“之于”,因此句中的“于”字应去掉。

  \item 约数赘余。例如:

    \begin{equotation}
      中国目前拥有网络文学写作者超过 200 多万人,每年有六七万部左右的作品被签约;
      全国网络文学用户达 1.94 亿,超过了电子商务用户。
    \end{equotation}

    “超过”与“多”语意重复,应删去其一;“左右”与“六七万部”都表示约数,应删去“左
    右”。
\end{itemize}

\subsubsection{结构混乱}

\begin{description}

  \item[句式杂糅] 同一内容,往往可以采取不同的说法。如果说话、写作时拿不定主意,
    既想用这种说法,又想用那种说法,结果两种说法都用上了,糅到一起,形成两句混
    杂。

    \begin{equotation}
      处理好人与自然的关系,要靠政府的力量,同时也不能不发挥民间力量在舆论动员,
      监督检查等方面起到无可替代的作用。
    \end{equotation}

    “发挥民间力量”与“民间力量……起到无可替代的作用”杂糅。可改为“同时也不能不发挥
    民间力量在舆论动员、监督检查等方面的无可替代的作用”或“同时民间力量在舆论动
    员、监督检查等方面也能发挥无可替代的作用”。

    当上级宣布我们摄制组成立并交给我们任务的时候,我们大家有既光荣又愉快的感觉
    是颇难形容的。

    “既光荣又愉快的感觉”是前一分句“有”的宾语,又是后一分句的主语,牵连在一起,
    形成杂糅。可以在“感觉”后加一个逗号,再在“是颇难形容的”前加上“这种感觉”四个
    字。

  \item[中途易辙] 意思是一件事还没做完就去做另一件事。“中途易辙”病句其特点往往
    是一句话说了一半,忽然另起炉灶,又去说另外一句。偷换主语等有可能造成“中途易
    辙”,但是“中途易辙”和“偷换主语”的病句有相同之处,又有不同。“中途易辙”主要表
    现为主谓宾等结构上逻辑关系,“偷换主语”是前后主语不一致。例如:

    \begin{equotation}
      冲突双方在民族仇恨的驱使下,虽然经过国际社会多次调解,紧张的局势不但没有
      得到缓和,反而愈演愈烈。
    \end{equotation}

    主语“冲突双方”没有陈述完整,主语又改换成“紧张的局势……”。可以改成“在民族仇恨
    的驱使下,冲突双方虽然经过国际社会多次调解,紧张的局势不但没有得到缓和,反
    而愈演愈烈”。

    \begin{equotation}
      屠呦呦先生经过40多年的努力,青蒿素的研发、推广和应用终于走出了一条独具中
      国特色的产业化道路,为人类的生命健康事业作出了杰出贡献。
    \end{equotation}

    全句可改为“屠呦呦先生经过40多年的努力,终于让青蒿素的研发、推广和应用走出了
    一条独具中国特色的产业化道路,为人类的生命健康事业作出了杰出贡献”。

\end{description}

\subsubsection{表意不明}

\begin{enumerate}
  \item 指代不明。例如:

    \begin{equotation}
      曾记否,我与你认识的时候,还是个十来岁的少年,纯真无瑕,充满幻想。
    \end{equotation}

    “还是个十来岁的少年”指的是“我”,“你”还是“我们”,指代不清。

  \item 对象不明。例如:

    \begin{equotation}
      祁爱群看到组织部里新来的援藏干部很高兴,就亲切地同他交谈起来。
    \end{equotation}

    是“祁爱群”,“很高兴”,还是“援藏干部”,“很高兴”,不明确。

  \item 词类不同造成歧义。例如:

    \begin{equotation}
      他背着总经理和副总经理偷偷把钱分别存进了两家银行。
    \end{equotation}

    “和”做介词时,“和”连接的成分“副总经理”作状语,“背着”的是“总经理”一个人;“和”
    作连词时,“和”连接的成分“副总经理”作宾语,“背着”的是“总经理和副总经理”两人。
    这样,句子就有了歧义。

  \item 断句不同,带来多种理解。例如:

    \begin{equotation}
      县里的通知说,让赵乡长本月15日前去汇报。
    \end{equotation}

    “15日前去”既可以停顿成“15日前/去”,也可以停顿成“15日/前去”,不同停顿,造成歧义。

  \item 修饰两可。例如:

    \begin{equotation}
      今年4月23日,全国几十个报社的编辑记者来到国家图书馆,参观展览,聆听讲座,
      度过了一个很有意义的“世界阅读日”。
    \end{equotation}

    “全国几十个报社的编辑记者”可作两种理解:“全国/几十个报社/的编辑记者”;“全国
    /几十个/报社的编辑记者”。

\end{enumerate}

\subsubsection{不合逻辑}

\begin{enumerate}
  \item 不合事理。例如:

    \begin{equotation}
      当前某些引起轰动的影视作品,也许在两年后,甚至五年以后就会被人遗忘得一干二净。
    \end{equotation}

    “甚至”表递进关系。“被人遗忘得一干二净”应该是“两年”比“五年”更进一层,所以应
    把“两年”与“五年”互换一下。

  \item 概念分类不合逻辑。种属概念不能并列,交叉关系概念不能并列,非同一范畴的
    概念不能并列。例如:

    \begin{equotation}
      市教委要求,各学校学生公寓的生活用品和床上用品由学生自主选购,不得统一配备。
    \end{equotation}

    “生活用品”包括“床上用品”,二者不能并列。

  \item 否定不当。双重否定表肯定,三重否定表否定,反问语气相当于一次否定。例如:

    \begin{equotation}
      睡眠三忌:一忌睡前不可恼怒,二忌睡前不可饱食,三忌卧处不可当风。
    \end{equotation}

    “忌”和“不可”构成双重否定,表意不当,应去掉三个“不可”。

  \item 表述前后矛盾。例如:

    \begin{equotation}
      由北京人民艺术剧院复排的大型历史剧《蔡文姬》定于5月1日在首都剧场上演,日
      前正在紧张的排练之中。
    \end{equotation}

    “日前”即“前几天”,表示时间已经过去。“正”表示正在进行,二者在时间上相互冲突。

  \item 主客颠倒。例如:

    \begin{equotation}
      鸦片战争以来的中国近代史,对于大多数中学生是比较熟悉的,重大的历史事件都
      能说得一清二楚。
    \end{equotation}

    “鸦片战争以来的中国近代史,对于大多数中学生”主客颠倒,应改为“大多数中学生对
    于鸦片战争以来的中国近代史”。

  \item 强加因果。例如:

    \begin{equotation}
      周谷城先生早年就投身于轰轰烈烈的五四运动,所以最终成为蜚声海内外的著名学
      者、历史学家。
    \end{equotation}

    投身五四运动和成为著名学者之间,并没有因果关系。另外,“学者、历史学家”因为
    有包含关系,不能并列。

\end{enumerate}
